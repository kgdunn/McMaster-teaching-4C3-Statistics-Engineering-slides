\begin{frame}\frametitle{Least squares models in context ...}
	\begin{itemize}
		\item	A new development: we start looking at more than one variable at a time.
		\item	We will use confidence intervals and visualization heavily though
	\end{itemize}
	\vspace{24pt}
	\begin{itemize}
		\item	Some of the most important classes are this section: we develop the base for DOE and almost all modelling tools you will see in your career.
	\end{itemize}
\end{frame}

\begin{frame}\frametitle{Usage examples}
	\begin{itemize}
		\item	\textbf{Quantify} relationship between 2 variables:
		\vspace{12pt}
		\begin{itemize}
			\item	\emph{Manager}: How does yield from the lactic acid batch fermentation relate to the purity of sucrose?
			\item	\emph{Engineer}: The yield can be predicted from sucrose purity with an error of plus/minus 8\%
			\item	\emph{Manager}: And how about the relationship between yield and glucose purity?
			\item	\emph{Engineer}: Over the range of our historical data, there is no discernible relationship.
		\end{itemize}
	\end{itemize}
\end{frame}

\begin{frame}\frametitle{Usage examples}
	\begin{itemize}
		\item	\textbf{Assess} relationship between 2 variables:
		\vspace{12pt}
		\begin{itemize}
			\item	\emph{Engineer 1}: the theoretical equation for the melt index is non-linearly related to the viscosity
			\item	\emph{Engineer 2}: the linear model does not show any evidence of that, but the model's prediction ability does improve slightly when we use a non-linear transformation in the least-squares model.
		\end{itemize}
	\end{itemize}
\end{frame}

\begin{frame}\frametitle{Where are we headed}
	\begin{itemize}
		\item	You've seen the least squares model hundreds of times:
		
		\begin{exampleblock}{}
			$$\mathrm{y} = \beta_0 + \beta_1 \mathrm{x} + \varepsilon$$
		\end{exampleblock}

		\begin{itemize}
			\item	What are confidence intervals for $\beta_0$ and $\beta_1$?
			\item	Why do we even care about these confidence intervals?
			\item	What does $R^2$ really mean?
			\item	How to judge predictions from the model? $\hat{y} \pm \_\_\_$
		\end{itemize}
	\end{itemize}
\end{frame}

\begin{frame}\frametitle{What we will cover}
	\begin{enumerate}
		\item	Covariance
		\item	Correlation
		\item	Bivariate least squares (2 variables)
		\item	Multiple linear regression (2+ variables)
		\item	Advanced linear modelling topics
	\end{enumerate}
	\begin{center}
		\includegraphics[width=\textwidth]{\imagedir/mindmaps/least-squares-section-mapping.png}
	\end{center}
\end{frame}

\begin{frame}\frametitle{References and readings}
	\begin{itemize}
		\item	\textbf{Recommended}: John Fox, \emph{Applied Regression Analysis and Generalized Linear Models}
		\item	\textbf{Recommended}: Draper and Smith, \emph{Applied Regression Analysis}
		\item	Box, Hunter and Hunter, \emph{Statistics for Experimenters}, portions of Chapter 10 (2nd edition)
		\item	Montgomery and Runger, \emph{Applied Statistics and Probability for Engineers}
	\end{itemize}
\end{frame}

\begin{frame}\frametitle{Covariance: by example}
	
	\begin{exampleblock}{}
		What does the term {\color{purple}{covariance}} mean to you?
	\end{exampleblock}

	
	\begin{itemize}
		\item	Consider measurements from a gas cylinder: temperature (K) and pressure (kPa).
		\item	Ideal gas law applies under moderate condition: $pV = nRT$
		\begin{itemize}
			\item	Fixed volume, $V = 20 \times 10^{-3} \text{m}^3$ = 20 L
			\item	Moles of gas, $n = 14.1$ mols of chlorine gas, (1 kg gas)
			\item	Gas constant, $R = 8.314$ J/(mol.K)
		\end{itemize}
	\end{itemize}
	\begin{itemize}
		\item	Simplify the ideal gas law to: $p=\beta_1 T$, where $\beta_1 = \dfrac{nR}{V} > 0$
	\end{itemize}
\end{frame}

\begin{frame}\frametitle{Covariance}

	Raw data measured:
	\begin{table}
		[ht] \centering
		\begin{tabular}
			{l c c c} \hline\hline &Cylinder & Cylinder & Room \\
			[0.5ex] &temperature (K) & pressure (kPa) & humidity (\%) \\
			[0.5ex] \hline & 273& 1600& 42\\
			& 285& 1670& 48\\
			& 297& 1730& 45\\
			& 309& 1830& 49\\
			& 321& 1880& 41\\
			& 333& 1920& 46\\
			& 345& 2000& 48\\
			& 357& 2100& 48\\
			& 369& 2170& 45\\
			& 381& 2200& 49\\
			\hline \textbf{Mean}& 327& 1910& 46.1\\
			\textbf{Variance} & 1320 & 43267 & 8.1\\
			\hline
		\end{tabular}
	\end{table}
\end{frame}

\begin{frame}\frametitle{Covariance}
	\begin{itemize}
		\item	Formal definition:
	\end{itemize}
	$$ \text{Cov}\left\{x, y\right\} = \mathcal{E}\left\{ (x - \overline{x}) (y - \overline{y})\right\} \qquad \text{where} \qquad \mathcal{E}\left\{ z \right\} = \overline{z} $$

	\begin{enumerate}
		\item	Calculate {\color{purple}{deviation variables}}: $T - \overline{T}$ and $p - \overline{p}$
		\begin{itemize}
			\item	Subtracting off mean centers the vector at zero
		\end{itemize}
		\item	Multiply the {\color{purple}{centered vectors}}: $(T - \overline{T}) (p - \overline{p})$
		\begin{itemize}
			\item	\texttt{16740 10080 5400 1440 180 60 1620 5700 10920 15660}
		\end{itemize}
		\item	Calculate the expected value (mean): 6780
		\item	Covariance has units: 6780 [K.kPa]
		\begin{itemize}
			\item	Product of the individual units
			\item	\emph{Awkward units}!
		\end{itemize}
	\end{enumerate}

	\vspace{6pt}
	\begin{itemize}
		\item	Covariance between temperature and humidity is: 202 [K.\%]
		\begin{itemize}
			\item	\emph{Awkward units}!
		\end{itemize}
	\end{itemize}
\end{frame}

\begin{frame}\frametitle{Covariance}
	\begin{itemize}
		\item	Covariance with itself is the variance:
		$$\text{Cov}\left\{x, x\right\} = \mathcal{V}(x) = \mathcal{E}\left\{ (x - \overline{x}) (x - \overline{x})\right\}$$

		\item	(Co)variance of a centered vector = (co)variance of the uncentered vector
		\item	What does $x^Tx$ represent?
		\item	And $x^Ty$?
	\end{itemize}
	\begin{itemize}
		\item	Go through the examples in the notes
		\begin{itemize}
			\item	$x$=hours worked per week and $y$=take home pay
			\item	$x$=age of married partner 1 and $y$=age of married partner 2
			\item	$x$=cigarettes smoked per month and $y$=age at death
			\item	$x$=temperature on top tray of distillation column and $y$=top product purity
		\end{itemize}
	\end{itemize}
	\begin{itemize}
		\item	Covariance and correlation describe overall tendency, outliers don't follow the expected pattern.
	\end{itemize}
\end{frame}

\begin{frame}\frametitle{Correlation: measures ``co-relationship''}
	\begin{itemize}
		\item	(Co)variance depends on units: e.g. different covariance for grams vs kilograms
		\item	Correlation removes the scaling effect:
	\end{itemize}

	$$r(x, y) = \dfrac{\mathcal{E}\left\{ (x - \overline{x}) (y - \overline{y})\right\}}{\sqrt{\mathcal{V}\left\{x\right\}\mathcal{V}\left\{y\right\}}} = \dfrac{\text{Cov}\left\{x, y\right\}}{\sqrt{\mathcal{V}\left\{x\right\}\mathcal{V}\left\{y\right\}}}$$
	\begin{itemize}
		\item	Divides by the units of $x$ and $y$: dimensionless result
		\item	$-1 \leq r(x,y) \leq 1$
	\end{itemize}
	\begin{itemize}
		\item	Gas cylinder example:
		\begin{itemize}
			\item	\emph{r}\,(temperature, pressure) = 0.997
			\item	\emph{r}\,(pressure, temperature) = 0.997
			\item	\emph{r}\,(temperature, humidity) = 0.380
		\end{itemize}
	\end{itemize}
\end{frame}

\begin{frame}\frametitle{Correlation examples}
	\begin{center}
		\includegraphics[height=0.9\textheight]{\imagedir/least-squares/correlation-calculation.png}
	\end{center}
\end{frame}

\begin{frame}\frametitle{Correlation examples}
	\begin{center}
		\includegraphics[width=\textwidth]{\imagedir/visualization/sbarlow_ROB_Twitter_image.png}
	\end{center}
	We say the two trends are correlated from 2001 to 2008: but by how much? What happens to the correlation from 2009 onwards?
\end{frame}

\begin{frame}\frametitle{Correlation examples}
	\begin{center}
		\includegraphics[width=\textwidth]{\imagedir/least-squares/Correlation_examples.png}
	\end{center}
	\see{\href{http://en.wikipedia.org/wiki/File:Correlation_examples.png}{Wikipedia}: \texttt{Pearson product-moment correlation coefficient}}
\end{frame}

\begin{frame}\frametitle{Formal definitions for covariance and correlation {\small (you've seen this before)}}
	\begin{itemize}
		\item	$\mathcal{E}\{x\} = \overline{x}$
		\item	$\mathcal{E}\{x+y\} = \mathcal{E}\{x\} + \mathcal{E}\{y\} = \overline{x} + \overline{y}$
		\item	$\mathcal{V}\{x\} = \mathcal{E}\{(x-\overline{x})^2\}$
		\item	$\mathcal{V}\{cx\} = c^2\mathcal{V}\{x\}$
		\item	$\mathcal{V}\{x+y\} \neq \mathcal{V}\{x\} + \mathcal{V}\{y\}$, in general
		\item	$\mathcal{V}\{x+y\} = \mathcal{V}\{x\} + \mathcal{V}\{y\}$, only if $x$ and $y$ are independent
		\item	Go through the other definitions on your own. They build on these.
	\end{itemize}
\end{frame}

\begin{frame}\frametitle{Introduction to least squares: relating 2 variables}
	\begin{itemize}
		\item	Linear least squares model is a useful tool
		\item	It is the basis for a number of algorithms in data analysis
		\begin{itemize}
			\item	DOE
			\item	Latent variable methods
			\item	Almost all other advanced methods build on its concepts
		\end{itemize}
	\end{itemize}
	\begin{itemize}
		\item	We consider only 2 variables: $\mathrm{x}$ and $\mathrm{y}$.
	\end{itemize}
	\begin{enumerate}
		\item	Model definition
		\item	Building (fitting) the model
		\item	Interpret model parameters and model outputs
		\item	Software
	\end{enumerate}
\end{frame}

\begin{frame}\frametitle{Model definition}

	We have 2 vectors of data, $\mathrm{x}$ and $\mathrm{y}$. Presume the relationship between them:

	$$
	\begin{array}{rcl}
		\mathcal{E}\left\{\mathrm{y}\right\} &=& \beta_0 + \beta_1 \mathrm{x} \\
		\mathrm{y} &=& \beta_0 + \beta_1 \mathrm{x} + \varepsilon
	\end{array}
	$$
	\begin{itemize}
		\item	$\beta_0$, $\beta_1$ and $\varepsilon$: are \emph{population} parameters,
		\item	$\varepsilon$ term:
		\begin{itemize}
			\item	unmodelled components of the linear model
			\item	measurement error
			\item	other random variation
		\end{itemize}
	\end{itemize}

	\textbf{Important}: error is from $y$, not from $x$.
\end{frame}

\begin{frame}\frametitle{Model definition}

	Use \textbf{any method} to obtain an estimate of these parameters:
	\begin{itemize}
		\item	$b_0 = \hat{\beta_0}$
		\item	$b_1 = \hat{\beta_1}$
		\item	$e = \hat{\varepsilon}$
	\end{itemize}

	So we can write:

	$$
	\begin{array}{rcl}
		y_i &=& b_0 + b_1 x_i + e_i \\
		\hat{y}_i &=& b_0 + b_1 x_i
	\end{array}
	$$
\end{frame}

\begin{frame}\frametitle{Model definition}
	\begin{itemize}
		\item	Calculate the estimates $b_0$ and $b_1$
		\item	We want: $\mathcal{E}\left\{e\right\} = 0$
		\item	Then for a new $x$-observation, $x_i$, we can predict $\hat{y}_i =b_0 + b_1x_{i}$
	\end{itemize}
	\begin{center}
		\includegraphics[height=0.67\textheight]{\imagedir/least-squares/least-squares-picture.png}
	\end{center}
\end{frame}

\begin{frame}\frametitle{Minimizing errors}

	How do we calculate $b_0$ and $b_1$?

	\textbf{Given}: that we have $n$ pairs of data collected: $(x_i, y_i)$

	\textbf{Aim}: make the $e_i$ values small in some way and hopefully also have $\mathcal{E}\left\{e\right\} = 0$

	\vspace{24pt}
	Some options we could use:
	\begin{enumerate}
		\item	$\sum_{i=1}^{n}{(e_i)^2}$: the standard least squares objective function
		\item	$\sum_{i=1}^{n}{(e_i)^4}$
		\item	sum of perpendicular distances to the model's line
		\item	$\sum_{i=1}^{n}{\|e_i\|}$ (aka least absolute deviations, $\ell$-1 norm problem)
		\item	$\text{median}\left\{ e_i^2 \right\}$: least median of squared errors model (one type of robust least squares model)
	\end{enumerate}
\end{frame}

\begin{frame}\frametitle{Why minimize the sum of squares ?}

	The least squares model:
	\begin{itemize}
		\item	has the lowest possible variance for $b_0$ and $b_1$ when certain assumptions are met (more later)
		\item	computationally tractable by hand
		\item	very fast on computers
		\item	easy to prove various mathematical properties
		\item	intuitive: penalize deviations quadratically
	\end{itemize}

	Other forms: multiple solutions, unstable, high variance solutions, mathematical proofs are difficult
\end{frame}

\begin{frame}\frametitle{Solving the model}

	Has to be an optimization problem: \textbf{\emph{minimizing}} the sum of squared errors
	\begin{itemize}
		\item	Easy to solve! Unconstrained optimization problem (know this from an undergrad optimization course)
	\end{itemize}

	$$
	\begin{array}{rcl}
		\min_{\displaystyle b_0, b_1} f(b_0, b_1) &=& \sum_{i=1}^{n}{(e_i)^2} \\
		\\
		&=& \sum_{i=1}^{n}{\left(y_i - b_0 - b_1 x_i\right)^2}
	\end{array}
	$$
\end{frame}

\begin{frame}\frametitle{Solving the model: grid search}

	Let's come back to the gas cylinder example: $p = \beta_0 + \beta_1 T$
	\begin{itemize}
		\item	We know that $\beta_0 = 0$ from theoretical principles
		\item	Solve for $\beta_1$ by trial and error. Initial guess?
		\item	{\small $b_1 \approx \beta_1 = \dfrac{nR}{V} = \dfrac{(14.1 \,\, \text{mol})(8.314 \,\,\text{J/(mol.K)})}{20 \times 10^{-3} \,\,\text{m}^3} = 5.861 \,\,\text{kPa/K}$}
	\end{itemize}
	\vspace{12pt}
	\begin{enumerate}
		\item	Construct equally spaced points between 5.0 and 6.5,
		\item	Set $b_1 = $ guessed value
		\item	Calculate the objective function
		\item	Plot $b_1$ value vs objective function
	\end{enumerate}
\end{frame}

\begin{frame}\frametitle{Solving the model: grid search}
	\begin{center}
		\includegraphics[height=0.9\textheight]{\imagedir/least-squares/cylinder-case-study-objective.png}
	\end{center}
\end{frame}

\begin{frame}\frametitle{Solving the model: grid search in 2 variables}
	\begin{center}
		\includegraphics[height=0.7\textheight]{\imagedir/least-squares/least-squares-objective-function-annotated.png}
	\end{center}
	\begin{itemize}
		\item	objective function shape is a bowl
		\item	a unique minimum can always be found
		\item	because the objective function is convex
	\end{itemize}
\end{frame}

\begin{frame}\frametitle{Solving the model: analytically}
	\begin{block}{Least squares objective function}
		$$ f(b_0, b_1) = \sum_{i=1}^{n}{\left(y_i - b_0 - b_1 x_i\right)^2} $$
	\end{block}

	At the optimal point we know that:
	\begin{itemize}
		\item	The partial derivatives wrt $b_0$ and $b_1$ are zero
		\item	This represents 2 equations in 2 unknowns
		\item	Take the 2nd derivative to confirm that the optimum is indeed a minimum $\left(\dfrac{
		\partial^2 f}{
		\partial^2 b_0} > 0 \quad \text{and} \quad \dfrac{
		\partial^2 f}{
		\partial^2 b_1} > 0\right)$
	\end{itemize}
\end{frame}

\begin{frame}\frametitle{Solving the model: analytically}

	$$
	\begin{array}{rcl}
		\dfrac{\partial f(b_0, b_1)}{\partial{b_0}} &=& -2 \sum_i^{n}{(y_i - b_0 - b_1 x_i)} = 0 \\
		\dfrac{\partial f(b_0, b_1)}{\partial{b_1}} &=& -2 \sum_i^{n}{(x_i)(y_i - b_0 - b_1 x_i)} = 0 \\
	\end{array}
	$$

	\vspace{12pt}
	Least squares optimum for $f(b_0, b_1)$:

	$$
	\begin{array}{rcl}
		b_0 &=& \overline{\mathrm{y}} - b_1\overline{\mathrm{x}} \\
		b_1 &=& \dfrac{ \sum_i{\left(x_i - \overline{\mathrm{x}}\right)\left(y_i - \overline{\mathrm{y}}\right) } }{ \sum_i{\left( x_i - \overline{\mathrm{x}}\right)^2} }
	\end{array}
	$$
\end{frame}

\begin{frame}\frametitle{Remarks}
	\begin{block}{Least squares solution}
		$$
			\begin{array}{rcl}
				b_0 &=& \overline{\mathrm{y}} - b_1\overline{\mathrm{x}} \\
				b_1 &=& \dfrac{ \sum_i{\left(x_i - \overline{\mathrm{x}}\right)\left(y_i - \overline{\mathrm{y}}\right) } }{ \sum_i{\left( x_i - \overline{\mathrm{x}}\right)^2} } = \dfrac{\left(\mathrm{x} - \overline{\mathrm{x}}\right)'\left(\mathrm{y} - \overline{\mathrm{y}}\right)}{\left(\mathrm{x} - \overline{\mathrm{x}}\right)'\left(\mathrm{x} - \overline{\mathrm{x}}\right)}
			\end{array}
		$$
	\end{block}
	\begin{enumerate}
		\item	$\sum_i{e_i} = 0$
		\item	The straight line equation passes through $(\overline{\mathrm{x}}, \overline{\mathrm{y}})$ without error
		\item	Prove to yourself that $\sum_i{(x_i e_i)} = \mathrm{x}^T\mathrm{e} = 0$
		\begin{itemize}
			\item	The residuals are uncorrelated with the input variables, $\mathrm{x}$
			\item	There is no information in the residuals that is in the $\mathrm{x}$'s
		\end{itemize}
		\item	Prove and interpret that $\sum_i{(\hat{y}_i e_i)} = \hat{\mathrm{y}}^T\mathrm{e} = 0$
		\begin{itemize}
			\item	The fitted values are uncorrelated with the residuals
			\item	Use this in the ANOVA table (so we can write \texttt{RegSS} $\perp$ \texttt{RSS})
		\end{itemize}
		\item	Estimate of $b_0$ depends on $b_1$: the estimates are correlated
	\end{enumerate}
\end{frame}

\begin{frame}\frametitle{Questions}
	\begin{block}{Least squares solution}
			$$
			\begin{array}{rcl}
				b_0 &=& \overline{\mathrm{y}} - b_1\overline{\mathrm{x}} \\
				b_1 &=& \dfrac{ \sum_i{\left(x_i - \overline{\mathrm{x}}\right)\left(y_i - \overline{\mathrm{y}}\right) } }{ \sum_i{\left( x_i - \overline{\mathrm{x}}\right)^2} } = \dfrac{\left(\mathrm{x} - \overline{\mathrm{x}}\right)'\left(\mathrm{y} - \overline{\mathrm{y}}\right)}{\left(\mathrm{x} - \overline{\mathrm{x}}\right)'\left(\mathrm{x} - \overline{\mathrm{x}}\right)}
			\end{array}
			$$
	\end{block}
	\begin{enumerate}
		\item	Units of $b_1$ ?
		\begin{itemize}
			\item	The units of $\mathrm{y}$ divided by the units of $\mathrm{x}$
		\end{itemize}
		\item	Gas cylinder example. Let $\hat{p}_i = b_0 + b_1 T_i$:
		\begin{itemize}
			\item	What is the interpretation of coefficient $b_1$?
			\item	What is the interpretation of coefficient $b_0$?
		\end{itemize}
		\item	How could the denominator term for $b_1$ equal zero? And what would that mean?
		\begin{itemize}
			\item	As long as there is variation in the $x$-data that we will obtain a solution
		\end{itemize}
	\end{enumerate}
\end{frame}

\begin{frame}\frametitle{Example}
	\begin{columns}
		\column{5cm}
			\begin{enumerate}
				\item	Calculate model parameter estimates: $y = b_0 + b_1 x$ from the given data
				\item	Calculate predicted value $\hat{y}_i$ when $x_i = 5.5$
			\end{enumerate}
			\begin{center}
				\includegraphics[width=\textwidth]{\imagedir/least-squares/show-anscombe-problem-1.png}
			\end{center}
		\column{5cm}
			\begin{center}
				\includegraphics[height=0.8\textheight]{\imagedir/least-squares/regression-exercise.png}
			\end{center}
	\end{columns}
\end{frame}

\begin{frame}\frametitle{Least squares model \textcolor{myOrange}{analysis}}

	Assume we have calculated values for $b_0$ and $b_1$ from the data. Now:
	\begin{enumerate}
		\item	How well does the model perform?
		\item	What part of the data is error
		\begin{itemize}
			\item	noise
		\end{itemize}
		\item	What part of the data is systematic
		\begin{itemize}
			\item	signal
		\end{itemize}
		\item	Confidence interval for the model coefficients: $b_0$ and $b_1$
		
		\vspace{12pt}
		For example:
		\begin{itemize}
			\item	intercept: $-2.3 \leq \beta_0 \leq +3.2$
			\item	slope = rate constant (1/seconds): $0.55 \leq \beta_1 \leq 1.07$
		\end{itemize}
		\item	Prediction interval for the $y$-variable
		\begin{itemize}
			\item	e.g. the predicted yield is $8 \pm 1.7$ kg
		\end{itemize}
	\end{enumerate}

	\vspace{12pt}
	But, we need to \textbf{make assumptions} about the data first. We will get to points 4 and 5 soon. Let's look at points 1, 2, and 3 next.
\end{frame}

\begin{frame}\frametitle{The variance breakdown}

	\emph{Recall}: \textbf{life is pretty boring without variability}
	\begin{center}
		\includegraphics[width=0.5\textwidth]{\imagedir/concepts/variation/variation-none.png}
	\end{center}

	Analysis of variance (ANOVA) - just a tool to show the breakdown of variability in the $\mathrm{y}$ vector: $\mathrm{y} = b_0 + b_1 \mathrm{x} + \mathrm{e}$
	\begin{enumerate}
		\item	doing nothing, no model: implies $\hat{y} = b_0 = \overline{y}$ (intercept only)
		\item	the model: $\hat{y}_i = b_0 + b_1 x_i$ (intercept \emph{plus} slope)
		\item	then, how much variance is left over in the errors $e_i$?
	\end{enumerate}
	\vspace{12pt}
	All 3 add up to the total variance. Note that:
	\begin{itemize}
		\item	variance is quantified as {\color{orange}{deviation from the mean}}
		\item	Variance for first portion is 0, by definition
		\item	Total variance of $y$ = model's variance + error variance
		\item	This is why it was important that $\sum_i{(\hat{y}_i e_i)} = 0$ (so we can sum them independently)
	\end{itemize}
\end{frame}

\begin{frame}\frametitle{The variance breakdown: geometrically}
	\begin{center}
		\includegraphics[width=0.95\textwidth]{\imagedir/least-squares/ANOVA-graphically.png}
	\end{center}
\end{frame}

\begin{frame}\frametitle{The variance breakdown: geometrically}

	For a fixed value of $x_i$, and for any $y_i$ (above or below):

	$$
	\begin{array}{rcl}
		(y_i - \overline{\mathrm{y}}) &=& (\hat{y}_i - \overline{\mathrm{y}}) + (y_i - \hat{y}_i) \\
		(y_i - \overline{\mathrm{y}})^2 &=& (\hat{y}_i - \overline{\mathrm{y}})^2 + 2(\hat{y}_i - \overline{\mathrm{y}})(y_i - \hat{y}_i) + (y_i - \hat{y}_i)^2 \\
		\sum{(y_i - \overline{\mathrm{y}})^2} &=& \sum{(\hat{y}_i - \overline{\mathrm{y}})^2} + \sum{(y_i - \hat{y}_i)^2} \\
		\text{Total SS (TSS)} &=& \text{Regression SS (RegSS)} + \text{Residual SS (RSS)}
	\end{array}
	$$
	\begin{center}
		\includegraphics[width=.6\textwidth]{\imagedir/least-squares/ANOVA-graphically.png}
	\end{center}
\end{frame}

\begin{frame}\frametitle{The variance breakdown: table form}

	\newpage
	\begin{table}
		[ht] \centering
		\begin{tabular}
			{lllll} \hline\ \textbf{Variance of} 	& \textbf{Distance} 					&\textbf{DOF} 	&\textbf{SSQ} 	&\textbf{Mean square}* \\
			\hline Regression 						& $\hat{y}_i - \overline{\mathrm{y}}$ 	& $k$ ($k$=2) 	& RegSS 		& RegSS/$k$ \\
			Residuals 								& $y_i - \hat{y}_i$ 					& $n-k$ 		& RSS 			& RSS/($n-k$) \\\hline \hline 
			Total 									& $y_i - \overline{\mathrm{y}}$ 		& $n$ 			& TSS 			& TSS/$n$ \\
			\hline
		\end{tabular}
	\end{table}
	* {\color{purple}{mean square}} = ``variance''
	%\begin{itemize}
	%	\item	DOF = {\color{purple}{degrees of freedom}} = number of independent pieces of information involving the $n$ points uses to calculate the {\color{purple}{sum of squares}} (SSQ)
	%\end{itemize}
	\vspace{18pt}
	\begin{itemize}
		\item	Regression SS (RegSS) = $\sum{(\hat{y}_i - \overline{\mathrm{y}})^2}$ 
			\begin{itemize}
				\item	from the mean to the prediction
			\end{itemize}
			
		\vspace{12pt}
		\item	Residual SS (RSS) = $\sum{(y_i - \hat{y}_i)^2}$
		\begin{itemize}
			\item	from the prediction to the data point
		\end{itemize} 
		
		\vspace{12pt}
		\item	Total SS (TSS) = $\sum{(y_i - \overline{\mathrm{y}})^2}$ 
		\begin{itemize}
			\item	from the data point to the mean
		\end{itemize}
	\end{itemize}
\end{frame}

\begin{frame}\frametitle{Exercise}

	For each of these cases:
	\begin{enumerate}
		\item	$y_i = b_0 + e_i$, i.e. where $b_0 = \overline{y}$ and $b_1 = 0$
		\item	$y_i = b_0 + b_1 x_i + e_i$, for any values of $b_0$ and $b_1$, and the models fits the data without error
	\end{enumerate}

	\vspace{12pt}
	Do the following:
	\begin{itemize}
		\item	draw a generic plot
		\item	create an ANOVA table with fake values
		\item	write down the value of the ratio $\dfrac{\text{RegSS}}{\text{TSS}}$
	\end{itemize}
\end{frame}

\begin{frame}\frametitle{Exercise: what did we learn}
	\begin{itemize}
		\item	No linear model (just noise around a baseline): 
		$$y_i = \overline{y} + e_i = b_0 + e_i$$
		
		\item	ratio = $\dfrac{\text{RegSS}}{\text{TSS}} = 0$
		\item	Perfect predictions: $\dfrac{\text{RegSS}}{\text{TSS}} = 1$
		\item	$\dfrac{\text{RegSS}}{\text{TSS}} = R^2$
	\end{itemize}
\end{frame}

\begin{frame}\frametitle{Derivation of $R^2$}
	\begin{itemize}
		\item	$R^2 = \dfrac{\text{RegSS}}{\text{TSS}} = \dfrac{\sum_i{ \left(\hat{y}_i - \overline{\mathrm{y}}\right)^2}}{\sum_i{ \left(y_i - \overline{\mathrm{y}}\right)^2}}$
		\item	RegSS = variance we can explain with the model
		\item	TSS = total variance we started off with
		\item	TSS = RegSS + RSS, so $R^2 = 1-\dfrac{\text{RSS}}{\text{TSS}}$ also
	\end{itemize}

	If
	\begin{itemize}
		\item	$R^2 = 0$: requires $\hat{y}_i = \overline{\mathrm{y}}$: predicting a flat line = $\mathrm{y}$ data
		\item	$R^2 = 1$: requires $\hat{y}_i = y_i$: no error in predictions
	\end{itemize}
\end{frame}

\begin{frame}\frametitle{Meaning of $R^2$}

	$R^2$ name: square of the correlation between $\mathrm{x}$ and $\mathrm{y}$.
	\begin{exampleblock}{}
		$$ r(x, y) = \dfrac{\mathcal{E}\left\{ (x - \overline{x}) (y - \overline{y})\right\}}{\sqrt{\mathcal{V}\left\{x\right\}\mathcal{V}\left\{y\right\}}} = \dfrac{\text{Cov}\left\{x, y\right\}}{\sqrt{\mathcal{V}\left\{x\right\}\mathcal{V}\left\{y\right\}}} $$
	\end{exampleblock}

	\vspace{12pt}
	\begin{itemize}
		\item	$-1 \leq r(x,y) \leq +1$
		\item	$R^2 = \left[r(x,y)\right]^2$
		\item	$0 \leq R^2 \leq +1 $
	\end{itemize}
	
	\vspace{12pt}
	Just tells us how far we are between predicting from \textbf{no variation} to \textbf{predictions without error}
\end{frame}

\begin{frame}\frametitle{Meaning of $R^2$}
	\begin{itemize}
		\item	$R^2$ is computed from the data used to build the model
		\item	{\color{myOrange}{Note}}:  $R^2$ can be calculated without even fitting the model (i.e. without finding $b_0$ and $b_1$)
		\item	\textbf{No guarantee at all} as to future performance
		\item	It is widely \textbf{abused} as a way to measure ``\emph{how good is my model}''
		\begin{enumerate}
			\item	``the $R^2$ value is really high, 90\%, this is a good model''
			\item	``Wow, that's a really low $R^2$, this model can't be right - it's no good''
		\end{enumerate}
	\end{itemize}
\end{frame}

\begin{frame}\frametitle{Meaning of $R^2$}
	\begin{exampleblock}{}
		How \textbf{good} a model is \emph{for a particular purpose} is almost never related to the $R^2$ value.
	\end{exampleblock}

	\vspace{12pt}
	Goodness of a model is better judged by:
	\begin{itemize}
		\item	your engineering judgment: does the interpretation of model parameters make sense?
		\item	use the {\color{purple}{standard error}}, $S_E$ (next section)
		\item	use testing data to verify the model's predictive performance,
		\item	using cross-validation/randomization tools (more later)
	\end{itemize}

	$R^2$ can be arbitrarily inflated by adding terms to the model. Use adjusted $R^2$ to account for this:

	$$ R^2_\text{adj} = 1 - \dfrac{\text{RSS}/(n-k)}{\text{TSS}/(n-1)} $$

	where $k=2$ for the case of estimating a model $y_i = b_0 + b_1 x_i$
\end{frame}

\begin{frame}\frametitle{Judging the {\color{purple}{standard error}}, $S_E$}

	A better way to judge model's predictive performance:
	\begin{itemize}
		\item	mean square error = $\text{RSS}/(n-k)$
		\item	standard error = $S_E = \sqrt{\text{mean square error}} = \sqrt{\text{RSS}/(n-k)}$
		\item	The standard deviation of the errors, accounting for degrees of freedom
		\item	$S_E^2 = \frac{\displaystyle \text{RSS}}{\displaystyle n-k} = \dfrac{e^Te}{\displaystyle n-k}$
		\item	This is related to the model's objective function: minimizing RSS
	\end{itemize}

	Assume our model predicts
	\begin{itemize}
		\item	$\hat{y}_i$ = batch yield [kg]
		\item	$x_i$ = raw material purity
		\item	What does a standard error of 3.4 kg mean?
		\begin{itemize}
			\item	If errors are normally distributed:
			\begin{itemize}
				\item	About 2/3 of predictions lie within $\pm 3.4$ kg
				\item	About 95\% of predictions lie within $\pm 2 \times 3.4$ kg
			\end{itemize}
		\end{itemize}
	\end{itemize}
\end{frame}

\begin{frame}\frametitle{Confidence intervals for {\color{red}$\beta_0$ and $\beta_1$}}
	\begin{itemize}
		\item	We have made no assumptions about the data
		\item	We can calculate $b_0$ and $b_1$
		\item	We can calculate model predictions
	\end{itemize}
	
	
	\vspace{12pt}
	Now we want to know more. Let's learn from the data.
\end{frame}

\begin{frame}\frametitle{Assumptions required}
	\begin{itemize}
		\item	Model structure: $y_i = \beta_0 + \beta_1 x_i + \varepsilon_i$
		\item	$b_0 = \hat{\beta}_0$ and $b_1= \hat{\beta}_1$
	\end{itemize}
	\begin{block}{Assumption 1}
		\begin{center}
			Model is linear
		\end{center}
	\end{block}
	\begin{itemize}
		\item	Implies $\varepsilon$ is the error of $\mathrm{y}$ ($\beta_0 + \beta_1 \mathrm{x}$ terms are fixed)
		\item	Means your $\mathrm{x}$ variable has much less uncertainty than the $\mathrm{y}$ variable
	\end{itemize}
\end{frame}

\begin{frame}\frametitle{Assumptions required}
	\begin{block}{Assumption 2}
		\begin{center}
			Variance of \emph{y} is constant at all values of \emph{x}
		\end{center}
	\end{block}
	\begin{itemize}
		\item	Constant error variance assumption
		\item	Practice: variability of $\mathrm{y}$ often non-constant
		\begin{itemize}
			\item	measurement accuracy deteriorates at extreme conditions of x and y
		\end{itemize}
	\end{itemize}
	\begin{center}
		\includegraphics[width=0.65\textwidth]{\imagedir/least-squares/constant-error-variance.png}
	\end{center}
\end{frame}

\begin{frame}\frametitle{Assumptions required}
	\begin{block}{Assumption 3}
		\begin{center}
			Errors are normally distributed: $e_i \sim \mathcal{N}(0, \sigma_\varepsilon^2)$
		\end{center}
	\end{block}
	\begin{itemize}
		\item	Implies that $y_i \sim \mathcal{N}(\beta_0 + \beta_1x_i, \sigma_\varepsilon^2)$ (from first linearity assumption)
		\item	Practice: not always certain
		\item	But at least it is easy to test with a q-q plot
	\end{itemize}
\end{frame}

\begin{frame}\frametitle{Assumptions required}
	\begin{center}
		\includegraphics[width=\textwidth]{\imagedir/least-squares/residual-pattern-heavy-tails.png}
	\end{center}
\end{frame}

\begin{frame}\frametitle{Assumptions required}
	\begin{block}{Assumption 4}
		\begin{center}
			Each error is independent of the other
		\end{center}
	\end{block}
	\begin{itemize}
		\item	Often violated in practice
		\item	Observations taken in time on a slow processes
		\item	E.g. if you have a positive error now, your next sample is also likely to have a positive error (autocorrelation)
	\end{itemize}
\end{frame}

\begin{frame}\frametitle{Assumptions required}
	\begin{center}
		\includegraphics[width=\textwidth]{\imagedir/least-squares/residual-pattern-unmodelled-dynamics.png}
	\end{center}
\end{frame}

\begin{frame}\frametitle{Assumptions required}
	\begin{block}{Assumption 5}
		\begin{center}
			Assume $x$ are fixed and independent of the error
		\end{center}
	\end{block}
	\begin{itemize}
		\item	Closely tied to first assumption
		\item	Most engineering cases we sample the $\mathrm{x}$ and $\mathrm{y}$
		\item	This assumption requires that our error variance does not change as we sample $\mathrm{x}$
	\end{itemize}
\end{frame}

\begin{frame}\frametitle{Assumptions required}
	\begin{block}{Assumption 6}
		\begin{center}
			All $y_i$ values are independent of each other.
		\end{center}
	\end{block}
	\begin{itemize}
		\item	Closely tied to 4th assumption
		\item	Is the same as assumption 4 if the $\mathrm{x}$ are constant
		\item	Violated in cases where data are collected in time order and $y_i$ are autocorrelated
	\end{itemize}
\end{frame}

\begin{frame}\frametitle{Confidence intervals for {\color{red}$\beta_0$ and $\beta_1$}}

	To construct a confidence interval, we must know the distribution

	$$
	\begin{array}{lcr}
		b_0 \sim \mathcal{N}(\beta_0, \mathcal{V}\{\beta_0\}) &\qquad\text{and}\qquad& b_1 \sim \mathcal{N}(\beta_1,\mathcal{V}\{\beta_1\})
	\end{array}
	$$
	\begin{enumerate}
		\item	Create $z$-value for $b_0$ and $b_1$
		\item	Calculate the confidence interval for $\beta_0$ and $\beta_1$
	\end{enumerate}

	\vspace{12pt}
	\textbf{Objective}: Calculate a suitable $\mathcal{V}\{\beta_0\}$ and $\mathcal{V}\{\beta_1\}$
\end{frame}

\begin{frame}\frametitle{Variance of $b_1$}

	Details are in your notes. Strongly encouraged to go through the derivation.
	\begin{itemize}
		\item	Recall the standard error:
	\end{itemize}

	$$ S_E^2 = \mathcal{V}\{e_i\} = \mathcal{V}\{y_i\} = \dfrac{\sum{e_i^2}}{n-k} \qquad \text{or}\qquad S_E = \sqrt{ \dfrac{\sum{e_i^2}}{n-k} } $$
	\begin{itemize}
		\item	The variance of $b_1 = \mathcal{V}\{b_1\}$:
	\end{itemize}
	$$ \mathcal{V}\{b_1\} = \dfrac{S_E^2}{\sum_j{\left( x_j - \overline{\mathrm{x}} \right)^2}} $$
	\begin{itemize}
		\item	Apart from the numerator, how could you decrease the variance your model's $b_1$ coefficient?
		\begin{itemize}
			\item	Use samples that are far from the mean of the $\mathrm{x}$-data
			\item	Use more samples
		\end{itemize}
		\item	The numerator term: this is the standard error $S_E^2$ term we saw earlier.
	\end{itemize}
\end{frame}

\begin{frame}\frametitle{Variance of $b_0$}

	Variance of $b_0$:

	$$ \mathcal{V}\{b_0\} = S_E^2 \left(\dfrac{1}{N} + \dfrac{\overline{\mathrm{x}}^2}{\sum_j{\left( x_j - \overline{\mathrm{x}} \right)^2}} \right) $$

	\textbf{Summary of important equations}:
	\begin{enumerate}
		\item	$\mathcal{V}\{\beta_0\} \approx \mathcal{V}\{b_0\} = S_E^2(b_0) = S_E^2 \left(\dfrac{1}{N} + \dfrac{\overline{\mathrm{x}}^2}{\sum_j{\left( x_j - \overline{\mathrm{x}} \right)^2}} \right)$
		\item	$\mathcal{V}\{\beta_1\} \approx \mathcal{V}\{b_1\} = S_E^2(b_1) = \dfrac{S_E^2}{\sum_j{\left( x_j - \overline{\mathrm{x}} \right)^2}}$
		\item	$\mathcal{V}\{y_i\} = \mathcal{V}\{e_i\} \approx S_E^2 = \dfrac{\sum{e_i^2}}{n-k}$
	\end{enumerate}
	\begin{itemize}
		\item	$S_E$ standard deviation of the error (residuals)
		\item	$S_E(b_0)$ standard deviation of $b_0$
		\item	$S_E(b_1)$ standard deviation of $b_1$
	\end{itemize}
\end{frame}

\begin{frame}\frametitle{Confidence intervals for {\color{red}$\beta_0$ and $\beta_1$}}

	Finally! Construct confidence intervals for the least squares model parameters:

	$$
	\begin{array}{rcccl}
		- c_t &\leq& \dfrac{b_0 - \beta_0}{S_E(b_0)} &\leq & +c_t \\
		b_0 - c_t S_E(b_0) &\leq& \beta_0 &\leq& b_0 + c_t S_E(b_0)
	\end{array}
	$$

	\vspace{24pt}

	$$
	\begin{array}{rcccl}
		- c_t &\leq& \dfrac{b_1 - \beta_1}{S_E(b_1)} &\leq & +c_t\\
		b_1 - c_t S_E(b_1) &\leq& \beta_1 &\leq& b_1 + c_t S_E(b_1)
	\end{array}
	$$
	\begin{itemize}
		\item	Use $t$-distribution: we used estimates of the variance
	\end{itemize}
\end{frame}

\begin{frame}\frametitle{Example}

	Confidence interval for $\beta_0$ and $\beta_1$ from previous example. Already calculated:
	\begin{itemize}
		\item	$b_0$ = 3.0
		\item	$b_1$ = 0.5
		\item	$S_E$ = 1.237
		\begin{itemize}
			\item	\texttt{predictions = b0 + x*b1}
			\item	\texttt{8.001 7.000 9.501 7.501 8.501 10.001 6.00 5.000 9.001 6.500 5.501}
			\item	\texttt{error = y - predictions}
			\item	\texttt{SE = sqrt(sum(error)**2 / (N-2))}
		\end{itemize}
		\item	$c_t = 2.26$ at 95\% confidence
		\begin{itemize}
			\item	\texttt{qt(0.975, df=(11-2))}
		\end{itemize}
	\end{itemize}
\end{frame}

\begin{frame}\frametitle{Example}
	\begin{itemize}
		\item	$S_E^2(b_1) = \dfrac{S_E^2}{\sum_j{\left( x_j - \overline{\mathrm{x}} \right)^2}} = \dfrac{1.237^2}{110} = 0.0139; {\color{myRed}{S_E(b_1) = 0.1179}}$
		\item	{ $S_E^2(b_0) = S_E^2 \left(\dfrac{1}{N} + \dfrac{\overline{\mathrm{x}}^2}{\sum_j{\left( x_j - \overline{\mathrm{x}} \right)^2}} \right) = \left(\dfrac{1}{11} + \dfrac{9^2}{110} \right)1.237^2 = 1.266; \qquad {\color{myRed}{S_E(b_0) = 1.125}}$}
	\end{itemize}
	\begin{itemize}
		\item	For $\beta_0$:
	\end{itemize}
	$
	\begin{array}{rccclrcccl}
		- c_t &\leq& \dfrac{b_0 - \beta_0}{S_E(b_0)} &\leq & +c_t \\
		3.0 - 2.26 \times \sqrt{1.266} &\leq& \beta_0 &\leq& 3.0 + 2.26 \times \sqrt{1.266} \\
		0.457 &\leq& \beta_0 &\leq& 5.54
	\end{array}
	$
\end{frame}

\begin{frame}\frametitle{Example}
	\begin{itemize}
		\item	For $\beta_1$:
	\end{itemize}
	$
	\begin{array}{rccclrcccl}
		- c_t &\leq& \dfrac{b_1 - \beta_1}{S_E(b_1)} &\leq & +c_t \\
		0.5 - 2.26 \times \sqrt{0.0139} &\leq& \beta_1 &\leq& 0.5 + 2.26 \times \sqrt{0.0139}\\
		0.233 &\leq& \beta_1 &\leq& 0.767 \\
	\end{array}
	$
	\begin{center}
		\includegraphics[height=0.6\textheight]{\imagedir/least-squares/show-anscome-solution-marked.png}
	\end{center}
\end{frame}

\begin{frame}\frametitle{Software output}
	\begin{center}
		\includegraphics[width=\textwidth]{\imagedir/least-squares/software-output-lm.png}
	\end{center}
\end{frame}

\begin{frame}\frametitle{Example}
	\begin{center}
		\includegraphics[width=0.85\textwidth]{\imagedir/least-squares/cameras-road-deaths.jpg}
	\end{center}
	\vspace{-8pt}
	\begin{itemize}
		\item	What is the causal direction (line of reasoning) that the plot's author is wanting you to follow?
		\item	Interpret the software output: $y = 15.7 - (0.70) \text{cameras}$
		\vspace{-16pt}
		\begin{columns}[t]
			\column{0.10\textwidth}
			\column{0.45\textwidth}
				\begin{itemize}
					\item	$9.61 \leq \beta_0 \leq 21.8$
					\item	$-1.8 \leq \beta_1 \leq 0.36$
				\end{itemize}
			\column{0.45\textwidth}
				\begin{itemize}
					\item	$S_E = 10.9$
					\item	$R^2 = 0.075$
				\end{itemize}
		\end{columns}
		
	\end{itemize}
\end{frame}

\begin{frame}\frametitle{Example}

	Thermocouples produce an almost linear voltage (millivolt) response at different temperatures. (Data are on the course website). Temperatures usually recorded to an accuracy of $\pm 0.5$K with inexpensive thermocouples.
	\begin{columns}
		\column{5cm}
		\includegraphics[width=\textwidth]{\imagedir/least-squares/thermocouple.png} \column{5cm} 1. Calculate and interpret model parameter estimates, $y = b_0 + b_1 x$, from the given data
		\begin{itemize}
			\item	Model calculated is $\hat{T} = 278.6 + 135.3 V$. Interpretation?
		\end{itemize}

		2. Temperature prediction at 1.00 mV?

		3. The $R^2 = 0.996$. What can you say about the model?

		4. The $S_E = 3.9K$. Satisfied with the model's prediction ability?
	\end{columns}
\end{frame}

\begin{frame}\frametitle{Review so far}

	$$
	\begin{array}{rcl}
		\text{Population model}: \mathrm{y} &=& \beta_0 + \beta_1 \mathrm{x} + \varepsilon \\
		\text{Estimated model}: y_i &=& b_0 + b_1 x_i + e_i \\
		\\
	\end{array}
	$$

	Solve an optimization problem:

	$$
	\begin{array}{rcl}
		\min_{\displaystyle b_0, b_1} f(b_0, b_1) &=& \sum_{i=1}^{n}{(e_i)^2} \\
		&=& \sum_{i=1}^{n}{\left(y_i - b_0 - b_1 x_i\right)^2} \\
		\\
	\end{array}
	$$

	$$
	\begin{array}{rcl}
		b_0 &=& \overline{\mathrm{y}} - b_1\overline{\mathrm{x}} \\
		b_1 &=& \dfrac{ \sum_i{\left(x_i - \overline{\mathrm{x}}\right)\left(y_i - \overline{\mathrm{y}}\right) } }{ \sum_i{\left( x_i - \overline{\mathrm{x}}\right)^2} } \\
		\\
	\end{array}
	$$
	
	
	$S_E = \sqrt{\dfrac{\text{RSS}}{n-k}} = \sqrt{\dfrac{ e^Te}{n-k}} \qquad\qquad \sum{e_i}=0$
	
\end{frame}

\begin{frame}\frametitle{Review of assumptions}
	\begin{enumerate}
		\item	Model is linear
		\item	Variance of $y$ is constant at all values of $x$
		\item	Errors are normally distributed
		\item	Each error is independent of the other
		\item	Assume $x$ are fixed and independent of the error
		\item	All $y_i$ values are independent of each other
	\end{enumerate}
	\vspace{12pt}
	{\color{myGreen}{Fortunately}}: least squares is pretty ``{\color{purple}{robust}}'' to violations in these assumptions.
\end{frame}

\begin{frame}\frametitle{Analysis}
	If those above assumptions hold then:
	\begin{enumerate}
		\item	$\mathcal{V}\{b_0\} = S_E^2(b_0) = S_E^2 \left(\dfrac{1}{N} + \dfrac{\overline{\mathrm{x}}^2}{\sum_j{\left( x_j - \overline{\mathrm{x}} \right)^2}} \right)$
		\item	$\mathcal{V}\{b_1\} = S_E^2(b_1) = \dfrac{S_E^2}{\sum_j{\left( x_j - \overline{\mathrm{x}} \right)^2}}$
		\item	$\mathcal{V}\{y_i\} = \mathcal{V}\{e_i\} = S_E^2 = \dfrac{\sum{e_i^2}}{n-k}$
	\end{enumerate}
	\vspace{12pt}
	and then these are accurate estimates of the variance.\\
	We then go on to use those variances to calculate confidence intervals.
\end{frame}

\begin{frame}\frametitle{An approximate prediction interval for $y$}
	\begin{itemize}
		\item	True model: $y = \beta_0 + \beta_1 x + \varepsilon \quad \Longrightarrow \quad \hat{y} = \beta_0 + \beta_1 x$
		\item	Predict a new $y$-value: $\hat{y}_i = b_0 + b_1 x_i$
		\item	We could assume $\hat{y}_i \pm 2S_E$
	\end{itemize}
	\vspace{-6pt}
	\begin{center}
		\includegraphics[height=0.70\textheight]{\imagedir/least-squares/show-naive-predictions.png}
	\end{center}
	\vspace{-6pt}
	Use this when you are in a hurry: {\small you're going to get a good estimate!}
\end{frame}

\begin{frame}\frametitle{More accurate prediction interval for $y$}
	But, the previous slide ignores that $b_0$ and $b_1$ have error: that gets propagated into $\hat{y}_i$. True prediction error:
	\begin{exampleblock}{}
		$$\mathcal{V}\{\hat{y}_i\} = S_E^2 \left(1 + \dfrac{1}{n} + \dfrac{(x_i - \overline{\mathrm{x}})^2}{\sum_j{\left( x_j - \overline{\mathrm{x}} \right)^2}}\right)$$
	\end{exampleblock}
	\begin{itemize}
		\item	Example in the notes
		\item	General shape: quadratic shape
		\item	Smallest prediction error at the model center
		\item	Prediction error expands progressively wider as one moves away from the model center
		\item	Makes intuitive sense
	\end{itemize}
\end{frame}

\begin{frame}\frametitle{Prediction interval for $y$: \emph{it has curvature}}
	\begin{center}
		\includegraphics[height=0.9\textheight]{\imagedir/least-squares/show-anscome-solution-with-yhat-bounds.png}
	\end{center}
\end{frame}

\begin{frame}\frametitle{Investigating a linear model}
	\begin{itemize}
		\item	Transformations of data
		\item	Leverage, outliers, influence and discrepancy
		\item	Multiple linear regression, MLR
		\item	Training and testing data
	\end{itemize}

	\vspace{12pt}
	The residuals contain the clues to problems with the model.
\end{frame}

\begin{frame}\frametitle{Normally distributed errors}

	\begin{exampleblock}{Aim}
		\begin{center}
			Ensure the residuals are normally distributed
		\end{center}
	\end{exampleblock}

	If non-normal: $S_E$ and other variances such as $\mathcal{V}(b_1)$ are inflated.

	\vspace{12pt}
	\textbf{Detecting it}:
	\begin{itemize}
		\item	Use a q-q plot: should match 45 degree line
		\begin{itemize}
			\item	Most accurate way
		\end{itemize}
		\item	A histogram is somewhat helpful
		\begin{itemize}
			\item	Human eye not good at picking up heavy tails
		\end{itemize}
		\item	Do \emph{not} plot the residuals to try detect if normally distributed
		\begin{itemize}
			\item	Human eye not good at picking up heavy tails
		\end{itemize}
	\end{itemize}
\end{frame}

\begin{frame}\frametitle{Normally distributed errors}

	\textbf{Dealing with it}:
	\begin{itemize}
		\item	Remove any outlying observation(s) (after investigation)
		\item	Transform the $y$-variable
		\item	Add additional explanatory terms to the model (see this in MLR later)
	\end{itemize}
\end{frame}

\begin{frame}\frametitle{Normally distributed errors}
	\begin{center}
		\includegraphics[width=0.7\textwidth]{\imagedir/least-squares/non-normal-errors-outliers.png}
	\end{center}
	$
	\begin{array}{rcccl}
		\text{Before}: \qquad & b_1 = -0.173 & \qquad -0.255 \leq \beta_1 \leq -0.0898 \\
		& S_E = 9789\\ \\
		\text{After}: \qquad & b_1 = -0.155 & \qquad -0.230 \leq \beta_1 \leq -0.0807 \\
		&S_E = 8655 & {\color{myOrange}{\longleftarrow} \text{it has been reduced}}
	\end{array}
	$
\end{frame}

\begin{frame}\frametitle{Normally distributed errors}
	\begin{center}
		\includegraphics[width=\textwidth]{\imagedir/least-squares/non-normal-errors-transformation-required.png}
	\end{center}
\end{frame}

\begin{frame}\frametitle{Non-constant error variance}
	\begin{center}
		\includegraphics[width=\textwidth]{\imagedir/least-squares/residual-pattern-non-contant-error.png}
	\end{center}
	\begin{itemize}
		\item	Variability in $y$ is non-constant as $x$ or $y$ change.
		\item	Increases $S_E$, undermining confidence intervals
		\item	Not usually too problematic (LS is pretty robust)
	\end{itemize}
\end{frame}

\begin{frame}\frametitle{Non-constant error variance}

	\begin{center}
		\includegraphics[width=0.45\textwidth]{\imagedir/least-squares/constant-error-variance.png}
	\end{center}

	\textbf{Detecting it}:
	\begin{itemize}
		\item	plot $\hat{y}$ {\scriptsize (on the horizontal axis)} against residuals {\scriptsize (on the vertical axis)}
		\item	plot $x$ against residuals
		\item	look for fan-shapes, and regions of non-constant variance
	\end{itemize}

	\textbf{Dealing with it}:
	\begin{itemize}
		\item	Use weighted least squares: $f(\mathrm{b}) = \sum_i^n{(w_ie_i)^2}$
		\item	Weights inversely proportional to the variance
		\item	See: \emph{Draper and Smith} (p 224 to 229, 3rd edition)
	\end{itemize}

	\textbf{Everything OK when}:
	\begin{itemize}
		\item	no visible structure in the above plots
	\end{itemize}
\end{frame}

\begin{frame}\frametitle{Lack of independence in the data}
	\begin{itemize}
		\item	\textbf{Serious issue}:
		\begin{itemize}
			\item	independence required to use CLT
			\item	CLT is used for confidence intervals of $\beta_0$ and $\beta_1$
		\end{itemize}
		\item	Not-independent: \emph{autocorrelated}, e.g. a slow-moving process.
		\item	Dealing with it: topic is called time-series analysis; many good textbooks, e.g. 
			\begin{itemize}
				\item	Chatfield: {\scriptsize The Analysis of Time Series} \hfill {\color{myOrange}{$\longleftarrow$}highly recommended}
				\item	Box and Jenkins: {\scriptsize Time Series Analysis} \hfill {\color{myOrange}{$\longleftarrow$}comprehensive}
			\end{itemize}
		\item	We will use a very simple method here
	\end{itemize}
\end{frame}

\begin{frame}\frametitle{Lack of independence in the data}

	\textbf{Detecting it}: time-ordered plot of residuals
	\begin{center}
		\includegraphics[width=\textwidth]{\imagedir/least-squares/residual-pattern-unmodelled-dynamics-slides.png}
	\end{center}
	Look for patterns:
	\begin{itemize}
		\item	slow drifts
		\item	rapid criss-crossing of the zero axis
		\item	cycles (e.g. daily cycle)
	\end{itemize}
\end{frame}

\begin{frame}\frametitle{Lack of independence in the data}

	\textbf{Dealing with it}: sub-sample the data, every $k^\text{th}$ sample
	\begin{center}
		\includegraphics[width=\textwidth]{\imagedir/least-squares/demonstrate-autocorrelation.png}
	\end{center}
\end{frame}

\begin{frame}\frametitle{Lack of independence in the data}
	\begin{itemize}
		\item	Another test: Durbin-Watson test: \emph{Draper and Smith} (chapter 7)
	\end{itemize}

	\textbf{Everything OK when}:
	\begin{itemize}
		\item	\texttt{acf(y)} and \texttt{acf(e)} show no lags beyond lag 0
	\end{itemize}
\end{frame}

\begin{frame}\frametitle{Linear model (incorrect model specification)}
	\begin{itemize}
		\item	Systems are known to be non-linear; linear might be good enough (depends on model's purpose)
	\end{itemize}
	\begin{center}
		\includegraphics[width=0.8\textheight]{\imagedir/least-squares/nonlinear-linear-region.png}
	\end{center}
\end{frame}

\begin{frame}\frametitle{Linear model (incorrect model specification)}

	\textbf{Detecting it}:
	\begin{itemize}
		\item	plot $\hat{y}$ against residuals (y-axis)
		\item	plot $x$ against residuals
		\item	q-q plot might show it also
	\end{itemize}
	\begin{center}
		\includegraphics[width=\textwidth]{\imagedir/least-squares/nonlinear-detection.png}
	\end{center}
\end{frame}

\begin{frame}\frametitle{Linear model (incorrect model specification)}

	\textbf{Dealing with it}:

	1. Non-linear least squares
	\begin{itemize}
		\item	Too many details to go into for this course
		\item	\emph{Example}: $y = \beta_1\left(1-e^{-\beta_2 x} \right)$
		\begin{itemize}
			\item	$\beta_1 = [10, 20, 30, 40, 50]$ {\scriptsize (e.g. picked 5 values that span $\beta_1$ range)}
			\item	$\beta_2 = [0.0, 0.2, 0.4, 0.8]$ {\scriptsize (e.g. and here I picked 4 values for $\beta_2$)}
		\end{itemize}
		\item	Evaluate sum of squares of errors (SSE = $\displaystyle \sum{e_i^2}$) objective function at all $4 \times 5 = 20$ combinations of $\beta_1$ and $\beta_2$
	\end{itemize}
\end{frame}

\begin{frame}\frametitle{Linear model (incorrect model specification)}

	\textbf{Dealing with it}:

	2. Transform the $x$ or $y$ variable; then use linear model
	\begin{itemize}
		\item	$x_\text{transformed} \leftarrow x^p_\text{original}$
		\item	$p$: 1 up to 1.5, 1.75, 2.0, \emph{etc}
		\item	$p$: 1 down to 0.5 ($\sqrt{x}$), 0.25, $-0.5$, $-1.0$ ($1/x$), $-1.5$, $-2.0$, \emph{etc}
		\item	$\log(x)$: approximates $p=0$ ito severity
	\end{itemize}
\end{frame}

\begin{frame}\frametitle{Linear model (incorrect model specification)}

	3. Rearrange first-principles equations
	\begin{itemize}
		\item	Distillation: $T$ inversely proportional to log(VP). $y = b_0 + b_1x$:
		\begin{itemize}
			\item	$x \leftarrow 1/T$
			\item	$y \leftarrow \log(P)$.
			\item	Slope coefficient: ...
		\end{itemize}
		\item	$y = p \times q^x$; take logs so that $\log(y) = \log(p) + x \log(q)$,
		\begin{itemize}
			\item	Slope coefficient = $\log(q)$
		\end{itemize}
		\item	$y = \dfrac{1}{p+qx}$, invert to get: $y = b_0 + b_1 x$:
		\begin{itemize}
			\item	$b_0 \leftarrow p$
			\item	$b_1 \leftarrow q$
			\item	$y\leftarrow 1/y$
		\end{itemize}
	\end{itemize}

	\textbf{Everything OK when}:
	\begin{itemize}
		\item	No more structure in the detection plots
	\end{itemize}
\end{frame}

\begin{frame}\frametitle{Summary of steps to build and investigate a linear model}

	1. Plot the data

	2. Fit model and examine text:
	\begin{center}
		\includegraphics[width=\textwidth]{\imagedir/least-squares/source-code-1.png}
	\end{center}
	\begin{itemize}
		\item	Model's standard error, compare it to $y$
		\item	Interpret confidence interval
	\end{itemize}

	3. Visualize predictions:
	\begin{center}
		\includegraphics[width=\textwidth]{\imagedir/least-squares/source-code-2.png}
	\end{center}
\end{frame}

\begin{frame}\frametitle{Summary of steps to build and investigate a linear model}

	4. q-q plot of the residuals. Are they normally distributed?

	5. Residuals against the $x$-values.
	\begin{itemize}
		\item	Expect to see no particular structure
	\end{itemize}

	6. Residuals in time (sequence) order
	\begin{itemize}
		\item	Also try \texttt{acf(model\$residuals))}
	\end{itemize}

	7. Residuals against the fitted-values.

	8. Plot the $\hat{y}$ against actual values of $y$
\end{frame}

\begin{frame}\frametitle{Multiple linear regression (MLR)}

	\textbf{AIM}: Include more than one $x$-variable in the model.
	\begin{enumerate}
		\item	introduce some matrix notation
		\item	redo the optimization problem
		\item	interpret the model coefficients
		\item	use integer variables
	\end{enumerate}
\end{frame}

\begin{frame}\frametitle{Multiple linear regression (MLR)}
	\begin{enumerate}
		\item	Improve our understanding of systems:
		\begin{itemize}
			\item	$x_1$ = reactant concentration
			\item	$x_2$ = temperature
			\item	$y$ = reaction rate
			\item	Understand the effect of temperature also: $y = b_0 + b_1x_1 + b_2x_2$
		\end{itemize}
		\item	Improve predictions:
		\begin{itemize}
			\item	$x_1$ = temperature
			\item	$x_2$ = feed flowrate
			\item	$y$ = melt index
			\item	Better predictions with $y = b_0 + b_1x_1 + b_2x_2$ than with $y = b_0 + b_1x_1$
		\end{itemize}
		\item	Integer variables at more than 2 levels:
		\begin{itemize}
			\item	mixing tank
			\item	operator
			\item	supplier
			\item	day of week
		\end{itemize}
	\end{enumerate}
\end{frame}

\begin{frame}\frametitle{Notation for MLR}

	We will first center the data:

	$$
	\begin{array}{rcl}
		y_i &=& b_0 + b_1 x_i \\
		\overline{y} &=& b_0 + b_1 \overline{x} \\
		y_i - \overline{y} &=& 0 +b_1(x_i - \overline{x}) \qquad \text{by subtracting previous lines} \\
	\end{array}
	$$
	\begin{itemize}
		\item	Let $x = x_\text{original} - \text{mean}\left(x_\text{original} \right)$
		\item	Let $y = y_\text{original} - \text{mean}\left(y_\text{original} \right)$
		\item	Model is still the same, except intercept term is forced to zero: $b_0 = 0$
	\end{itemize}
	\begin{itemize}
		\item	If you work with \emph{uncentered} data, but add a $b_0 = 0$ term to the model, you will get an \textbf{identical} MLR model
		\begin{itemize}
			\item	Centered data is more interpretable: $\mathbf{X}^T\mathbf{X}$ and $\mathbf{X}^T\mathbf{y}$
			\item	Intercept can always be recovered afterwards, if required
		\end{itemize}
	\end{itemize}
\end{frame}

\begin{frame}\frametitle{Notation for MLR}

	The general linear model is given by:

	$$
	\begin{array}{rcl}
		\\
		y_i &=& \beta_1 x_{i,1} + \beta_2x_{i,2} + \ldots + \beta_kx_{i,k} + \varepsilon_i \\
		y_i &=& [x_{i,1},\,\, x_{i,2},\,\, \ldots,\,\, x_{i,k}]
		\begin{bmatrix}
			\beta_1 \\
			\beta_2 \\
			\vdots \\
			\beta_k
		\end{bmatrix}
		+ \varepsilon_i \\
		y_i &=& \underbrace{\mathit{x}_i^T}_{(1 \times k)} \underbrace{\beta}_{(k \times 1)} + \,\varepsilon_i
	\end{array}
	$$
	% \begin{itemize}
	% 	\item	where each $x_k$ column (variable) and the $y$ column have been centered
	% \end{itemize}
\end{frame}

\begin{frame}\frametitle{Notation for MLR}

	$$
	\begin{array}{rl}
		\begin{bmatrix}
			y_1\\
			y_2\\
			\vdots \\
			y_n
		\end{bmatrix}
		&=
		\begin{bmatrix}
			x_{1,1} & x_{1,2} & \ldots & x_{1,k}\\
			x_{2,1} & x_{2,2} & \ldots & x_{2,k}\\
			\vdots & \vdots & \ddots & \vdots\\
			x_{n,1} & x_{n,2} & \ldots & x_{n,k}\\
		\end{bmatrix}
		\begin{bmatrix}
			b_1 \\
			b_2 \\
			\vdots \\
			b_k
		\end{bmatrix}
		+
		\begin{bmatrix}
			e_1\\
			e_2\\
			\vdots \\
			e_n
		\end{bmatrix}
		\\
		& \\
		\mathbf{y} &= \mathbf{X} \mathbf{b} + \mathbf{e}
	\end{array}
	$$
	\begin{itemize}
		\item	$\mathbf{y}$: $n \times 1$
		\item	$\mathbf{X}$: $n \times k$
		\item	$\mathbf{b}$: $k \times 1$
		\item	$\mathbf{e}$: $n \times 1$
	\end{itemize}
\end{frame}

\begin{frame}\frametitle{Estimating the model parameters via optimization}

	\textbf{\emph{Objective function}}: minimize sum of squares of the errors

	$$
	\begin{array}{rl}
		& \\
		f(\mathbf{b}) &= \mathbf{e}^T\mathbf{e} \\
		&= \left(\mathbf{y} - \mathbf{X} \mathbf{b} \right)^T \left( \mathbf{y} - \mathbf{X} \mathbf{b} \right) \\
		&= \mathbf{y}^T\mathbf{y} - 2 \mathbf{y}^T\mathbf{X}\mathbf{b} + \mathbf{b}\mathbf{X}^T\mathbf{X}\mathbf{b} \\
		\\
	\end{array}
	$$
	\begin{itemize}
		\item	Solving $\dfrac{f(\mathbf{b})}{
		\partial{\mathbf{b}}} = 0 $ gives us that $\mathbf{b} = \left( \mathbf{X}^T\mathbf{X} \right)^{-1}\mathbf{X}^T\mathbf{y}$.
	\end{itemize}
\end{frame}

\begin{frame}\frametitle{Estimating the model parameters via optimization}
	\begin{enumerate}
		\item	$\mathcal{E}\{\mathbf{b}\} = \mathbf{\beta}$
		\item	$\mathcal{V}\{\mathbf{b}\} = \left( \mathbf{X}^T\mathbf{X} \right)^{-1} S_E^2$
		\item	Standard error = $\sigma_e \approx S_E = \sqrt{\dfrac{\mathbf{e}^T\mathbf{e}}{n-k}}$,
	\end{enumerate}
	\begin{itemize}
		\item	$k$ is the number of parameters estimated in the model
		\item	$n$ is the number of observations
	\end{itemize}

	Earlier we had: $\mathcal{V}\{b_1\} = \dfrac{S_E^2}{\sum_j{\left( x_j - \overline{\mathrm{x}} \right)^2}}$
	\begin{itemize}
		\item	Matrix form is identical
		\item	Decrease $\mathcal{V}\{\mathbf{b}\}$ by:
		\begin{itemize}
			\item	taking more samples (increases denominator size)
			\item	including observations away from the center of the model
		\end{itemize}
	\end{itemize}
\end{frame}

\begin{frame}\frametitle{Example}
	\begin{itemize}
		\item	$x_{1,\text{raw}} = [1, 3, 4, 7, 9, 9]$
		\begin{itemize}
			\item	$x_1 = [-4.5, -2.5, -1.5 , 1.5 , 3.5, 3.5]$
		\end{itemize}
		\item	$x_{2,\text{raw}} = [9, 9, 6, 3, 1, 2]$
		\begin{itemize}
			\item	$x_2 = [4, 4, 1, -2, -4, -3]$
		\end{itemize}
		\item	$y_{\text{raw}} = [3,5,6,8,7,10]$
		\begin{itemize}
			\item	$y = [-3.5, -1.5, -0.5, 1.5, 0.5, 3.5]$
		\end{itemize}
	\end{itemize}

	$$
	\begin{array}{lr}
		\mathbf{X} =
		\begin{bmatrix}
			-4.5 & 4\\
			-2.5 & 4 \\
			-1.5 & 1 \\
			1.5 & -2 \\
			3.5 & -4 \\
			3.5 & -3
		\end{bmatrix}
		&\qquad\qquad \mathbf{y} =
		\begin{bmatrix}
			-3.5 \\
			-1.5\\
			-0.5\\
			1.5\\
			0.5\\
			3.5
		\end{bmatrix}
	\end{array}
	$$

	$$
	\begin{array}{lr}
		\mathbf{X}^T\mathbf{X} =
		\begin{bmatrix}
			55.5 & -57.0 \\-57.0 & 62
		\end{bmatrix}
		&\qquad\qquad \mathbf{X}^T\mathbf{y} =
		\begin{bmatrix}
			36.5 \\
			-36.0
		\end{bmatrix}
	\end{array}
	$$
\end{frame}

\begin{frame}\frametitle{Example}

	$$
	\begin{array}{lr}
		\mathbf{X}^T\mathbf{X} =
		\begin{bmatrix}
			55.5 & -57.0 \\-57.0 & 62
		\end{bmatrix}
		&\qquad\qquad \mathbf{X}^T\mathbf{y} =
		\begin{bmatrix}
			36.5 \\
			-36.0
		\end{bmatrix}
	\end{array}
	$$
	\begin{itemize}
		\item	$\mathbf{X}^T\mathbf{X}$: scaled version of the covariance matrix of $\mathbf{X}$.
		\item	Diagonal entries: variance, always positive
		\item	Off-diagonal entries: symmetrical, strength of relationship between variables
		\item	What does $\mathbf{X}^T\mathbf{X}$ look like for uncorrelated variables?
	\end{itemize}
	\vspace{24pt}
	Let's take a look at building the model in R.
	\begin{itemize}
		\item	\texttt{model <- lm(y \textasciitilde\,\, x1 + x2)}
		\item	\texttt{summary(model)}
	\end{itemize}
\end{frame}

\begin{frame}\frametitle{Example: {\color{myOrange}{600-level students self-study}}}
	\begin{itemize}
		\item	Standard error = $\mathcal{V}\{\mathbf{b}\} = \left( \mathbf{X}^T\mathbf{X} \right)^{-1} S_E^2$
	\end{itemize}

	$$ \qquad\qquad \left(\mathbf{X}^T\mathbf{X}\right)^{-1}=
	\begin{bmatrix}
		0.323 & 0.297 \\
		0.297 & 0.289
	\end{bmatrix}
	$$
	\begin{itemize}
		\item	Variances are not independent of each other! CI are not independent either.
		\item	For 2 variables:
	\end{itemize}

	$$
	\begin{array}{lcr}
		\qquad\qquad \mathcal{V}\left(b_1\right) &=& \dfrac{1}{1-r^2_{12}} \times \dfrac{S_E^2}{\sum{x_1^2}} \\
		\qquad\qquad \mathcal{V}\left(b_2\right) &=& \dfrac{1}{1-r^2_{12}} \times \dfrac{S_E^2}{\sum{x_2^2}}
	\end{array}
	$$
	\begin{itemize}
		\item	$r^2_{12}$ = correlation between $x_1$ and $x_2$
		\item	What happens as the correlation between the two variables increases?
		\item	Each $\mathcal{V}\left(b_k\right)$ variance has $n-k-1$ degrees of freedom
	\end{itemize}
\end{frame}

\begin{frame}\frametitle{Interpretation of the model coefficients}
	\begin{block}{Model}
		\begin{center}
			$y = b_1x_1 + b_2x_2$
		\end{center}
	\end{block}
	\begin{center}
		\includegraphics[width=0.65\textwidth]{\imagedir/least-squares/least-squares-two-x-variables.png}
	\end{center}
\end{frame}

\begin{frame}\frametitle{Interpretation of the model coefficients}
	\begin{itemize}
		\item	$y = b_1x_1 + b_2x_2$
	\end{itemize}
	\begin{itemize}
		\item	Coefficient $b_1$ is the average change in $\mathbf{y}$ for a one unit change in ${x}_1$ \textbf{provided we hold} ${x}_2$ \textbf{\emph{fixed}}
	\end{itemize}
	
	\vspace{24pt}
	\begin{itemize}
		\item	\emph{Example}: $y = b_T T + b_S S = -0.52 T + 3.2 S$
		
		\vspace{12pt}
		\begin{itemize}
			\item	$T$ = reactor temperature in Kelvin,
			\item	$S$ = substrate concentration in g/L
			\item	$y$ = yield in $\mu\text{g}$
			
			\vspace{12pt}
			\item	$b_T = -0.52 \mu\text{g}/\text{K}$: decrease in yield for every 1 Kelvin increase in temperature, holding the substrate concentration fixed
		\end{itemize}
	\end{itemize}
\end{frame}

\begin{frame}\frametitle{Interpretation of the model coefficients}

	$$y = b_T T + b_S S = -0.52 T + 3.2 S$$
	\begin{itemize}
		\item	CI for $b_T$ spans zero: ``\emph{the effect of the temperature, controlling for concentration, is not significant}"
		\item	The ``controlling for" indicates the controlled variable was used in the model
		\item	CI for $b_S$ does not span zero: ``\emph{the effect of concentration, controlling for temperature, is to increase the yield by 3.2 $\mu\text{g}$ for every 1g/L increase in concentration, S}"
	\end{itemize}
\end{frame}

\begin{frame}\frametitle{Integer variables in the model}
	\begin{center}
		\includegraphics[width=0.6\textwidth]{\imagedir/least-squares/Mixing_-_flusso_assiale_e_radiale.jpg}
	\end{center}
	\vspace{-16pt}
	\see{From \href{http://en.wikipedia.org/wiki/File:Mixing_-_flusso_assiale_e_radiale.jpg}{Wikipedia}}

	\vspace{12pt}
	Yield = \emph{f}\,(temperature, impeller type); impeller = [radial, axial]
	\begin{itemize}
		\item	Build two models: one for radial, one for axial
		\begin{itemize}
			\item	Not efficient use of the data
			\item	Temperature effect is the same, independent of impeller
			\item	Fewer degrees of freedom in each model
			\item	Increases $S_E$
		\end{itemize}
	\end{itemize}
\end{frame}

\begin{frame}\frametitle{Integer variables in the model}
	$$
	\begin{array}{lcr}
		y &=& \beta_0 + \beta_1 T + \gamma d + \varepsilon \qquad\hfill {\color{myOrange}{\longleftarrow}\text{population model}}\\
		y &=& b_0 + b_1 T + g d_i + e_i \qquad \hfill {\color{myOrange}{\longleftarrow}\text{statistical model}}\\
		\\
	\end{array}
	$$
	\begin{itemize}
		\item	Let: $d_i = 0$ for axial impeller
		\item	Let: $d_i = 1$ for radial impeller
	\end{itemize}
\end{frame}

\begin{frame}\frametitle{Integer variables in the model}

	Assume $\beta_1 = 0$ (temperature has no effect) for now.
	
	$$y = b_0 + \cancel{b_1 T} + g d_i$$
	
	Geometrically: \vspace{-0.5cm}
	\begin{center}
		\includegraphics[width=0.7\textwidth]{\imagedir/least-squares/least-squares-dummy-variable-and-intercept.png}
	\end{center}
	$$
	\begin{array}{llll}
		\text{Axial impeller:} \qquad & d_i=0 &\qquad y = b_0 + gd_i &\qquad y = b_0 + 0 \\
		\text{Radial impeller:} \qquad & d_i=1 &\qquad y = b_0 + gd_i & \qquad y = b_0 + g
	\end{array}
	$$
	\begin{itemize}
		\item	$g$ = difference in yield when changing the impeller from axial to radial
	\end{itemize}
\end{frame}

\begin{frame}\frametitle{Integer variables in the model}

	Now let $\beta_1 \neq 0$:

	$$
	\begin{array}{ll}
		& \\
		\text{Axial impellers:} \qquad &\qquad y = b_0 + b_1 T + 0 \\
		\text{Radial impellers:} \qquad &\qquad y = b_0 + b_1 T + g
	\end{array}
	$$
	\begin{itemize}
		\item	The lines are still parallel
		\item	If $g = -56 \mu\text{g}$: the decrease in yield is expected to be 56 $\mu\text{g}$ when changing from an axial to a radial impeller
	\end{itemize}
\end{frame}

\begin{frame}\frametitle{Integer variables in the model}
	\begin{center}
		\includegraphics[width=0.85\textheight]{\imagedir/least-squares/least-squares-two-x-variables-one-integer.png}
	\end{center}
	\begin{itemize}
		\item	Example: assume the 95\% confidence interval for impeller parameter was: $ -32 \mu\text{g} \leq \gamma \leq 21 \mu\text{g}$; i.e. no significant effect on yield
		\item	CI for integer variables: no different to other variables
	\end{itemize}
\end{frame}

\begin{frame}\frametitle{Integer variables in the model}

	Raw material from Spain, India, or Vietnam [3 levels]
	\begin{itemize}
		\item	2 integer variables: $y = \beta_0 + \beta_1x_1 + \ldots + \beta_k x_k +\mathbf{ \gamma_1 d_1 + \gamma_2 d_2} + \varepsilon$
		\begin{itemize}
			\item	$d_{i1} = 0$ and $d_{i2} = 0$ for Spain
			\item	$d_{i1} = 1$ and $d_{i2} = 0$ for India
			\item	$d_{i1} = 0$ and $d_{i2} = 1$ for Vietnam
			\item	Interpret coefficients relative to the (0,0) baseline for Spain
		\end{itemize}
	\end{itemize}
	\begin{itemize}
		\item	3 integer variables: $y = \beta_0 + \beta_1x_1 + \ldots + \beta_k x_k + \mathbf{\gamma_1 d_1 + \gamma_2 d_2 + \gamma_3 d_3} + \varepsilon$
		\begin{itemize}
			\item	$d_{i1} = 1$ and $d_{i2} = 0$ and $d_{i3} = 0$ for Spain
			\item	$d_{i1} = 0$ and $d_{i2} = 1$ and $d_{i3} = 0$ for India
			\item	$d_{i1} = 0$ and $d_{i2} = 0$ and $d_{i3} = 1$ for Vietnam
			\item	These coefficients are more easily interpretable
			\item	Loose a single degree of freedom
			\item	IMPOSSIBLE to calculate model: the $\mathbf{X}^T\mathbf{X}$ is perfectly collinear, and cannot be inverted.
		\end{itemize}
	\end{itemize}
\end{frame}

\begin{frame}\frametitle{Outliers: discrepancy, leverage, and influence}
	\begin{itemize}
		\item	Unusual observations influence the model parameters and our interpretation
		\item	Outliers are the most interesting data:
		\begin{itemize}
			\item	Do not delete without investigating them
			\item	If outlier is correct it could be very valuable information
			\item	Might have to add a new variable in model (avoid overfitting)
		\end{itemize}
	\end{itemize}

	\vspace{12pt}
	\begin{exampleblock}{For unusual data: }
		$$\text{Leverage} \times \text{Discrepancy} = \text{Influence on the model} $$
	\end{exampleblock}
\end{frame}

\begin{frame}\frametitle{Model A}
	\begin{center}
		\includegraphics[width=0.8\textwidth]{\imagedir/least-squares/influence-of-outliers-slide-1.png}
	\end{center}
\end{frame}

\begin{frame}\frametitle{Model B}
	\begin{center}
		\includegraphics[width=0.8\textwidth]{\imagedir/least-squares/influence-of-outliers-slide-2.png}
	\end{center}
\end{frame}

\begin{frame}\frametitle{Model C}
	\begin{center}
		\includegraphics[width=0.8\textwidth]{\imagedir/least-squares/influence-of-outliers-slide-3.png}
	\end{center}
\end{frame}

\begin{frame}\frametitle{Leverage}
	\begin{itemize}
		\item	How much each observation contributes to $\hat{y}_i$; called hat value, $h_i$
		\item	Measures how far the data point is from the center of the model
		\item	$ h_i = \dfrac{1}{n} + \dfrac{\left(x_i -\overline{x}\right)^2}{\sum_{j=1}^{n}{\left(x_j -\overline{x}\right)^2}} \qquad \overline{h} = \dfrac{k}{n} \qquad \dfrac{1}{n} \leq h_i \leq 1.0$
		\item	Show limits at 2 and 3 times average hat value
	\end{itemize}
	\begin{center}
		\includegraphics[width=0.9\textwidth]{\imagedir/least-squares/hatvalue-of-outliers-slides.png}
	\end{center}
\end{frame}

\begin{frame}\frametitle{Discrepancy}

	Discrepancy: unusual data pair \emph{in the context of the least squares model}
	\begin{itemize}
		\item	Measured as residual distance, accounting for leverage: \emph{studentized residuals}
	\end{itemize}
	$$ \quad e_i^* = \dfrac{e_i}{S_{E(-i)}\sqrt{1-h_i}} = \dfrac{e_i - 0}{\text{``corrected'' standard deviation}\left\{e_i \right\}} $$
	\begin{itemize}
		\item	$S_{E(-i)}$ = standard error without $i^\text{th}$ point in model
	\end{itemize}
	\begin{center}
		\includegraphics[width=\textwidth]{\imagedir/least-squares/studentized-residuals-slides.png}
	\end{center}
\end{frame}

\begin{frame}\frametitle{Influence}
	\begin{itemize}
		\item	\emph{How much does the model change when omitting a data point}
		\item	$\text{Influence on the model} = \text{Discrepancy} \times \text{Leverage} $
		\item	\emph{Cook's statistic}, called $D_i = \dfrac{e_i^2}{k \times \frac{1}{n}\sum{e_i^2}} \times \dfrac{h_i}{1-h_i} $
		\item	\texttt{cooks.distance(model)} in R
		\item	Cut-off value: $D_i > \dfrac{4}{n-k} $
	\end{itemize}
\end{frame}

\begin{frame}\frametitle{Influence}
	\begin{columns}[t]
		\column{1.1\textwidth}
			\begin{center}
				\includegraphics[width=\textwidth]{\imagedir/least-squares/cooks-distances-slides.png}
			\end{center}
		\column{0.01\textwidth}
	\end{columns}
\end{frame}

\begin{frame}\frametitle{Combining all measures of influence}
	\begin{columns}[t]
		\column{1.1\textwidth}
			\begin{center}
				\includegraphics[width=\textwidth]{\imagedir/least-squares/influence-plots-slides.png}
			\end{center}
		\column{0.01\textwidth}
	\end{columns}
	
	\begin{itemize}
		\item	\texttt{library(car)}
		\item	\texttt{model <- lm(y \string~ x)}
		\item	\texttt{influencePlot(model)}
		\item	\texttt{influenceIndexPlot(model)}
	\end{itemize}
\end{frame}

\begin{frame}\frametitle{Enrichment Topics}
	\begin{itemize}
		\item	Enrichment: \emph{ robust least squares models}
		\begin{itemize}
			\item	Useful for automated model building
			\item	Human reviewer not trained in understanding plots
		\end{itemize}
		
		\vspace{12pt}
		\item	Enrichment: \emph{Logistic modelling}
		\begin{itemize}
			\item	\emph{y}-variable is discrete (yes/no or pass/fail)
			\item	Model then predicts probability of success/failure from various x's
		\end{itemize}
	\end{itemize}
\end{frame}

\begin{frame}\frametitle{Enrichment: Testing of least-squares models}
	\begin{enumerate}
		\item	Objective: learn more about our system
		\begin{itemize}
			\item	effect of one variable on another,
			\item	significant effect (use the CI)
		\end{itemize}
		\item	Objective: prediction
		\begin{itemize}
			\item	How well does it work?
		\end{itemize}
	\end{enumerate}

	\textbf{Gold standard}: use an independent testing data set
	\begin{itemize}
		\item	$\text{RMSEP} = \sqrt{\dfrac{1}{n}\sum_{i}^{n}{\left(y_{\text{new}, i} - \hat{y}_{\text{new}, i}\right)^2}}$
		\item	RMSEE is the same as RMSEP, but for building the model
		\item	The RMSEE $\approx S_E$ = standard error
		\item	Or, use some other measure of ``closeness''
	\end{itemize}
\end{frame}

\begin{frame}\frametitle{Testing of least-squares models}

	\textbf{Example}:
	\begin{itemize}
		\item	Need to build a predictive model for product viscosity using 3 $x$-variables
		\item	Observations: one per day, from 2006 and 2007 (730 observations)
	\end{itemize}

	Which situation is better?
	\begin{itemize}
		\item	\textbf{A}
		\begin{itemize}
			\item	Observations: 1, 3, 5, 7, ... 729 to build
			\item	Observations: 2, 4, 6, 8, ... 730 to test
		\end{itemize}
		\item	or \textbf{B}
		\begin{itemize}
			\item	Observations: 1 to 365 (2006 data) to build
			\item	Observations: 366 to 730 (2007 data) to test
		\end{itemize}
	\end{itemize}
\end{frame}

\begin{frame}\frametitle{Enrichment: Nonparametric modelling}

	Relationship between $x$ and $y$ is of the form: $y = f(x) + \varepsilon$,
	\begin{itemize}
		\item	The function (model), $f(x)$, is unspecified
		\item	The model is usually a smooth function
	\end{itemize}
	\vspace{-12pt}
	\begin{center}
		\includegraphics[height=0.8\textheight]{\imagedir/least-squares/Income-Prestige-raw-data.png}
	\end{center}
\end{frame}

\begin{frame}\frametitle{Enrichment: Nonparametric modelling}
	\begin{center}
		\includegraphics[height=0.65\textheight]{\imagedir/least-squares/Income-Prestige-with-smoothing.png}
	\end{center}
	\begin{itemize}
		\item	The smoothed line is the nonparametric function, $f(x)$
		\item	See the Cleveland reference \emph{Robust locally weighted regression and smoothing scatterplots}
		\item	See the \texttt{lowess} function in R
	\end{itemize}
\end{frame}

\begin{frame}\frametitle{Enrichment: Bootstrapping}
	\begin{itemize}
		\item	Enrichment topic
		\item	\emph{Example}: bootstrapping strictly not required, but useful.
		\begin{itemize}
			\item	Model: $y = \beta_0 + \beta_1 x$
			\item	Interested in the slope $\beta_1$
			\item	Uncertainty associated with it? How much does it depend on the data?
			\item	The CI is:
		\end{itemize}
	\end{itemize}
	$$
	\begin{array}{rcccl}
		- c_t &\leq& \dfrac{b_1 - \beta_1}{S_E(b_1)} &\leq & +c_t\\
		b_1 - c_t S_E(b_1) &\leq& \beta_1 &\leq& b_1 + c_t S_E(b_1)
	\end{array}
	$$
\end{frame}

\begin{frame}\frametitle{Bootstrapping example}
	\begin{itemize}
		\item	x = dose of radiation administered
		\item	y = survival percentage
	\end{itemize}
	\begin{center}
		\includegraphics[width=0.5\textwidth]{\imagedir/least-squares/bootstrap-example-slide1.png}
	\end{center}
	\begin{itemize}
		\item	Point 13 is influential
		\begin{itemize}
			\item	slope = $-0.0059$ with point 13
			\item	slope = $-0.0078$ without it
		\end{itemize}
	\end{itemize}
\end{frame}

\begin{frame}\frametitle{Bootstrapping example}
	\begin{itemize}
		\item	Randomly select n=14 rows from the raw data
		\begin{itemize}
			\item	\texttt{[4] 9 1 6 5 5 5 4 11 8 2 9 13 8 13}
			\item	\texttt{[5] 4 7 12 10 13 12 12 3 8 4 14 1 6 4}
			\item	\texttt{[6] 3 4 10 14 14 9 10 10 8 11 1 1 4 14}
		\end{itemize}
		\item	Calculate slope for each round
		\item	CI: $-0.0082 \leq \beta_1 \leq -0.0036$
	\end{itemize}
	\begin{center}
		\includegraphics[width=0.75\textwidth]{\imagedir/least-squares/bootstrap-example-slide2.png}
	\end{center}
\end{frame}

