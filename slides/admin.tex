\begin{frame}\frametitle{Plan for today's class}
	\begin{enumerate}
		\item	Background 
		\item	Administrative issues 
		\item	Course contents 
	\end{enumerate}
\end{frame}

\begin{frame}\frametitle{Background}
	{\color{myGreen}{About myself}}
	\begin{itemize}
		\item	Undergraduate degree from University of Cape Town, 1999
		\item	Masters degree from McMaster, 2002 (not a ``doctor'', please)
		\item	Worked with a number of companies since then on data analysis and consulting projects
		\item	Worked at GSK on a 1-year contract until June 2012
		\item	Now working full-time at McMaster since July 2012
		\item	Drop-in hours: Wednesday AM, Thursday PM, Friday PM
		\item	Office is in BSB, room B105
		\item	Arrange a meeting: \url{kevin.dunn@mcmaster.ca}
		\item	Cell: (905) 921 5803 and not \sout{extension 27337}
	\end{itemize}
	
	\vspace{12pt}
	\textbf{My objective}
	
	\vspace{6pt}
	I hope to make this class \textbf{worthwhile} and \textbf{practically} applicable to you. Please let me know how I'm doing at any time; there will be anonymous course evaluations at least twice throughout the course for your feedback.
\end{frame}

\begin{frame}\frametitle{Acknowledgments}
	\begin{itemize}
		\item	Dr. John MacGregor, who taught this course since 1983
		\begin{itemize}
			\item	Happy 30th anniversary 4C3/6C3
		\end{itemize}
		\item	McMaster Advanced Control Consortium (MACC) 
		\item	The many companies I've worked with over the past 10 years
			\begin{itemize}
				\item	their problems and data appear in the course, often in disguised form
			\end{itemize}
	\end{itemize}
\end{frame}

\begin{frame}\frametitle{Administrative issues}
	\begin{itemize}
		\item	TA introduction 
		\item	Video and audio 
		\item	Website 
		\item	References 
		\item	Software 
		\item	Expectations 
		\item	Grading 
	\end{itemize}
\end{frame}

\begin{frame}\frametitle{Teaching assistants}
	\vspace{12pt}
	{\color{myGreen}{Maryam Emami}}
	\begin{itemize}
		\item	\url{emamis2@mcmaster.ca}
		\item	JHE, room 361
		\item	extension  23263
		\item	Currently doing her Ph.D with John Vlachopoulos
	\end{itemize}
	\vspace{12pt}
	{\color{myGreen}{Shailesh Patel}}
	\begin{itemize}
		\item	\url{patelsr@mcmaster.ca}
		\item	JHE, room 369
		\item	extension 24031
		\item	Currently doing his Ph.D with Chris Swartz
	\end{itemize}
	\vspace{24pt}

	{\color{myOrange}{Office hours for both TAs are by email appointment}}
\end{frame}

\begin{frame}\frametitle{Course website}
	
	\begin{exampleblock}{}
		\centering 
		\href{http://learnche.mcmaster.ca/4C3}{http://learnche.mcmaster.ca/4C3}
	\end{exampleblock}
	\begin{itemize}
		\item	\emph{Not an Avenue website!}
		\item	Slides will be added to the site before class
		\item	Please print slides and bring to class
		\item	Assignments and solutions will be posted there
		\item	Data sets and many resources you will require are posted there
	\end{itemize}
	\vspace{12pt}
	\textbf{ Website is the main reference for all things course-related} 
	\begin{itemize}
		\item	expected to check it about 3 times per week {\tiny (top left)}
		\item	or follow on Twitter to get updates: \href{https://twitter.com/stats4eng}{@stats4eng}
	\end{itemize}
\end{frame}

\begin{frame}\frametitle{Video and audio}
	\begin{itemize}
		\item	Video recordings from 2010 and 2012 on website
		\item	Audio recordings from 2011 on website
		\item	Purpose: for your review, and to prepare for assignments and exams 
		\item	Might be useful if you miss a class
		\item	As long as feasible, I will try to video record all classes
		\begin{itemize}
			\item	Try to record just myself, the board and the projector 
			\item	Can't guarantee the quality will be very good (background noise, etc) 
			\item	Video should be available within 24 to 48 hours after the class 
		\end{itemize}
		\item	Audio recordings will also be made available, when possible 
	\end{itemize}
\end{frame}

\begin{frame}\frametitle{Course text book and references}
	\begin{itemize}
		\item	All you require are the slides and ... 
		\begin{itemize}
			\item	\textbf{Process Improvement using Data} 
			\item	Draft book; it gets updated every week while I'm teaching 4C3/6C3
			\item	\textbf{\href{http://learnche.mcmaster.ca/pid?source=4C3-admin-notes}{http://learnche.mcmaster.ca/pid}} 
			\item	Use website to report errors; suggest improvements for slides and book 
		\end{itemize}
	\end{itemize}
	\begin{itemize}
		\item	Pre-printed copies available from Titles bookstore 
		\begin{itemize}
			\item	$\sim$\$35 (just the cost of printing) 
			\item	should be available at the end of this week, or early next week
		\end{itemize}
	\end{itemize}
	\begin{itemize}
		\item	Some suggested books on the course website: 
		\begin{columns}[t]
			\column{0.80\textwidth}
				\begin{itemize}
					\item	\emph{Recommended}: Box, Hunter and Hunter, ``Statistics for Experimenters: Design, Innovation, and Discovery'', 2nd edition
					\item	Other references on course website (self-directed learning)
				\end{itemize}
			\column{0.20\textwidth}
				\vspace{-1cm}
				\begin{center}
					\includegraphics[width=\textwidth]{\imagedir/statistics/flickr-Box-Hunter-Hunter-cover-3056749047_1c1f633fcb_o.jpg}
				\end{center}
		\end{columns}
	\end{itemize}
\end{frame}

\begin{frame}\frametitle{Course feedback via Learning website}
	\begin{itemize}
		\item	I might not have explained something clearly;
		\item	You didn't get a chance to ask a question, \emph{etc}
	\end{itemize}
	\href{http://learnche.mcmaster.ca/feedback-questions}{http://learnche.mcmaster.ca/feedback-questions}
	\vspace{12pt}
	\hrule
	\begin{center}
		\includegraphics[width=0.65\textwidth]{\imagedir/teaching/anonymous-feedback.png}
	\end{center}
	\hrule
\end{frame}

\begin{frame}\frametitle{Course software}
	\begin{itemize}
		\item	A computer is required for assignments, DOE project, and take-home exam 
		\item	Main software: R statistical computing language; we also support Python, Minitab and MATLAB 
	\end{itemize}
	\begin{itemize}
		\item	Why use R? 
		\begin{itemize}
			\item	Widely used: Google, Pfizer, Merck, Bank of America, the InterContinental Hotels Group, Shell. 
			\item	Runs on Windows, Linux and Mac computers 
			\item	Excellent add-on libraries available for almost anything related to data analysis 
			\item	Free (both for academic and commercial use): you can use it after you graduate 
			\item	Promotes good statistical practice: write self-documenting code 
		\end{itemize}
	\end{itemize}
	\begin{itemize}
		\item	Tutorial on website 
		\begin{itemize}
			\item	\textbf{\href{http://connectmv.com/tutorials}{http://connectmv.com/tutorials}} 
			\begin{itemize}
				\item	R, MATLAB, and Python tutorials available there 
			\end{itemize}
			\item	How to install and use software 
			\item	Example of loading data, plotting, data analysis, etc 
		\end{itemize}
	\end{itemize}
\end{frame}

\begin{frame}\frametitle{Course software}	
	\includegraphics[height=\textheight]{\imagedir/statistics/New-York-Times-article-about-R.png}
\end{frame}

\begin{frame}\frametitle{Expectations outside class}
	\begin{itemize}
		\item	You can expect TAs and I to answer emails promptly 
		\item	If you have questions: 
		\begin{enumerate}
			\item	Please email the TAs with CC to me \hfill {\tiny{\color{myOrange}{$\longleftarrow$ hopefully this solves your problem}}}
			\item	Please send from your McMaster address
			\item	Set up in-person meeting with TAs or myself 
			\item	My office hours: Wednesday AM, Thursday PM and Friday PM.
		\end{enumerate}
	\end{itemize}
\end{frame}

\begin{frame}\frametitle{What this course is about}
	There are 6 main sections, spread over 12 weeks
	\begin{enumerate}
		\item	\emph{Visualization}: high-density, efficient graphics 
		\item	\emph{Univariate data analysis}: probability distributions, confidence intervals 
		\item	\emph{Process monitoring}: tracking process behaviour to detect abnormalities 
		\item	\emph{Least squares models}: correlation, covariance, ordinary and multiple least squares models 
		\item	Design and analysis of \emph{experimental data} and response surface methods to improve a process 
		\item	Introduction to \emph{latent variable methods}: a general overview 
	\end{enumerate}
\end{frame}

\begin{frame}\frametitle{We are surrounded by interesting data}
	\begin{center}
		\includegraphics[width=\textwidth]{\imagedir/mindmaps/data-sources.png}
	\end{center}
\end{frame}

\begin{frame}\frametitle{What this course is about}
	
	\textbf{Extracting value from data}
	
	Ankit - student in 2010 - now at Oneira Corporation (Oakville):
	\begin{itemize}
		\item	\emph{Now, having worked for over a year, I find myself referring back to my notes all the time and appreciating the concepts about how to look at data and represent the data in the best possible manner, especially since on a daily basis I look at a gigantic amount of data and am required to make sense of it.} 
	\end{itemize}
	\begin{itemize}
		\item	\emph{I think what I loved most about the course was the emphasis on the} \textbf{\emph{thinking}} \emph{and} \textbf{\emph{process of getting to a solution}} \emph{instead of the the final solution itself which has been an important attribute to becoming a good engineer or a problem solver/troubleshooter.} 
	\end{itemize}
	
	\vspace{12pt}
	\textbf{Tiffany at PepsiCo}
\end{frame}

\begin{frame}\frametitle{Section 1}
	\includegraphics[width=\textwidth]{\imagedir/visualization/visualization-subject-mapping.png}
\end{frame}

\begin{frame}\frametitle{Section 2}
	
	\includegraphics[width=\textwidth]{\imagedir/mindmaps/univariate-section-mapping.png}
\end{frame}

\begin{frame}\frametitle{Section 3}
	
	\includegraphics[width=\textwidth]{\imagedir/mindmaps/process-monitoring-section-mapping.png}
\end{frame}

\begin{frame}\frametitle{Section 4}
	
	\includegraphics[width=\textwidth]{\imagedir/mindmaps/least-squares-section-mapping.png}
\end{frame}

\begin{frame}\frametitle{Section 5}
	
	\includegraphics[width=\textwidth]{\imagedir/mindmaps/DOE-section-mapping.png}
\end{frame}

\begin{frame}\frametitle{Section 6}
	
	\includegraphics[width=\textwidth]{\imagedir/mindmaps/latent-variable-modelling-simple.png}
\end{frame}

\begin{frame}\frametitle{Enrichment topics}
	\begin{itemize}
		\item	robust methods 
		\item	cross-validation 
		\item	nonparametric methods 
		\item	real-time applications of statistical methods 
		\item	missing data handling 
	\end{itemize}
\end{frame}

\begin{frame}\frametitle{Grading}	
	What we look for in the grading is demonstration that you/group: 
	\begin{enumerate}
		\item	understand the concept, 
		\item	have the ability to apply the concept to new instances, 
		\item	think creatively about problems, 
		\item	my questions are seldom ``plug-and-chug'',
		\item	numerical accuracy, 
		\item	grammar and spelling.
	\end{enumerate}
	
	\vspace{12pt}
	Not all questions will be engineering related: 
	\begin{itemize}
		\item	mostly (chemical) engineering questions 
		\item	but also expect: current world events, public policy, bioengineering, anywhere there are data to analyze 
	\end{itemize}
\end{frame}

\begin{frame}\frametitle{Grading}
	\begin{itemize}
		\item	\emph{Appropriate} group work is highly encouraged 
		\begin{itemize}
			\item	Up to 30\% of course grade 
			\item	\emph{Learn with each other} 
			\item	Assignments done in groups of 2 or by yourself 
			\item	Hand-in assignments as one submission 
			\item	Extra credit for 4C3 students if you do the 6C3 questions
		\end{itemize}
		\item	Late grading 
		\begin{itemize}
			\item	\( -30 \)\% per day
			\item	2 ``late day'' credits for assignments 
			\item	solutions posted after $\approx 2$ days of due date 
		\end{itemize}
		\item	Assignment grading: 
		\begin{itemize}
			\item	No make-ups for assignments 
			\item	Counts \textbf{20\%} of course grade 
			\item	Best $N-1$ assignments ($N \approx 7$) will be used 
		\end{itemize}
		\item	Assignment dates: see website 
	\end{itemize}
\end{frame}

\begin{frame}\frametitle{Group-based assignments}
	\begin{itemize}
		\item	``Appropriate'' group work is highly encouraged (about 40\% of course)
		\item	Learn with each other: groups of 2, no larger
		\vspace{2pt}\hrule\vspace{2pt}
		\item	Optimal group work: \emph{an example of one approach}
			\begin{itemize}
				\item	Sarah and Brad work on an assignment
				\item	Both Sarah and Brad do {\color{myRed}{\textbf{all questions}}} in draft: quick notes at home, on the bus, \emph{etc}, $\pm 4$ days before assignment due
				\pause
				\item	Meet in the library next day and go over each other's notes
				\item	Explain to the other why you disagree
				\item	e.g. Sarah sees a mistaken interpretation in Brad's work
				\begin{itemize}
					\item	She explains why it is a mistake to Brad: Sarah learns
					\item	Brad also learns: he's heard this in class, and from Sarah now
					\item	If neither can resolve it? speak with TA or Kevin					
				\end{itemize}
				\pause
				\item	Write up a joint solution; e.g. Sarah Q1 and 2, Brad does Q3
				\item	Both review it before submitting
			\end{itemize}
		\vspace{2pt}\hrule\vspace{2pt}
		\pause		
		\item	Other approaches are possible: your group decides
		\item	\color{myOrange}{What doesn't work}: Sarah does Q1 and Q2, Brad does Q3; staple and submit
		\item	\textbf{Do not share files or written work} \emph{between} groups 
	\end{itemize}
\end{frame}

\begin{frame}\frametitle{Weekly tests}
	\begin{exampleblock}{}
		Experiments are part of statistics; weekly tests are an experimental feature of 4C3/6C3 in 2013. Like any clinical trial, I reserve the right to stop the trial or keep going, based on data collected while running the trial.
	\end{exampleblock}

	\begin{itemize}
		\item	\textbf{Fact}: frequent small tests help understand and retain material
		\item	``\emph{{\color{purple}{Testing effect}}}'': probability of remembering a tested item (``an item that has to be recalled from memory'') is greater than an item that was simply \emph{studied}. Coupled with ``feedback'', this effect can be enhanced (but timing of feedback is important).
		\item	``{\color{purple}{Spacing effect}}'': breaks between multiple reviews of material, increasing the duration of the breaks, improves recall.
			\begin{itemize}
				\item	``{\color{myGreen}{\emph{tent-in-the-wind-with-pegs effect}}}''
			\end{itemize}
		
		\item	This approach has been successfully used at McMaster in other courses. I'd like to use it this year and retain it, if successful. Perfectly suited to 4C3/6C3.
	\end{itemize}
\end{frame}

\begin{frame}\frametitle{Weekly tests}
	\begin{itemize}
		\item	More details \emph{coming soon} on the website
		\item	Testing window: 18:00 on Sunday night until 03:00 Tuesday morning (a 33-hour window)
		\item	Test is only about 1 hour in duration, depending on questions
		\item	Covers work reviewed in class the prior week
		\item	However there will occasionally be questions from the weeks before that
		\item	Feedback, where possible, will be automatically given immediately after the window closes
		\item	Each test counts 2\% of your grades; total of \textbf{22\%}
		\item	Replaces the two midterms (10\% and 15\% that were previously used)
		\item	No make-ups; {\color{myGreen}{\textbf{much lower stakes than the 2 midterms}}}
		\item	Any notes, materials, websites, electronic documents are allowed, and encouraged.
		\item	\textbf{\emph{Code of honour}}: you must do the test on your own
	\end{itemize}
\end{frame}

\begin{frame}\frametitle{Grading for project and exams}
	\begin{itemize}
		\item	Midterms:
		\begin{itemize}
			\item	only if weekly tests are cancelled; optional; no make-up 
		\end{itemize}
	\end{itemize}
	\begin{itemize}
		\item	Written final exam: \textbf{48\%} 
		\begin{itemize}
			\item	Covers all material 
			\item	\textbf{{\color{myRed}{You must achieve 50\% or greater in final exam to pass 4C3/6C3}}}
		\end{itemize}
	\end{itemize}
	\begin{itemize}
		\item	Midterm and final exam: 
		\begin{itemize}
			\item	Open notes -- anything on paper is allowed 
			\item	No electronic devices unfortunately 
			\item	Any calculator 
		\end{itemize}
	\end{itemize}
	\begin{itemize}
		\item	Experimental report due on 26 March: \textbf{10\%} 
		\begin{itemize}
			\item	Perform you own \emph{designed experiment} 
			\item	More details \href{http://learnche.mcmaster.ca/4C3/Designed_experiments_project_-_2013}{on the course website} already
			\item	The earlier you start, the better (start early March)
			\item	You can, and should, provide an outline of your experimental plan for me to review by 06 March 
			\item	Can collaborate, \textbf{but only within your group}: not between groups 
		\end{itemize}
	\end{itemize}
\end{frame}

\begin{frame}\frametitle{Important dates}
	\begin{itemize}
		\item	Seminar this Thursday, 10:30, JHE 326H ``\emph{{\color{myGreen}{Six Sigma for Chemical Engineers}}}''
		\item	15 February: (potential mid-term)
		\item	06 March: project outline due 
		\item	26 March: course project due
		\item	10 April: last, review class
		\item	NN April: \textbf{Final-exam }
	\end{itemize}
	\begin{itemize}
		\item	Due dates for assignments: see course website 
	\end{itemize}
\end{frame}