\begin{frame}\frametitle{Topics covered in this section on data visualization}
	\begin{center}
		\includegraphics[width=\textwidth]{\imagedir/visualization/visualization-subject-mapping.png}
	\end{center}
\end{frame}

\begin{frame}\frametitle{Some ways you can impress your colleagues and manager in the future}
	\begin{itemize}
		\item	\emph{Co-worker}: Here are the yields from a batch system for the last 3 years (1256 data points), can you help me: 
		\begin{itemize}
			\item	understand more about the time-trends in the past 3 year? 
			\item	efficiently summarize the yield from all batches run in 2010?
		\end{itemize}
		\item	\emph{Manager}: effectively summarize the (a) number and (b) types of defects on 17 aluminum grades for the past 12 months 
		\item	\emph{Yourself}: 24 different variables being measured \emph{vs} time (5 readings per minute, over 300 minutes) for each batch we produce; how can we visualize these 36,000 data points? 
		\begin{itemize}
			\item	see next slides
		\end{itemize}
	\end{itemize}
\end{frame}

\begin{frame}\frametitle{Batch systems generate large quantities of \emph{extremely} valuable data}
	\begin{columns}[t]
		\column{0.50\textwidth}
			\begin{center}
				\includegraphics[width=.8\textwidth]{\imagedir/batch/batch-system.png}
			\end{center}
			\see{From Cecilia Rodrigues' M.A.Sc thesis, 2006, McMaster University, used with permission}
			
		\column{0.45\textwidth}
			\begin{center}
				\includegraphics[width=\textwidth]{\imagedir/batch/flickr-batch-reactor-2516220152_074fbbb489_o.jpg}
			\end{center}
			\see{Flickr: \href{http://www.flickr.com/photos/polapix/2516220152/}{\#2516220152}}
	\end{columns}
\end{frame}

\begin{frame}\frametitle{Batch systems generate large quantities of \emph{extremely} valuable data}
	\begin{columns}[t]
		\column{0.60\textwidth}
			Data from a single batch
			\begin{center}
				\includegraphics[width=\textwidth]{\imagedir/examples/fmc/fmc-phases-4-trajectories.png}
			\end{center}
		\column{0.40\textwidth}
			Data from many batches
			\begin{center}
				\includegraphics[width=.8\textwidth]{\imagedir/examples/fmc/unaligned-data.png}
			\end{center}
	\end{columns}
\end{frame}

\begin{frame}\frametitle{Recommended references for this section: Data visualization}
	\begin{enumerate}
		\item	Edward Tufte, \emph{Envisioning Information}, Graphics Press, 1990. (10th printing in 2005) 
		\item	Edward Tufte, \emph{The Visual Display of Quantitative Information}, Graphics Press, 2001. 
		\item	Edward Tufte, \emph{Visual Explanations: Images and Quantities, Evidence and Narrative}, 2nd edition, Graphics Press, 1997. 
		\item	William Cleveland, \emph{Visualizing Data}, and \emph{The Elements of Graphing Data}, Hobart Press; 2nd edition, 1994. 
		\item	Stephen Few, \emph{Show Me the Numbers}, and ``Now You See It'', Analytics Press. 
		\item	Su, It's easy to produce chartjunk using Microsoft Excel 2007 but hard to make good graphs, \emph{Computational Statistics and Data Analysis}, \textbf{52} (10), 4594-4601, 2008, \href{http://dx.doi.org/10.1016/j.csda.2008.03.007}{http://dx.doi.org/10.1016/j.csda.2008.03.007} 
	\end{enumerate}
\end{frame}

\begin{frame}\frametitle{Why bother learning about this topic: it's too easy!}
	
	This class might seem too easy; too obvious. It is!
	\begin{itemize}
		\item	The human eye and brain are excellent at pattern recognition, sorting through signal and noise.
		\item	We can easily cope with bad plots; but good plots save time and show a clearer, more honest picture.   \pause 
		\item	Cliches: ``Let the data speak for themselves'', ``Plot the data'' 
	\end{itemize}
	 \pause 
	
	\vspace{12pt}
	\begin{exampleblock}{}
		{\color{myGreen}\textbf{Strong suggestion:}} find a bad plot (journal publications, an old lab report that you have written); upload it to the forums and criticize the plot. Why is it bad?
	\end{exampleblock}
	
\end{frame}

\begin{frame}\frametitle{We need good plots to make decisions quickly, correctly, and confidently} 
	
	\vspace{6pt}
	\centerline{\includegraphics[width=.78\textwidth]{\imagedir/visualization/flickr-7848902570_cde4f30ac0_o.jpg}}
	\see{Flickr: \href{https://www.flickr.com/photos/twosevenoneonenineeightthreesevenatenzerosix/7848902570/in/photostream/player/}{\#7848902570}}
	
	
\end{frame}

\begin{frame}\frametitle{Time-series plots show a univariate piece of information in 2 dimensions}
		\begin{itemize}
			\item	(usually) have the horizontal $x$-axis show time or sequence order 
			\item	the other axis: the data values 
		\end{itemize}
		\centerline{\includegraphics[width=.78\textwidth]{\imagedir/monitoring/flotation-feedrate-EWMA-time.png}}
	\pause
	\begin{itemize}
		\item	Our eyes can deal with high data density, sinusoids, spikes, patterns, can separate noise from signal, and recognize outliers.
	\end{itemize}
\end{frame}

\begin{frame}\frametitle{An example of a bad time-series plot; what problems can you identify?}

	
	\centerline{\includegraphics[width=\textwidth]{\imagedir/visualization/CPU-temperature_-_from_www_aw_org_on_26_Dec_2009.png}}
	(and bad axis scaling and colour choices)
	
	% Terrible x-axis: repeated "Saturday"; no context (e.g. is it 2008, 2009?)
	% Wasted white space at the bottom
	% Colour scheme is poorly selected: green on red
	% Rather show error in lighter colour and the smoothed trend as a darker colour
\end{frame}

\begin{frame}\frametitle{Notice how the plot's ``message'' is entirely different now}
	\includegraphics[width=\textwidth]{\imagedir/visualization/CPU-temperature_-_from_www_aw_org_on_26_Dec_2009-fixed}
	
	A first attempt at fixing the prior visualization
\end{frame}

\begin{frame}\frametitle{Poor plots; default settings in plotting software create cluttered plots}
	\vspace{2pt}
	\centerline{\includegraphics[width=\textwidth]{\imagedir/visualization/3_correlated_variables_-_badly_displayed_in_Numbers.png}}
	\begin{itemize}
		\item	The plot has been slightly improved here:
	\end{itemize}
	\centerline{\includegraphics[width=\textwidth]{\imagedir/visualization/3_correlated_variables_-_slightly_better.png}}
\end{frame}

\begin{frame}\frametitle{Use separate, parallel axes rather to compare plots (and minimal data ink)}
	
	\centerline{\includegraphics[width=\textwidth]{\imagedir/visualization/3_correlated_variables_-_better.png}}
	
	\vspace{12pt}
	\color{myOrange}{These non-default settings can take a long time to set (10 minutes for this example)}
\end{frame}

\begin{frame}\frametitle{Sparklines are a type of time-series plot}
	
	\begin{itemize}
		\item	except, we omit the horizontal and vertical axes (they are implicit)
		\item	Read more about them from \href{http://yint.org/sparklines}{http://yint.org/sparklines}
	\end{itemize}
	
	\begin{exampleblock}{}
		\centerline{\includegraphics[width=1in]{\imagedir/visualization/3-correlated-variables-as-sparklines.png}}
	\end{exampleblock}
	
	
	\begin{itemize}
		\item	Useful for financial trends 
		\item	Built into Excel 2010 
		\item	Great for iPods, cell phones, tablet computers
		\begin{itemize}
			\item	because they are of high density and small size
		\end{itemize}
		\item	Our eye can detect 250 dots (points) per linear inch and 650 points per square inch.
	\end{itemize}
\end{frame}

\begin{frame}\frametitle{Sparklines are used on various websites now to show high-density graphics}

	\centerline{\includegraphics[width=\textwidth]{\imagedir/visualization/Google-sparklines.png}}
	\begin{columns}[T]
		
		\column{0.5\textwidth}
			{\scriptsize Screenshot from Google Finance 08 January 2014}	

			\vspace{5pt}
			Notice how you clearly detect correlations in stock prices (stocks that move together).
			
		
		\column{0.5\textwidth}
			\vspace{-12pt}
			\centerline{\includegraphics[width=.6\textwidth]{\imagedir/visualization/Drupal-bug-tracking.png}}
			{\scriptsize  Screenshot from Drupal website to track software bugs.}
	\end{columns}
\end{frame}

\begin{frame}\frametitle{Example of sparklines in everyday use}
	
	\begin{center}
		\includegraphics[width=\textwidth]{\imagedir/classification/12leadECG.jpg}
	\end{center}
	\see{\href{http://en.wikipedia.org/wiki/File:12leadECG.jpg}{Wikipedia: File:12leadECG.jpg}}
\end{frame}

\begin{frame}\frametitle{Keep the x-axis spacing constant on time-series plots: helps interpretation}
	
	\begin{itemize}
		\item	Use another plot (e.g. below the original) to zoom in on details 
	\end{itemize}

	\centerline{\includegraphics[width=.7\textwidth]{\imagedir/visualization/magnifying-sciencedirect-10.1016-j.apgeochem.2008.05.006.jpg}}

	\vspace{-14pt}
	\see{DOI: 10.1016/j.apgeochem.2008.05.006}
\end{frame}

\begin{frame}\frametitle{Provide an honest message to your viewer}
	
	Adjust for inflation when plotting money values against time.
	
	\centerline{\includegraphics[width=.7\textwidth]{\imagedir/visualization/Google-Finance-stock-prices.png}}
	{\scriptsize Screenshot from Google Finance.}
	
	
	\pause
	\vspace{12pt}
	\begin{itemize}
		\item	Example of car sales: \href{http://www.duke.edu/~rnau/411infla.htm}{http://www.duke.edu/$\sim$rnau/411infla.htm} 
	\end{itemize}
	
\end{frame}

\begin{frame}\frametitle{Show a reasonable amount of historical data for context}
	\centerline{\includegraphics[width=.85\textwidth]{\imagedir/visualization/AAPL-stock-prices.png}}
\end{frame}

\begin{frame}\frametitle{Important learning points from time-series plots}
	
	\begin{itemize}
		\item	avoid using colour as your message
		\item	use honest scaling on your $x$-axis and $y$-axis
		\item	the human eye (and brain) can deal with vast quantities of data: exploit it
	\end{itemize}
	
\end{frame}

\begin{frame}\frametitle{Bar plots are univariate plots, on a 2-D axis}
	\centerline{\includegraphics[width=0.9\textwidth]{\imagedir/visualization/barplot-example-expenses.png}}
	\vspace{-8pt}
	
	\pause
	{\color{myOrange}Use a bar plot when you have many categories, and the literal interpretation does not depend on category order.}
\end{frame}

\begin{frame}\frametitle{Very different messages come across, even though the data are identical}
	\begin{columns}[b]
		
		\column{0.5\textwidth}
			There's no direct message here 
			
			\vspace{12pt}
			\centerline{\includegraphics[width=\textwidth]{\imagedir/visualization/barplot-example-expenses-unordered.png}}
			
		
		\column{0.5\textwidth}
			\onslide+<2->{
			This message is more clear: the reader can quickly see their greatest expenses 
			
			\vspace{12pt}
			\centerline{\includegraphics[width=\textwidth]{\imagedir/visualization/barplot-example-expenses-label-removed.png}}
			}
		
	\end{columns}
\end{frame}

\begin{frame}\frametitle{You should not use a bar plot to show time-series data}
	
	Rather use a time-series plot, which is much less wasteful and shows the trends more clearly.
	\begin{center}
		\includegraphics[width=.9\textwidth]{\imagedir/visualization/quarterly-profit-barplot-vs-lineplot.png}
	\end{center}
\end{frame}

\begin{frame}\frametitle{Bar plots can be wasteful as each data point is repeated several times:}
	
	\begin{columns}[c]
		\column{6cm} 
			\begin{center}
				\includegraphics[width=1.2\textwidth]{\imagedir/visualization/quarterly-profit-barplot-only.png} 
			\end{center}
		\column{4cm} 
			\begin{enumerate}
				\item	left edge (line) of each bar 
				\item	right edge (line) of each bar 
				\item	the height of the colour in the bar 
				\item	the number's position (up and down along the y-axis) 
				\item	the top edge of each bar, just below the number 
				\item	the number itself 
			\end{enumerate}
	\end{columns}
\end{frame}

\begin{frame}\frametitle{A general plotting principle: ``Maximize thee data ink ratio'', within reason}
	\begin{exampleblock}{}
		\begin{center}
			\[
				\begin{array}{rl}
					\text{Maximize data ink ratio} 	&= \dfrac{\text{total ink for data}}{{\text{total ink for graphics}}} \\ \\
													&= 1 - \text{proportion of ink that can be erased} \\
													&\qquad\qquad \text{without loss of data information} 
			\end{array}
			\]
		\end{center}
	\end{exampleblock}
	For example, rather {\color{myOrange}{use a table}} for a handful of data points: 
	\begin{center}
		\includegraphics[width=\textwidth]{\imagedir/visualization/profit-by-region.png}
	\end{center}
\end{frame}

\begin{frame}\frametitle{Don't use cross-hatching, textures, or unusual shading in the plots: it creates visual vibrations }

	\begin{center}
		\includegraphics[height=0.9\textheight]{\imagedir/visualization/hatched-barplot.png}
	\end{center}
\end{frame}

\begin{frame}\frametitle{Worst bar plot ever?}
	\begin{center}
		\includegraphics[width=.9\textwidth]{\imagedir/visualization/worst-examples/worst-bar-plot.jpg}
	\end{center}
	Actual example from a ``production report'' board at a company.
\end{frame}

\begin{frame}\frametitle{Bar plots often benefit from a horizontal presentation, especially if}
		\begin{itemize}
			\item	there is a some ordering to the categories 
			\item	the labels do not fit side-by-side
		\end{itemize}
		\begin{columns}[c]
			\column{0.70\textwidth}
				\centerline{\includegraphics[height=.8\textheight]{\imagedir/visualization/bar-plot-example-expenses-in-bar-labels.png}}
			\column{0.30\textwidth}
				
				{\color{myGreen} You can place the labels inside the bars
				
				\vspace{24pt}
				You should usually start the non-category axis at zero.}
		\end{columns}
\end{frame}

\begin{frame}\frametitle{Unnecessary plot embellishments are not required}
	\begin{itemize}
		\item	Avoid unnecessary ``extras'' to enliven the plot 
		\item	``\emph{If the statistics are boring, then you've got the wrong numbers}'' [Tufte]
	\end{itemize}
	\begin{center}
		\includegraphics[height=0.60\textheight]{\imagedir/visualization/Toronto-Star-Mortgage-Rates.png}
		\see{Toronto Star, 2010}
	\end{center}
\end{frame}

\begin{frame}\frametitle{Consider a vector of temperature data }
	For example, this is the weather in Hamilton, Ontario, on 8 January 2015
	
	\centerline{\includegraphics[width=\textwidth]{\imagedir/visualization/weather-data-Hamilton-Canada-08-Jan-2015.png}}
\end{frame}

\begin{frame}\frametitle{Fences and outliers illustrated on a box plot}
	
	\centerline{\includegraphics[width=\textwidth]{\imagedir/visualization/boxplot-outliers-outside-fence.png}}
	
	\vspace{24pt}
	Outliers will be defined later: they are unusual data points that are far away from the ``bulk'' of the data.
\end{frame}

\begin{frame}\frametitle{Box plots: compared to a pure normal distribution}
	\begin{center}
		\includegraphics[width=.67\textwidth]{\imagedir/visualization/wikipedia-Boxplot_vs_PDF.svg.png}
	\end{center}
	\vspace{-16pt}
	\see{\href{http://en.wikipedia.org/wiki/File:Boxplot_vs_PDF.svg}{Wikipedia} has some really great illustrations to explain statistical concepts, such as this plot}
\end{frame}

\begin{frame}\frametitle{Case study: lumber cutting}
	\begin{columns}[c]
		\column{0.30\textwidth}
			
			The log is completely scanned by lasers, and a few seconds later a computer determines lumber cuts that will maximize the  economic value per log.
			
			\vspace{12pt}
			The log is rotated and guided into rigid saw blades to achieve the predicted result.
			
		\column{0.70\textwidth}
			\vspace{-24pt}
			\begin{center}
				\includegraphics[height=\textheight]{\imagedir/examples/lumber-monitoring/lumber-profile.png}
			\end{center}
	\end{columns}
\end{frame}

\begin{frame}\frametitle{Case study: lumber cutting}
	
	After cutting, the thickness is measured at 6 locations; target = 1680 mils
	\begin{center}
		\includegraphics[width=\textwidth]{\imagedir/examples/board-thickness/board_measurement_locations.png}
	\end{center}
	\vfill
	Actual thickness of a 2x6 is = 1500 mils; a little extra is added to compensate for the lumber drying out
\end{frame}

\begin{frame}\frametitle{Box plots are very effective for comparing similar variables (in the same units of measurement)}
	\begin{center}
		\includegraphics[height=0.8\textheight]{\imagedir/visualization/boxplot-for-two-by-six-100-boards.png}
	\end{center}
\end{frame}

\begin{frame}\frametitle{Box plots: some alternatives you might see in practice}
	
	There is no agreed on definition:
	\begin{itemize}
		\item	can use the mean instead of the median 
		\item	outliers shown as dots, where an outlier is most commonly defined as any point 1.5 IQR distance units above and below the median. 
		\item	use the 2nd percentile (instead of $\text{median} - 1.5\cdot\text{IQR}$) 
		\item	use the 98th percentile (instead of $\text{median} + 1.5\cdot\text{IQR}$) 
		\item	add the density histogram onto the box plot: \emph{violin plot} 
		\begin{itemize}
			\item	Now we can see some of the distortion at positions 1 and 3 (next slide)
		\end{itemize}
	\end{itemize}
\end{frame}

\begin{frame}\frametitle{Variations on a theme: the violin plot as an alternative to the box plot}
	
	\vspace{-5pt}
	\centerline{\includegraphics[width=0.6\textwidth]{\imagedir/univariate/violin-plot.png}}
\end{frame}

\begin{frame}\frametitle{Scatter plots help understand the relationship between two variables}
	\begin{columns}[c]
		\column{0.50\textwidth}
			\begin{itemize}
				\item	It is a two dimensional plot of two variables (vectors). \pause
				\item	Each marker is the intersection of the values from the data vectors.
			\end{itemize}
		\column{0.60\textwidth}
			\centerline{\includegraphics[width=\textwidth]{\imagedir/visualization/explain-scatterplot-R.png}}
				% Video animation also used these two other .PNG files
				 	% explain-scatterplot-R-no-arrows.png}}
					% explain-scatterplot-R-no-labels.png}}
	\end{columns}
		
	\begin{block}
		{Intention of a scatter plot} 
		\begin{center}
			Asks the viewer to draw a causal relationship between the two variables 
		\end{center}
	\end{block}
\end{frame}

\begin{frame}\frametitle{A scatter plot showing a cause-and-effect relationship}
	\begin{center}
		\includegraphics[height=0.85\textheight]{\imagedir/visualization/scatterplot-temperature-vs-vapour-pressure.png}
	\end{center}
	\vspace{-24pt}
	{\color{myOrange}And in many cases, that causal relationship actually exists.}
\end{frame}

\begin{frame}\frametitle{Scatter plots}
	However, not all scatter plots show causal phenomenon. 
	\begin{center}
		\includegraphics[height=0.75\textheight]{\imagedir/visualization/scatterplot-white-hairs-vs-BMD.png}
	\end{center}
	\pause
	\vspace{-12pt}
	{\color{myOrange}Student 2013, Hawra: {\small ``Although scatter plots may imply a cause and effect relationship exists, it is not a `tool' to test the existence of a possible relationship.''}}
\end{frame}

\begin{frame}\frametitle{Three variables: so 3 scatter plot combinations could have been drawn. Which are correlations, and which are actually cause-and-effect?}
	\begin{center}
		\includegraphics[height=0.75\textheight]{\imagedir/doe/yield-pressure-impurity-correlation.png}
	\end{center}
	
	\vspace{-6pt}
	\color{blue}{We will answer the question in the Experiments section.}
\end{frame}


\begin{frame}\frametitle{Scatter plots: is there cause and effect here?}
	\begin{center}
		\includegraphics[width=.8\textwidth]{\imagedir/visualization/scatterplot-GDP-working-hours.png}
	\end{center}
	\href{http://www.economist.com/blogs/freeexchange/2013/09/working-hours}{http://yint.org/working-hours} explains the relationship and the data source.
\end{frame}

% But there are several cases where they do: lead in water vs number of crimes committed

\begin{frame}\frametitle{Scatter plots can be greatly improved from the software defaults by:}
	\begin{itemize}
		\item	making each axis as tight as possible 
		\item	avoiding heavy grid lines 
		\item	use the least amount of ink 
		\item	not distorting the axes 
	\end{itemize}
\end{frame}

\begin{frame}\frametitle{Note the tight axes, low amount of data ink: scatter plots are efficient.}
	\begin{columns}[T]
		\column{0.33\textwidth}
			\centerline{\includegraphics[width=\textwidth]{\imagedir/visualization/scatterplot-temperature-vs-vapour-pressure.png}}
			
		\column{0.33\textwidth}
			\centerline{\includegraphics[width=\textwidth]{\imagedir/visualization/scatterplot-white-hairs-vs-BMD.png}}
			
		\column{0.33\textwidth}
			\centerline{\includegraphics[width=\textwidth]{\imagedir/visualization/scatterplot-GDP-Hours-reduced.png}}
	\end{columns}
	
\end{frame}


\begin{frame}\frametitle{There is an unfounded fear that others won't understand your scatter plot}
	
	\begin{itemize}
		\item	Plant control room: seldom see scatter plots. 

		\item	Tufte study (VDQI, \href{http://yint.org/vdqi}{http://yint.org/vdqi}): no scatter plots in a sample of Western daily newspapers (1974 to 1980)
		\item	Japanese newspapers frequently use scatterplots 
		\item	He shows 12 year olds can interpret such plots. 
	\end{itemize}
	
	\pause
	\begin{block}
		{Key point} 
		\begin{center}
			The producers of charts must assume their audience is capable of interpreting them. 
			
			\vspace{12pt}
			{\color{red}Rather, assume that if you can understand the plot, so will your audience. }
		\end{center}
	\end{block}
\end{frame}

\begin{frame}\frametitle{Take a diversion to watch this YouTube video \href{http://www.youtube.com/v/jbkSRLYSojo}{http://yint.org/rosling-video}}
	\vspace{2pt}
	\centerline{\includegraphics[width=\textwidth]{\imagedir/visualization/Rosling-YouTube-screenshot.png}}
	
	Watch a video explanation of these data: \href{http://www.youtube.com/v/jbkSRLYSojo}{http://yint.org/rosling-video} 
\end{frame}

\begin{frame}\frametitle{Hans Rosling has used these to great effect to illustrate issues related to International Health}
	\begin{columns}[T]
		\column{0.60\textwidth}
			\centerline{\includegraphics[width=\textwidth]{\imagedir/visualization/scatterplot-with-2-extra-dimensions.png}}
			
		\column{0.45\textwidth}
			Variables shown in the figure:
			\begin{enumerate}
				\item	$x$-axis: income per person
				\item	$y$-axis: children per woman
				\item	marker area: population
				\item	colour: life-expectancy \scriptsize{[20 to 80]}
				\item	time-based animation: on the GapMinder website you can ``play''the graph over time  
			\end{enumerate}
			The \href{http://gapminder.org}{http://gapminder.org} site allows you to select many interesting variables on the axes.
	\end{columns}
	
	\vspace{10pt}
	Watch a video explanation of these data: \href{http://www.youtube.com/v/jbkSRLYSojo}{http://yint.org/rosling-video} 
\end{frame}

\begin{frame}\frametitle{3D glasses used to visualize process data in 6-dimensions?}
	It is possible with 3D glasses.
	
	\centerline{\includegraphics[height=.8\textheight]{\imagedir/visualization/flickr-3D-glasses-8443551051_3d746ac359_o.jpg}}
	\see{Flickr: \href{https://www.flickr.com/photos/smieyetracking/8443551051}{\#8443551051}}
	
\end{frame}


\begin{frame}\frametitle{Scatter plots lose density information: recover it with some jitter}
	\includegraphics[width=\textwidth]{\imagedir/visualization/scatterplots-with-jitter.png}
	
	\vspace{-12pt}
	R code:
	\begin{columns}[T]
		\column{0.50\textwidth}
			 {\scriptsize \texttt{~~~~~~~~~~~plot(education, vocabulary)}}
			 
			
			
		\column{0.50\textwidth}
			{\scriptsize \texttt{plot(jitter(education), jitter(vocabulary))}}
	\end{columns}
	
\end{frame}

\begin{frame}\frametitle{Recover distribution and spread information with box plots}
	\includegraphics[width=\textwidth]{\imagedir/visualization/scatterplot-figures-with-regression-lines.png}
	
	Used the \texttt{scatterplot(...)} function from the \texttt{library(car)} in R to create these.
\end{frame}

\begin{frame}\frametitle{Consider adding histograms when you are exploring the data to learn about the system}
	\begin{center}
		\includegraphics[height=0.80\textheight]{\imagedir/visualization/scatterplot-with-histograms.png}
	\end{center}
\end{frame}

\begin{frame}\frametitle{Investigate this plot in your own time}
	\begin{columns}[T]
		\column{0.50\textwidth}
			\vspace{-12pt}
			\begin{center}
				\includegraphics[width=\textwidth]{\imagedir/visualization/MotherJones-Lead-Crime-correlation.png}
			\end{center}
		\column{0.55\textwidth}
			\begin{itemize}
				\item	Why did the author use a time-series plot to show correlation?
				\item	Would the plot be more informative as 2D-scatter plot?
				\item	Redraw a rough version of this plot as a scatter plot instead.
				\item	What if you were to repeat this analysis for multiple regions/countries/cities. How would you show (visualize) the correlations effectively?
				\item	Read the full article for details: \href{http://www.motherjones.com/environment/2013/01/lead-crime-link-gasoline}{http://yint.org/lead-and-crime}
			\end{itemize}
			\hrule
			\vspace{4pt}
			%{\hfill Pb(CH$_2$CH$_3$)$_4$}
	\end{columns}
\end{frame}

\begin{frame}\frametitle{Some data visualization blogs worth reading}
	\begin{center}
		\includegraphics[height=0.9\textheight]{\imagedir/visualization/data-visualization-blogs.png}
	\end{center}
\end{frame}

\begin{frame}\frametitle{Tables}
	
	Tables are for \textbf{\emph{comparative}} data analysis on \textbf{\emph{categorical objects}}.
	\begin{center}
		\includegraphics[width=\textwidth]{\imagedir/visualization/table-car-payments.png}
	\end{center}
	\begin{itemize}
		\item	\textbf{categorical objects}: the cars
		\item	Note the rows are in \emph{default} alphabetical order. 
		\item	We can make the table ``tell a story'' if we reorder the rows by some other variable. 
		\begin{itemize}
			\item	e.g. monthly insurance payment 
		\end{itemize}
	\end{itemize}
\end{frame}

\begin{frame}\frametitle{Tables}
	\begin{itemize}
		\item	Compare defect types (columns) for different product grades (rows)
		\item	Categorical variables appear in the \textbf{rows} and \textbf{columns} here
	\end{itemize}
	\begin{center}
		\includegraphics[width=.7\textwidth]{\imagedir/visualization/table-defect-counts.png}
	\end{center}
	\vspace{-12pt}
	\begin{itemize}
		\item	Which defects cost us the most money? 
	\end{itemize}
\end{frame}

\begin{frame}\frametitle{Tables}
	\begin{itemize}
		\item	Defect frequency 
		\begin{itemize}
			\item	If 1850 lots of grade A4636 (first row): defect A rate = 1/50 
			\item	If 250 lots of grade A2610 (last row): defect A rate = 1/50 
			\item	Redraw table on production rate basis 
		\end{itemize}
		\item	If comparing defects over different grades: go down the table (show fraction within the column) 
		\item	If comparing defects within grade: go across table (show fraction with the row) 
		\begin{itemize}
			\item	Could weight each column by cost of defect 
		\end{itemize}
	\end{itemize}
\end{frame}

\begin{frame}\frametitle{Tables}
	Common pitfalls: using pie charts when tables will do 
	\begin{center}
		\includegraphics[width=.8\textwidth]{\imagedir/visualization/where-MACC-alumni-work.png}
	\end{center}
	\hrule
	\vspace{4pt}
	I cannot explain the pitfalls of pie charts as well as Stephen Few does: \href{http://www.perceptualedge.com/articles/08-21-07.pdf}{Save the pies for dessert} (please read)
\end{frame}

\begin{frame}\frametitle{Tables vs pie charts: plenty of bad examples}
	\begin{columns}[t]
		\column{0.5\textwidth}
			\begin{center}
				\includegraphics[width=\textwidth]{\imagedir/visualization/Gratitous-pie-chart-www.theglobeandmail.com-with-this-budget-ontario-is-banking-on-the-near-impossible-article1510634.png}
			\end{center}
			\begin{center}
				\includegraphics[width=\textwidth]{\imagedir/visualization/worst-examples/not-an-effective-visualization-SDL-4N4-2012.png}
			\end{center}
		\column{0.5\textwidth}
			\begin{center}
				\includegraphics[width=\textwidth]{\imagedir/visualization/worst-examples/another-really-bad-visualization-SDL-4N4-2012.png}
			\end{center}
			\begin{center}
				\includegraphics[width=\textwidth]{\imagedir/visualization/worst-examples/idiotic-pie-chart-SDL-4N4-2012.png}
			\end{center}
	\end{columns}
	\vspace{12pt}
	\see{\href{http://www.theglobeandmail.com/news/politics/with-this-budget-ontario-is-banking-on-the-near-impossible/article4312189/}{Globe and Mail, March 2010} (top left); SDL reports, 4N4, 2012 (all others)}
\end{frame}

\begin{frame}\frametitle{Tables}
	2. arbitrarily ordering of the rows 
	\begin{center}
		\includegraphics[width=\textwidth]{\imagedir/visualization/table-car-payments.png}
	\end{center}
\end{frame}

\begin{frame}\frametitle{Tables}
	3. using excessive grid lines 
	\begin{center}
		\includegraphics[width=\textwidth]{\imagedir/visualization/table-grid-comparison.png}
	\end{center}
\end{frame}

\begin{frame}\frametitle{Tables}
	Interesting example: comparing two treatments
	\begin{center}
		\includegraphics[width=\textwidth]{\imagedir/visualization/two-treatments-table.png}
	\end{center}
	\begin{itemize}
		\item	Coating A or B are applied to different products
		\item	K-series, P-series, S-series
		\item	How does the coating affect corrosion and surface roughness?
	\end{itemize}
\end{frame}

\begin{frame}\frametitle{Tables}
	\begin{center}
		\includegraphics[width=.7\textwidth]{\imagedir/visualization/tables-recast-as-plots-both.png}
	\end{center}	
\end{frame}

\begin{frame}\frametitle{Sankey diagrams}
	\begin{center}
		\includegraphics[width=.67\textwidth]{\imagedir/visualization/Sankey-diagram-Canada-energy-flows.png}
	\end{center}
	\see{\href{http://gnewton.ca/gn/sankey/sankey_canada_energy_flow_2007.html}{http://gnewton.ca/gn/sankey/sankey\_canada\_energy\_flow\_2007.html}}
\end{frame}

% \begin{frame}\frametitle{Data frames}
% 	
% 	Frames are the basic containers that surround the data and give context to our numbers. Here are some tips:
% 	\begin{enumerate}
% 		\item	Use round numbers 
% 		\item	Tighten the axes as much as possible, except ... 
% 		\item	when showing comparison plots: \emph{all axes must have the same minima and maxima} 
% 	\end{enumerate}
% \end{frame}

\begin{frame}\frametitle{Aesthetics and style}
	
	I highly recommend reading Tufte's 4 books: contain remarkable examples of how to bring data to life.
\end{frame}

\begin{frame}\frametitle{Colour}
	\begin{itemize}
		\item	Colour is effective, but: 
		\begin{itemize}
			\item	readers could be colour-blind, 
			\item	document read from a gray-scale print out 
		\end{itemize}
	\end{itemize}
	\begin{itemize}
		\item	There is \textbf{no standard colour progression} (blues, greens, yellows, orange, red). 
		\item	Safest colour progression is gray-scale axis: from black to white 
		\begin{itemize}
			\item	satisfies colour-blind readers 
			\item	looks good in printed form 
		\end{itemize}
	\end{itemize}
\end{frame}

\begin{frame}\frametitle{General summary}
	
	\textbf{No general advice that applies in every instance}. Useful tips nevertheless:
	\begin{itemize}
		\item	To understand causality, you must show causality: use bivariate scatter plots (sometimes line plots also work well) 
		\item	Plots and text go together: a plot = paragraph of text 
		\begin{itemize}
			\item	add labels to plots for outliers and interesting points 
			\item	add equations 
			\item	add small summary tables 
		\end{itemize}
		\item	Avoid codes: ``A = grade TK133'', ``B = grade RT231'' 
	\end{itemize}
\end{frame}

\begin{frame}\frametitle{General summary}
	\begin{itemize}
		\item	Avoid unnecessary ``extras'' to enliven the plot 
		\item	``\emph{If the statistics are boring, then you've got the wrong numbers}''. 
	\end{itemize}
	\begin{center}
		\includegraphics[height=0.60\textheight]{\imagedir/visualization/Toronto-Star-Mortgage-Rates.png}
	\end{center}
\end{frame}

\begin{frame}\frametitle{General summary}
	\begin{itemize}
		\item	Adjust for inflation if plot involves money and time 
		\item	Maximize the data-ink ratio = (ink for data) / (total ink for graphics). 
		\begin{enumerate}
			\item	eliminate non-data ink 
			\item	erase redundant data-ink. 
		\end{enumerate}
		\item	Maximize data density: 250 data points per linear inch, and 625 data points per square inch. 
	\end{itemize}
\end{frame}

\begin{frame}\frametitle{General summary}
	Good plotting is not difficult. It just takes time and thought.
\end{frame}