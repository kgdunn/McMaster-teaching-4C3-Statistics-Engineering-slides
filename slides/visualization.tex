\begin{frame}\frametitle{Plot your data}
	\begin{center}
		\includegraphics[width=\textwidth]{\imagedir/visualization/visualization-subject-mapping.png}
	\end{center}
\end{frame}

\begin{frame}\frametitle{Usage examples}
	\begin{itemize}
		\item	\emph{Co-worker}: Here are the yields from a batch system for the last 3 years (1256 data points), can you help me: 
		\begin{itemize}
			\item	understand more about the time-trends in the past 3 year? 
			\item	efficiently summarize the yield from all batches run in 2010? \pause 
		\end{itemize}
		\item	\emph{Manager}: effectively summarize the (a) number and (b) types of defects on 17 aluminum grades for the past 12 months \pause 
		\item	\emph{Yourself}: 24 different measurements vs time (5 readings per minute, over 300 minutes) for each batch we produce; how can we visualize these 36,000 data points? 
	\end{itemize}
\end{frame}

\begin{frame}\frametitle{References}
	\begin{enumerate}
		\item	Edward Tufte, \emph{Envisioning Information}, Graphics Press, 1990. (10th printing in 2005) 
		\item	Edward Tufte, \emph{The Visual Display of Quantitative Information}, Graphics Press, 2001. 
		\item	Edward Tufte, \emph{Visual Explanations: Images and Quantities, Evidence and Narrative}, 2nd edition, Graphics Press, 1997. 
		\item	William Cleveland, \emph{Visualizing Data}, and \emph{The Elements of Graphing Data}, Hobart Press; 2nd edition, 1994. 
		\item	Stephen Few, \emph{Show Me the Numbers}, and ``Now You See It'', Analytics Press. 
		\item	Su, It's easy to produce chartjunk using Microsoft Excel 2007 but hard to make good graphs, \emph{Computational Statistics and Data Analysis}, \textbf{52} (10), 4594-4601, 2008, \href{http://dx.doi.org/10.1016/j.csda.2008.03.007}{http://dx.doi.org/10.1016/j.csda.2008.03.007} 
	\end{enumerate}
\end{frame}

\begin{frame}\frametitle{Background}
	
	This class might seem too easy, too obvious. It is!
	\begin{itemize}
		\item	The human eye and brain are excellent at pattern recognition, sorting through signal and noise. \pause 
		\item	We can easily cope with bad plots; but good plots save time and show a clearer, more honest picture. 
		\item	Cliches: ``Let the data speak for themselves'', ``Plot the data'' 
		\item	We will look at: \textbf{how} 
	\end{itemize}
\end{frame}

\begin{frame}\frametitle{Time-series plots}
	\begin{itemize}
		\item	It is a 2-dimensional plot: 
		\begin{itemize}
			\item	(usually) horizontal x-axis: time or sequence order 
			\item	other axis: the data values 
		\end{itemize}
		\item	Univariate plot 
	\end{itemize}
	\begin{itemize}
		\item	Our eyes can deal with high data density: 
		\begin{itemize}
			\item	sinusoids 
			\item	spikes 
			\item	outliers 
			\item	separate noise from signal 
		\end{itemize}
	\end{itemize}
\end{frame}

\begin{frame}\frametitle{Time-series plots}
	
	Good, automated labelling is important.
	
	Here's an example of bad labelling
	\begin{center}
		\includegraphics[width=\textwidth]{\imagedir/visualization/CPU-temperature_-_from_www_aw_org_on_26_Dec_2009.png}
	\end{center}
	(and bad axis scaling and colour choices)
\end{frame}

\begin{frame}\frametitle{Time-series plots}
	\begin{itemize}
		\item	Multiple lines (trajectories): should not cross and jumble 
	\end{itemize}
	\begin{center}
		\includegraphics[width=\textwidth]{\imagedir/visualization/3_correlated_variables_-_badly_displayed_in_Numbers.png}
	\end{center}
	\begin{itemize}
		\item	Colours and markers help only slightly 
	\end{itemize}
	\begin{center}
		\includegraphics[width=\textwidth]{\imagedir/visualization/3_correlated_variables_-_slightly_better.png}
	\end{center}
\end{frame}

\begin{frame}\frametitle{Time-series plots}
	
	Use separate, parallel axes rather; and minimal ink 
	\begin{center}
		\includegraphics[width=\textwidth]{\imagedir/visualization/3_correlated_variables_-_better.png}
	\end{center}
	These non-default settings can take a long time to set (10 minutes for this example)
\end{frame}

\begin{frame}\frametitle{Time-series plots}
	
	\textbf{Sparklines}
	\begin{itemize}
		\item	Read the website link (in the notes) 
	\end{itemize}
	\begin{center}
		\includegraphics[width=1in]{\imagedir/visualization/3-correlated-variables-as-sparklines.png}
	\end{center}
	\begin{itemize}
		\item	Used for financial trends (example) 
		\item	Built into Excel 2010 
		\item	Good for iPods, cell phones, tablet computers: 
		\begin{itemize}
			\item	high density, small size. 
		\end{itemize}
	\end{itemize}
\end{frame}

\begin{frame}\frametitle{Time-series plots}
	
	Example of sparklines in everyday use:
	\begin{center}
		\includegraphics[width=\textwidth]{\imagedir/classification/12leadECG.jpg}
	\end{center}
	Figure from \href{http://en.wikipedia.org/wiki/File:12leadECG.jpg}{Wikipedia}
\end{frame}

\begin{frame}\frametitle{Time-series plots}
	
	\textbf{Further tips}
	\begin{itemize}
		\item	Keep the x-axis spacing constant: helps interpretation 
		\begin{itemize}
			\item	don't reposition the time-axis labels 
			\item	don't use magnifying glass concept. 
		\end{itemize}
	\end{itemize}
	\begin{itemize}
		\item	Adjust for inflation when plotting money values against time 
		\begin{itemize}
			\item	sales of polymer to DuPont over the past 10 years 
			\item	example of car sales: \href{http://www.duke.edu/~rnau/411infla.htm}{http://www.duke.edu/~rnau/411infla.htm} 
		\end{itemize}
	\end{itemize}
\end{frame}

\begin{frame}\frametitle{Time-series plots}
	
	Show reasonable amount of data for context 
	\begin{center}
		\includegraphics[width=\textwidth]{\imagedir/visualization/AAPL-stock-prices.png}
	\end{center}
\end{frame}

\begin{frame}\frametitle{Bar plots}
	\begin{itemize}
		\item	A univariate plot on a two dimensional axis. 
		\item	Has a \emph{category axis} and \emph{value axis} 
	\end{itemize}
	\begin{center}
		\includegraphics[width=0.7\textwidth]{\imagedir/visualization/barplot-example-expenses.png}
	\end{center}
	Use a bar plot when: 
	\begin{itemize}
		\item	many categories 
		\item	interpretation does not change if category axis is reordered 
	\end{itemize}
\end{frame}

\begin{frame}\frametitle{Bar plots}
	
	Rather use a time-series plot if the data have a sequence: 
	\begin{center}
		\includegraphics[width=\textwidth]{\imagedir/visualization/quarterly-profit-barplot-vs-lineplot.png}
	\end{center}
	You can see the trends more clearly.
\end{frame}

\begin{frame}\frametitle{Bar plots}
	
	Bar plots can be wasteful as each data point is repeated several times:
	\begin{columns}
		\column{6cm} 
			\begin{center}
				\includegraphics[width=1.2\textwidth]{\imagedir/visualization/quarterly-profit-barplot-only.png} 
			\end{center}
		\column{4cm} 
			\begin{enumerate}
				\item	left edge (line) of each bar 
				\item	right edge (line) of each bar 
				\item	the height of the colour in the bar 
				\item	the number's position (up and down along the y-axis) 
				\item	the top edge of each bar, just below the number 
				\item	the number itself 
			\end{enumerate}
	\end{columns}
\end{frame}

\begin{frame}\frametitle{Bar plots}
	\begin{exampleblock}{Maximize data ink ratio within reason}
		\begin{center}
			\[
				\begin{array}{rl}
					\text{Maximize data ink ratio} 	&= \dfrac{\text{total ink for data}}{{\text{total ink for graphics}}} \\
													&= 1 - \text{proportion of ink that can be erased} \\
													&\qquad\qquad \text{without loss of data information} 
			\end{array}
			\]
		\end{center}
	\end{exampleblock}
	\begin{itemize}
		\item	Rather \textcolor{red}{use a table} for a handful of data points: 
	\end{itemize}
	\begin{center}
		\includegraphics[width=\textwidth]{\imagedir/visualization/profit-by-region.png}
	\end{center}
\end{frame}

\begin{frame}\frametitle{Bar plots}
	\begin{itemize}
		\item	Don't use cross-hatching, textures, or unusual shading in the plots: it creates visual vibrations 
	\end{itemize}
	\begin{center}
		\includegraphics[height=0.75\textheight]{\imagedir/visualization/hatched-barplot.png}
	\end{center}
\end{frame}

\begin{frame}\frametitle{Bar plots}
	\begin{itemize}
		\item	Use horizontal bars if: 
		\begin{itemize}
			\item	there is a some ordering to the categories 
			\item	the labels do not fit side-by-side\pause 
		\end{itemize}
	\end{itemize}
	\begin{itemize}
		\item	You can place the labels inside the bars\pause 
	\end{itemize}
	\begin{itemize}
		\item	You should usually start the non-category axis at zero 
	\end{itemize}
\end{frame}

\begin{frame}\frametitle{Box plots}
	
	A graphical display of the ``5-number summary'' for 1 variable 
	\begin{itemize}
		\item	minimum sample value 
		\item	25th percentile (1st quartile) 
		\item	50th percentile (median) 
		\item	75th percentile (3rd quartile) 
		\item	maximum sample value 
	\end{itemize}
	
	\emph{Notes}: 
	\begin{enumerate}
		\item	25th percentile is the value below which 25 percent of the observations in the sample are found 
		\item	distance from 3rd to 1st quartile = interquartile range (IQR) 
	\end{enumerate}
	
	Box plots are effective for comparing similar variables (same units of measurement)
\end{frame}

\begin{frame}\frametitle{Box plots}
	\begin{itemize}
		\item	Video of data source 
	\end{itemize}
\end{frame}

\begin{frame}\frametitle{Box plots}
	
\end{frame}

\begin{frame}\frametitle{Box plots}
	\begin{center}
		\includegraphics[width=0.9\textwidth]{\imagedir/visualization/boxplot-for-two-by-six-100-boards.png}
	\end{center}
\end{frame}

\begin{frame}\frametitle{Box plots}
	
	Some variations:
	\begin{itemize}
		\item	use the mean instead of the median 
		\item	outliers shown as dots, where an outlier is most commonly defined as any point 1.5 IQR distance units above and below the median. 
		\item	use the 2nd percentile (instead of $\text{median} - 1.5\text{IQR}$) 
		\item	use the 98th percentile (instead of $\text{median} + 1.5\text{IQR}$) 
		\item	add the density histogram onto the box plot: \emph{violin plot} 
	\end{itemize}
\end{frame}

\begin{frame}\frametitle{Box plot variation: violin plot}
	\begin{center}
		\includegraphics[width=0.8\textwidth]{\imagedir/univariate/violin-plot.png}
	\end{center}
\end{frame}

\begin{frame}\frametitle{Scatter plots}
	\begin{itemize}
		\item	Used to help understand the relationship between two variables: a bivariate plot 
		\item	Collection of points in the 2 axes 
		\item	Each point is the intersection of the values on each axis 
	\end{itemize}
	\begin{block}
		{Intention of a scatter plot} 
		\begin{center}
			Asks the viewer to draw a causal relationship between the two variables 
		\end{center}
	\end{block}
\end{frame}

\begin{frame}\frametitle{Scatter plots}
	\begin{center}
		\includegraphics[height=0.85\textheight]{\imagedir/visualization/scatterplot-temperature-vs-vapour-pressure.png}
	\end{center}
\end{frame}

\begin{frame}\frametitle{Scatter plots}
	
	However, not all scatter plots show causal phenomenon. 
	\begin{center}
		\includegraphics[height=0.80\textheight]{\imagedir/visualization/scatterplot-white-hairs-vs-BMD.png}
	\end{center}
\end{frame}

\begin{frame}\frametitle{Scatter plots}
	
	Strive for graphical excellence by:
	\begin{itemize}
		\item	making each axis as tight as possible 
		\item	avoid heavy grid lines 
		\item	use the least amount of ink 
		\item	do not distort the axes 
	\end{itemize}
\end{frame}

\begin{frame}\frametitle{Scatter plots}
	
	There is an unfounded fear that others won't understand your 2D scatter plot. 
	\begin{itemize}
		\item	Tufte study (VDQI): no scatter plots in a sample (1974 to 1980) of Western dailies 
		\item	12 year olds can interpret such plots. 
		\item	Japanese newspapers frequently use scatterplots 
		\item	Plant control room: seldom see scatter plots. 
	\end{itemize}
	\begin{block}
		{Key point} 
		\begin{center}
			The producers of charts must assume their audience is capable of interpreting them. Rather, assume that if you can understand the plot, so will your audience. 
		\end{center}
	\end{block}
\end{frame}

\begin{frame}\frametitle{Scatter plots}
	\begin{itemize}
		\item	Add box plots or histograms to aide interpretation: 
	\end{itemize}
	\begin{center}
		\includegraphics[height=0.80\textheight]{\imagedir/visualization/scatterplot-with-histograms.png}
	\end{center}
\end{frame}

\begin{frame}\frametitle{Scatter plots}
	\begin{itemize}
		\item	Add a 3rd variable: different marker sizes 
		\item	Add a 4th variable: use colour or grayscale shading 
	\end{itemize}
	\begin{center}
		\includegraphics[height=0.60\textheight]{\imagedir/visualization/scatterplot-with-2-extra-dimensions.png}
	\end{center}
	\begin{itemize}
		\item	The GapMinder website allows you to ``play''the graph over time (the 5th variable) 
	\end{itemize}
\end{frame}

\begin{frame}\frametitle{Scatter plots}
	\begin{itemize}
		\item	Web-based demo from \href{http://gapminder.org}{http://gapminder.org} 
		\item	\href{http://www.youtube.com/v/jbkSRLYSojo}{Demo by Hans Rosling} (requires internet access) 
	\end{itemize}
\end{frame}

\begin{frame}\frametitle{Tables}
	
	Tables are for \textbf{\emph{comparative}} data analysis on \textbf{\emph{categorical objects}}.
	\begin{center}
		\includegraphics[width=\textwidth]{\imagedir/visualization/table-car-payments.png}
	\end{center}
	\begin{itemize}
		\item	Note the rows are in \emph{default} alphabetical order. 
		\item	We can make the table ``tell a story'' if we reorder the rows by some other variable. 
		\begin{itemize}
			\item	e.g. monthly insurance payment 
		\end{itemize}
	\end{itemize}
\end{frame}

\begin{frame}\frametitle{Tables}
	\begin{itemize}
		\item	Compare defect types (number of defects) for different product grades (categories): 
	\end{itemize}
	\begin{center}
		\includegraphics[width=\textwidth]{\imagedir/visualization/table-defect-counts.png}
	\end{center}
	\begin{itemize}
		\item	Which defects cost us the most money? 
	\end{itemize}
\end{frame}

\begin{frame}\frametitle{Tables}
	\begin{itemize}
		\item	Defect frequency 
		\begin{itemize}
			\item	If 1850 lots of grade A4636 (first row): defect A rate = 1/50 
			\item	If 250 lots of grade A2610 (last row): defect A rate = 1/50 
			\item	Redraw table on production rate basis 
		\end{itemize}
		\item	If comparing defects over different grades: go down the table (show fraction within the column) 
		\item	If comparing defects within grade: go across table (show fraction with the row) 
		\begin{itemize}
			\item	Could weight each column by cost of defect 
		\end{itemize}
	\end{itemize}
\end{frame}

\begin{frame}\frametitle{Tables}
	Three common pitfalls:
	
	1. using pie charts when tables will do 
	\begin{center}
		\includegraphics[width=\textwidth]{\imagedir/visualization/where-MACC-alumni-work.png}
	\end{center}
\end{frame}

\begin{frame}\frametitle{Pie charts}
	\todo{SDL project examples}
	Other bad examples:
	* Gratitous-pie-chart-www.theglobeandmail.com-with-this-budget-ontario-is-banking-on-the-near-impossible-article1510634
	* SDL project reports
\end{frame}

\begin{frame}\frametitle{Tables}
	2. arbitrarily ordering of the rows 
	\begin{center}
		\includegraphics[width=\textwidth]{\imagedir/visualization/table-car-payments.png}
	\end{center}
\end{frame}

\begin{frame}\frametitle{Tables}
	3. using excessive grid lines 
	\begin{center}
		\includegraphics[width=\textwidth]{\imagedir/visualization/table-grid-comparison.png}
	\end{center}
\end{frame}

\begin{frame}\frametitle{Tables}
	Interesting example: comparing two treatments
	\begin{center}
		\includegraphics[width=\textwidth]{\imagedir/visualization/two-treatments-table.png}
	\end{center}
\end{frame}

\begin{frame}\frametitle{Tables}
	\begin{center}
		\includegraphics[width=\textwidth]{\imagedir/visualization/tables-recast-as-plots-both.png}
	\end{center}	
\end{frame}

\begin{frame}[allowframebreaks]\frametitle{Data frames}
	
	Frames are the basic containers that surround the data and give context to our numbers. Here are some tips:
	\begin{enumerate}
		\item	Use round numbers 
		\item	Tighten the axes as much as possible, except ... 
		\item	when showing comparison plots: \emph{all axes must have the same minima and maxima} 
	\end{enumerate}
\end{frame}

\begin{frame}[allowframebreaks]\frametitle{Aesthetics and style}
	
	I highly recommend reading Tufte's 4 books: contain remarkable examples of how to bring data to life.
\end{frame}

\begin{frame}[allowframebreaks]\frametitle{Colour}
	\begin{itemize}
		\item	Colour is effective, but: 
		\begin{itemize}
			\item	readers could be colour-blind, 
			\item	document read from a gray-scale print out 
		\end{itemize}
	\end{itemize}
	\begin{itemize}
		\item	There is \textbf{no standard colour progression} (blues, greens, yellows, orange, red). 
		\item	Safest colour progression is gray-scale axis: from black to white 
		\begin{itemize}
			\item	satisfies colour-blind readers 
			\item	looks good in printed form 
		\end{itemize}
	\end{itemize}
\end{frame}

\begin{frame}\frametitle{General summary}
	
	\textbf{No general advice that applies in every instance}. Useful tips nevertheless:
	\begin{itemize}
		\item	To understand causality, you must show causality: use bivariate scatter plots (sometimes line plots also work well) 
		\item	Plots and text go together: a plot = paragraph of text 
		\begin{itemize}
			\item	add labels to plots for outliers and interesting points 
			\item	add equations 
			\item	add small summary tables 
		\end{itemize}
		\item	Avoid codes: ``A = grade TK133'', ``B = grade RT231'' 
	\end{itemize}
\end{frame}

\begin{frame}\frametitle{General summary}
	\begin{itemize}
		\item	Avoid unnecessary ``extras'' to enliven the plot 
		\item	``\emph{If the statistics are boring, then you've got the wrong numbers}''. 
	\end{itemize}
	\begin{center}
		\includegraphics[height=0.60\textheight]{\imagedir/visualization/Toronto-Star-Mortgage-Rates.png}
	\end{center}
\end{frame}

\begin{frame}\frametitle{General summary}
	\begin{itemize}
		\item	Adjust for inflation if plot involves money and time 
		\item	Maximize the data-ink ratio = (ink for data) / (total ink for graphics). 
		\begin{enumerate}
			\item	eliminate non-data ink 
			\item	erase redundant data-ink. 
		\end{enumerate}
		\item	Maximize data density: 250 data points per linear inch, and 625 data points per square inch. 
	\end{itemize}
\end{frame}

\end{document}
