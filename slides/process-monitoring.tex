\iftoggle{hrymak}{}{\begin{frame}\frametitle{}
	\begin{center}
		\includegraphics[width=.7\textwidth]{\imagedir/monitoring/SPC-position-23-Jan-2013.png}
	\end{center}
\end{frame}}

\begin{frame}\frametitle{In context}

	Sections covered so far:
	\begin{enumerate}
		\item	Visualizing data
		\item	Univariate statistics
		\item	Combine those two areas: create a system to visually monitor any process
	\end{enumerate}

	\textbf{AIM}: rapid problem detection
	\begin{itemize}
		\item	then comes diagnosis (cover this later)
		\item	and process adjustment (not covered)
	\end{itemize}
\end{frame}

\begin{frame}\frametitle{Examples}

	Systems you may have seen:
	\begin{itemize}
		\item	hospital (monitoring patients)
		\item	stock market charts (intraday trading)
		\item	processing/manufacturing facility
	\end{itemize}

	Examples:
	\begin{itemize}
		\item	\emph{Co-worker}: are our product dimensions (or some other production quality measurement) stable?
		\item	\emph{Yourself}: how can we quickly detect a slow drift in the process?
		\item	\emph{Manager}: track the hourly average profit, and process throughput and react to any problems.
		\item	\emph{Potential customer}: what is your process capability? We are looking for Cpk of at least 1.6.
	\end{itemize}

	Note: process monitoring is mostly \textbf{reactive} and not \emph{proactive}. So it is suited to \emph{incremental} process improvement.
\end{frame}

\begin{frame}\frametitle{What we will cover}

	\includegraphics[width=\textwidth]{\imagedir/mindmaps/process-monitoring-section-mapping.png}

	\textbf{Concepts}: ARL, LCL, UCL, Cpk, Shewhart charts, EWMA, CUSUM, Type I and II errors, false alarms, real-time application
\end{frame}

\begin{frame}\frametitle{What is process monitoring about?}
	\begin{itemize}
		\item	We know that quality is not optional; customers move onto suppliers that provide quality products
		\item	Quality is not a cost-benefit trade-off either
		\begin{itemize}
			\item	Example: car sales in North America: the steady rise of the Asian manufacturers; starting to change ...
		\end{itemize}
	\end{itemize}
\end{frame}

\begin{frame}\frametitle{Process monitoring: relationship to feedback control}
	\begin{itemize}
		\item	Also called ``Statistical Process Control'' (SPC)
		\item	We will avoid this term due to potential confusion:
		\item	Monitoring is \emph{similar} to (feedback) control:
		\begin{itemize}
			\item	continually applied
			\item	check for deviations (error)
		\end{itemize}
		\item	Monitoring is \emph{different} to (feedback) control:
		\begin{itemize}
			\item	no action taken, unless required
			\item	i.e. adjustments are \textbf{infrequent}
			\item	usually \textbf{manual}
			\item	adjust due to \textbf{special causes}
		\end{itemize}
	\end{itemize}
	\begin{itemize}
		\item	Process monitoring: make \emph{permanent} adjustments to reduce variability
		\item	Feedback control: \emph{temporarily} compensates for the problem all the time
	\end{itemize}
\end{frame}

\begin{frame}\frametitle{Control charts}

	Used to display and detect this unusual variability
	\begin{itemize}
		\item	it is most often a time-series plot, or sequence
		\item	a target value may be shown
		\item	one or more limit lines are shown
		\item	displayed in real-time, or pretty close to real-time
	\end{itemize}

	\includegraphics[width=\textwidth]{\imagedir/monitoring/demo-of-monitoring-chart.png}
\end{frame}

\iftoggle{hrymak}{}{\begin{frame}\frametitle{Control charts: demonstration}
	\begin{center}
		\includegraphics[width=0.8\textwidth]{\imagedir/examples/flotation/Flotation_cell-Wikipedia.jpg}
	\end{center}
	
	\see{\href{http://en.wikipedia.org/wiki/File:Flotation_cell.jpg}{Wikipedia}}
\end{frame}}

\begin{frame}\frametitle{General approach}
	\begin{itemize}
		\item	\textbf{Phase 1}: building and testing from off-line data
		\begin{itemize}
			\item	very iterative: remove outliers
			\item	calculate limits, test if they are useful, repeat
			\item	you will spend most of your time here
		\end{itemize}
	\end{itemize}
	\begin{itemize}
		\item	\textbf{Phase 2}: using the control chart
		\begin{itemize}
			\item	on new, unseen data
			\item	implemented with computer hardware and software
			\item	usually for real-time display
		\end{itemize}
	\end{itemize}
\end{frame}

\begin{frame}\frametitle{Step 0: Decide what to monitor?}

	\emph{Discuss these in groups}: what would you monitor
	\begin{itemize}
		\item	Waste water treatment process
		\item	Tablet/pharmaceutical manufacturing
		\item	Oil and gas (e.g. a distillation column)
		\item	Food-processing unit (e.g. a fryer)
		\item	Mineral processing plant (e.g. a flotation cell)
		\item	Plastics processing (e.g. a twin-screw extruder)
	\end{itemize}
\end{frame}

\begin{frame}\frametitle{In-control vs out-of-control}
	\begin{exampleblock}{}
		\begin{center}
			``{\color{purple}{In-control}}'': behaviour of the process is stable over time. Also called \emph{common cause} operation.
		\end{center}
	\end{exampleblock}
	\vspace{24pt}
	\emph{The opposite is}: {\color{purple}{out of control}}, assignable causes, {\color{purple}{special causes}}, destabilizing event, off-target.
\end{frame}

\begin{frame}\frametitle{Shewhart chart}
	\begin{itemize}
		\item	Named for \emph{Walter Shewhart} from Bell Telephone and Western Electric, parts manufacturing, 1920's
		\item	A chart for monitoring variable's \emph{location}
	\end{itemize}

	\vspace{12pt}

	It has:
	\begin{itemize}
		\item	lower control limit (LCL)
		\item	upper control limit (UCL)
		\item	a target
	\end{itemize}

	\vspace{12pt}
	No action is required as long as the variable plotted remains within limits.

	\begin{center}
		\includegraphics[width=0.8\textwidth]{\imagedir/monitoring/demo-of-monitoring-chart.png}
	\end{center}
\end{frame}

\begin{frame}\frametitle{Shewhart chart: Derivation}
	\begin{center}
		\includegraphics[width=0.8\textwidth]{\imagedir/monitoring/explain-Shewhart-data-source.png}
	\end{center}
	
	\begin{itemize}
		\item	Take a subgroup of samples, of size $n$ {\small ($n=5$ in the picture here)}
		\item	Calculate $\bar{x}$ from the $n$ values
		\item	What is the distribution of $\bar{x}$?
		\begin{itemize}
			\item	$\bar{x} \sim \mathcal{N}(\mu, \sigma^2/n)$
		\end{itemize}
		\item	Define: $\sigma_{\bar{X}} = \sigma/\sqrt{n}$
	\end{itemize}
\end{frame}

\begin{frame}\frametitle{Shewhart chart: Derivation}
	\begin{itemize}
		\item	\textbf{Thin line}: single process measurements have $\mu = 6$ and $\sigma = 2$
		\item	\textbf{Thick line}: $\sigma_{\overline{x}} = 2/\sqrt{5}$ = 0.894, because $n=5$
		\begin{itemize}
			\item	Upper bound = $\mu + 3 \times \sigma_{\overline{x}} = 6 + 3\times 0.894 = 8.68$
			\item	Lower bound = $\mu - 3 \times \sigma_{\overline{x}} = 6 - 3\times 0.894 = 3.318$
		\end{itemize}
	\end{itemize}

	\begin{center}
		\includegraphics[width=0.9\textwidth]{\imagedir/monitoring/Explain-shewhart.png}
	\end{center}
\end{frame}

\begin{frame}\frametitle{Shewhart chart: Derivation}
	\begin{itemize}
		\item	$z$-value: $z = \dfrac{\bar{x} - \mu}{\sigma_{\bar{X}}}$
	\end{itemize}
	\begin{itemize}
		\item	Confidence interval for $\mu$: $\bar{x} - c_n\sigma_{\bar{X}} < \mu < \bar{x} + c_n\sigma_{\bar{X}}$
	\end{itemize}
	\begin{itemize}
		\item	By convention we use $c_n = 3.0$
		\item	Area between LCL and UCL: 99.73\%
		\item	A chance of 1 in 370 that a data point, $\bar{x}$, will lie outside these bounds
		\item	The bounds are for $\bar{x}$, not for individual, raw $x$ values
	\end{itemize}
\end{frame}

\begin{frame}\frametitle{Shewhart chart: using estimates}

	But, we don't know population parameters
	\begin{itemize}
		\item	For $\mu$: use $\overline{\overline{x}} = \dfrac{1}{K} \displaystyle \sum_{k=1}^{K}{ \overline{x}_k}$
		\item	For $\mu$: or just use known target value
		\item	For $\mu$: or use the median of a long sequence of data
		\item	For $\sigma$:
		\begin{itemize}
			\item	Define: $\overline{S} = \dfrac{1}{K} \displaystyle \sum_{k=1}^{K}{s_k}$
			\item	Define: $s_k$ = standard deviation of $n$ values
			\item	For $\sigma$: use $\dfrac{\overline{S}}{a_n}$, where $a_n$ is a correction factor
		\end{itemize}
	\end{itemize}

	\begin{center}
		\includegraphics[width=0.8\textwidth]{\imagedir/monitoring/table-for-an-values.png}
	\end{center}
	
	$$
	\begin{array}{rcccl}
		\text{LCL} = \overline{\overline{x}} - 3 \cdot \dfrac{\overline{S}}{a_n\sqrt{n}} && && \text{UCL} = \overline{\overline{x}} + 3 \cdot \dfrac{\overline{S}}{a_n\sqrt{n}}
	\end{array}
	$$
\end{frame}

\begin{frame}\frametitle{Example}

	We measure 5 colour values on each rubber bale:
	\begin{itemize}
		\item	e.g. bale 1 raw values: [231, 251, 235, 241, 227]
		\begin{itemize}
			\item	$\bar{x}_1$ = 237 is our new data point for the Shewhart chart
			\item	$s_1 = 9.38$
		\end{itemize}
		\item	e.g. bale 2 raw values: [252, 253, 247, 232, 244]
		\begin{itemize}
			\item	$\bar{x}_2$ = 245.6
			\item	$s_2 = 8.44$
		\end{itemize}
	\end{itemize}

	Data from 20 such $\bar{x}_k$ calculations:

	\texttt{[245, 239, 239, 241, 241, 241, 238, 238, 236, 248, }

	\texttt{ 233, 236, 246, 253, 227, 231, 237, 228, 239, 240]}

	Calculated for you: $\bar{\bar{x}} = 238.8$ and $\bar{S} = 9.28$

	\textbf{Phase 1} workflow:
	\begin{itemize}
		\item	Calculate: LCL and UCL
		\item	Any points outside limits?
		\item	If so, exclude and repeat.
	\end{itemize}
\end{frame}

\begin{frame}\frametitle{Example}
	\begin{itemize}
		\item	LCL = $238.8 - 3 \cdot \dfrac{9.28}{(0.94)(\sqrt{5})} = 225.6$
		\item	UCL = $238.8 + 3 \cdot \dfrac{9.28}{(0.94)(\sqrt{5})} = 252.0$
		\item	Sample with value of 253 exceeds these limits
		\item	After excluding it:
		\begin{itemize}
			\item	new $\bar{\bar{x}} = 238.0$, and new $\bar{S} = 9.68$
			\item	new LCL = 224, new UCL = 252
		\end{itemize}
	\end{itemize}
	\begin{center}
		\includegraphics[width=0.8\textwidth]{\imagedir/monitoring/Bale-image-colour.png}
	\end{center}
\end{frame}

\begin{frame}\frametitle{Assessing the chart's performance: error probability}
	\begin{itemize}
		\item	\textbf{Type I error}: $\bar{x}$ typical of normal operation, but falls outside UCL or LCL limits
		\begin{itemize}
			\item	Theoretical derivation: that happens 1 in 370 times (99.73\% inside limits)
			\item	Also called $\alpha = 0.0027$ when using $\pm 3 \sigma_{\bar{x}}$ limits $(0 <\alpha < 1)$
			\item	\emph{Synonyms}: false alarm, false positive (diseases), producer's risk (acceptance sampling), false reject rate
		\end{itemize}
	\end{itemize}
	\begin{itemize}
		\item	\textbf{Type II error}: $\bar{x}$ is not stable, but falls within UCL and LCL limits
		\begin{itemize}
			\iftoggle{hrymak}{}{\item	Called $\beta$: is a function of the degree of difference (next)}
			\item	\emph{Synonyms}: false negative, consumer's risk, false acceptance rate
		\end{itemize}
	\end{itemize}
	\begin{itemize}
		\item	Asymmetrical risks: airport screening, disease diagnosis, trial by jury, making drugs
	\end{itemize}
	\begin{itemize}
		\item	In pseudo-math:
	\end{itemize}
	$
	\begin{array}{rcl}
		\alpha &=& Pr\left(\overline{x}\,\,\text{is in control, but lies outside the limits}\right)\\
		\iftoggle{hrymak}{}{\beta &=& Pr\left(\overline{x}\,\,\text{is not in control, but lies inside the limits}\right)}
	\end{array}
	$
\end{frame}

\iftoggle{hrymak}{}{
\begin{frame}\frametitle{Assessing the chart's performance: type II error}

	Assume $\bar{x}$ from shifted distribution: $\mu$ to $\mu + \Delta\sigma$

	\textbf{Question}: probability this new $\bar{x}$ will fall within existing LCL and UCL?

	\includegraphics[width=\textwidth]{\imagedir/monitoring/show-shift-beta-error.png}
\end{frame}

\begin{frame}\frametitle{Assessing the chart's performance: type II error}
	\begin{center}
		\includegraphics[width=0.7\textwidth]{\imagedir/monitoring/type-II-error-shift.png}
	\end{center}
	
	\begin{itemize}
		\item	$\beta$ is a function of the process shift, $\Delta$
		\item	Table is for $n=4$ and UCL and LCL at the $\pm 3 \sigma_{\bar{X}}$ limits
		\item	\texttt{beta <- pnorm(3 - delta*sqrt(n)) - pnorm(-3 - delta*sqrt(n))}
		\item	Interpretation: Shewhart chart not good at detecting a change in the location!
		\item	Surprising, given that we use Shewhart chart for this purpose.
		\item	E.g: a $0.75\sigma$ only be detected around 6.7\% of the time
	\end{itemize}
\end{frame}
}

\begin{frame}\frametitle{Adjusting the chart's performance}
	\begin{block}
		{Key point}
		\begin{center}
			Control chart limits are not set in stone. Adjust them!
		\end{center}
	\end{block}
	Nothing makes a control chart more useless to operators than frequent false alarms.
	\begin{itemize}
		\item	$\alpha$: simply move LCL and UCL up and down (not incorrect to do this!)
		\iftoggle{hrymak}{}{\item	$\beta$: as you increase UCL, $\alpha \rightarrow 0$, but $\beta \rightarrow 1$}
		\item	But note that as you decrease type I error, your type II error will increase
		\item	Cannot simultaneously have low type I and type II error
	\end{itemize}
\end{frame}

\begin{frame}\frametitle{Average run length (ARL)}
	\begin{itemize}
		\item	ARL = average number of sequential samples we expect before seeing a point outside limits
		\item	ARL = $\dfrac{1}{\alpha}$
		\item	ARL for in control process, with 3-sigma limits?
		\begin{itemize}
			\item	$\alpha$ = 0.0027 (0.27\% of false alarms)
			\item	ARL = 1/0.0027 = 370
		\end{itemize}
	\end{itemize}
\end{frame}

\begin{frame}\frametitle{Extensions: Western Electric Rules}
	\begin{itemize}
		\item	Basic Shewhart chart is not too sensitive to process shifts. Raise an alarm when these \emph{improbable} events occur:
		\begin{itemize}
			\item	2 out of 3 points lie beyond $2\sigma$ on the same side of the centre line
			\item	4 out of 5 points lie beyond $1\sigma$ on the same side of the centre line
			\item	8 successive points lie on the same side of the center line
		\end{itemize}
	\end{itemize}

	The theoretical ARL is reduced by using these rules
	\begin{itemize}
		\item	\textbf{Adding robustness}: use robust methods in phase 1: see notes for a journal reference.
	\end{itemize}
	\begin{itemize}
		\item	\textbf{Warning limits}:
		\begin{itemize}
			\item	warning at $\pm 2 \sigma$ (orange coloured lines or background)
			\item	action at $\pm 3\sigma$ (red coloured lines or background)
		\end{itemize}
	\end{itemize}
\end{frame}

\begin{frame}\frametitle{Mistakes to avoid}
	\begin{enumerate}
		\item	Adding product specification to the monitoring chart:
		\begin{itemize}
			\item	don't use spec limits instead of LCL and UCL
			\item	we are monitoring for stability, not defects (out-of-spec product)
		\end{itemize}
		\item	Variables with heavy autocorrelation: you get much longer ARL when a fault occurs
		\begin{itemize}
			\item	use EWMA chart (next section)
		\end{itemize}
		\item	Shewhart charts on \textbf{highly correlated quality variables}
		\begin{itemize}
			\item	an important topic - introduced in latent variable section
			\item	also see next two slides for a quick explanation
		\end{itemize}
	\end{enumerate}
\end{frame}

\begin{frame}\frametitle{Notice the trends between these variables?}
	\begin{center}
		\includegraphics[width=\textwidth]{\imagedir/monitoring/two-axis-monitoring-separate-plots.jpg}
	\end{center}
\end{frame}

\begin{frame}\frametitle{Monitoring correlated variables: we need a better tool}
	\begin{center}
		\includegraphics[width=\textwidth]{\imagedir/monitoring/two-axis-monitoring-plot.png}
	\end{center}
\end{frame}

\begin{frame}\frametitle{CUSUM charts}
	\begin{itemize}
		\item	Shewhart chart takes a long time to detect shift in the mean, away from target, $T$
		\item	CUSUM formula:
	\end{itemize}
	$
	\begin{array}{rcl}
		S_0 &=& (x_0 - T) \\
		S_1 &=& (x_0 - T) + (x_1 - T) = S_0 + (x_1 - T) \\
		S_2 &=& (x_0 - T) + (x_1 - T) + (x_2 - T) = S_1 + (x_2 - T) \\
	\end{array}
	$
	\begin{block}
		{CUSUM formula:}
		\begin{center}
			$
			\begin{array}{rcl}
				S_t &=& S_{t-1} + (x_t - T)
			\end{array}
			$
		\end{center}
	\end{block}
	\begin{itemize}
		\item	For process shifts: we are adding $\Delta$ to every $x_t$
		\item	Accumulates: creates a steep up or down slope
	\end{itemize}
\end{frame}

\begin{frame}\frametitle{CUSUM charts: in-control}
	\begin{itemize}
		\item	$\mu=20$ and $\sigma=3$
	\end{itemize}
	\begin{center}
		\includegraphics[width=\textwidth]{\imagedir/monitoring/CUSUM-no-shift.png}
	\end{center}
\end{frame}

\begin{frame}\frametitle{CUSUM charts: out-of-control}
	\begin{itemize}
		\item	Shift of 1.2 units at $t=150$, caught at around $t=180$
	\end{itemize}
	\begin{center}
		\includegraphics[width=\textwidth]{\imagedir/monitoring/CUSUM-with-shift.png}
	\end{center}
\end{frame}

\begin{frame}\frametitle{Using a CUSUM chart}
	\begin{itemize}
		\item	Type I and II error set by the angle and distance of the V-mask
		\item	Implemented by computers
		\item	If a fault is detected, reset $S_t$ to a new value and restart chart
	\end{itemize}
\end{frame}

\begin{frame}\frametitle{EWMA charts: what is a moving average?}
	\begin{enumerate}
		\item	Shewhart: each subgroup is independent (unrelated), no ``memory''
		\item	CUSUM: infinite build up of errors, all the way back to t=0
		\item	Moving average chart: has a ``window'' of memory:
	\end{enumerate}
	
	\begin{center}
		\includegraphics[width=\textwidth]{\imagedir/monitoring/explain-moving-average-data-source.png}
	\end{center}
	
	$$
	\begin{array}{rcl}
		\\
		\bar{x}_t &=& \dfrac{1}{n}x_{t-1} + \dfrac{1}{n}x_{t-2} + \ldots + \dfrac{1}{n}x_{t-n}
	\end{array}
	$$
\end{frame}

\begin{frame}\frametitle{EWMA derivation}
	\begin{itemize}
		\item	Exponentially weighted MA:
		\begin{itemize}
			\item	heavier weights for recent observations
			\item	small weights further back
		\end{itemize}
	\end{itemize}

	$
	\begin{array}{rcl}
		x_t &=& \text{new data at time step } t\\
		\hat{x}_t &=& \hat{x}_{t-1} + \lambda e_{t-1} \\
		e_t &=& x_t - \hat{x}_t \\
		\\
		\hat{x}_{t+1} &=& \hat{x}_{t} + \lambda e_{t} \\
		\\
	\end{array}
	$

	To start if off:
	\begin{itemize}
		\item	$\hat{x}_0 = T$
		\item	$e_0 = 0$
		\item	$\hat{x}_1 = T$
	\end{itemize}
\end{frame}

\begin{frame}\frametitle{EWMA derivation}

	$$
	\begin{array}{rcl}
		x_t &=& \text{new data at time step } t\\
		\hat{x}_{t+1} &=& \hat{x}_{t} + \lambda e_{t} \\
		e_t &=& x_t - \hat{x}_t \\
		\\
		\hat{x}_{t+1} &=& \hat{x}_{t} + \lambda \left(x_t - \hat{x}_t\right) \\
		\hat{x}_{t+1} &=& \lambda x_t + \left(1-\lambda \right)\hat{x}_{t}
	\end{array}
	$$
	
	\vspace{12pt}
	\begin{itemize}
		\item	Shows that EWMA is a one-step ahead predictor for $\hat{x}_{t+1}$
		\begin{enumerate}
			\item	$x_t$: current point, weighted by $\lambda$
			\item	$\hat{x}_{t}$: historical data, weighted by $\left(1-\lambda \right)$
		\end{enumerate}
	\end{itemize}
	\begin{itemize}
		\item	$\hat{x}_{t+1} = \sum_{i=0}^{i=t}{w_i x_i} = w_0x_0 + w_1x_1 + w_2x_2 + \ldots $
	\end{itemize}
\end{frame}

\begin{frame}\frametitle{EWMA derivation}
	\begin{center}
		\includegraphics[height=0.9\textheight]{\imagedir/monitoring/explain-EWMA-derivation-part1.png}
	\end{center}
\end{frame}

\begin{frame}\frametitle{EWMA derivation}
	\begin{center}
		\includegraphics[height=0.9\textheight]{\imagedir/monitoring/explain-EWMA-derivation-part2.png}
	\end{center}
\end{frame}

\begin{frame}\frametitle{EWMA derivation}
	\begin{center}
		\includegraphics[height=0.8\textheight]{\imagedir/monitoring/explain-weights-for-process-monitoring.png}
	\end{center}
	\vspace{-16pt}
	\begin{itemize}
		\item	As $\lambda \rightarrow 0$: smoother chart, uses more history, less current data
		\item	As $\lambda \rightarrow 1$: chart uses more current data (Shewhart-like)
	\end{itemize}
\end{frame}

\begin{frame}\frametitle{EWMA limits}

	$$
	\begin{array}{rcl}
		\text{LCL} &=& \bar{\bar{x}} - 3 \cdot \sigma_{\text{Shewhart}}\sqrt{\dfrac{\lambda}{2-\lambda}} \\
		\text{UCL} &=& \bar{\bar{x}} + 3 \cdot \sigma_{\text{Shewhart}} \sqrt{\dfrac{\lambda}{2-\lambda}}
	\end{array}
	$$
	\begin{itemize}
		\item	$\sigma_{\text{Shewhart}}$: standard deviation from the Shewhart chart.
	\end{itemize}

	\vspace{12pt}
	\textbf{Nice implementation}: show both Shewhart and EWMA on the same chart
	\begin{itemize}
		\item	Get the usual Shewhart monitoring, but
		\item	EWMA gives a one-step ahead prediction as well
		\item	Used for slow-moving processes with long gaps between samples
	\end{itemize}
\end{frame}

\begin{frame}\frametitle{EWMA example}

	\includegraphics[width=\textwidth]{\imagedir/monitoring/Hunter-EWMA-example.png}

	From: Hunter, ``The Exponentially Weighted Moving Average'', \emph{Journal of Quality Technology}, \textbf{18}, p 203-210, 1986.
\end{frame}

\begin{frame}\frametitle{Other charts}
	\begin{itemize}
		\item	The \emph{S chart}: monitor variance
		\item	The \emph{R chart}: precursor to the \emph{S chart} (not common anymore)
		\item	\emph{np chart} and \emph{p chart}: monitoring proportions of pass/fail or good/bad ratings
		\item	\emph{Exponentially weight moving variance} (EWMV): used for monitoring product variability
	\end{itemize}
\end{frame}

\begin{frame}\frametitle{What should we monitor?}

	Recall the aim is to \textbf{react early} to bad, or unusual operation:
	\begin{itemize}
		\item	implies monitoring variables in near real-time
		\item	laboratory measurements are good, but take longer to acquire
		\item	don't wait for bad production to be over, catch it early
	\end{itemize}
	\begin{block}
		{Key points}
		\begin{center}
			\begin{itemize}
				\item	Apply monitoring at every step in the manufacturing line/system
				\item	Obtain low variability early on; don't wait to the end
			\end{itemize}
		\end{center}
	\end{block}

	Problem isn't how to monitor, rather, \textcolor{red}{what do we monitor?}
\end{frame}

\begin{frame}\frametitle{What should we monitor?}

	Measurements from real-time systems are:
	\begin{itemize}
		\item	available more frequently (less delay) than lab measurements
		\item	often are more precise
		\item	more meaningful to the operating staff
		\item	contains ``fingerprint'' of problem (helps for diagnosis)
	\end{itemize}

	Variables don't need to be from on-line sensors: could also be a calculation

	\emph{Lab measurements} have long time delay:
	\begin{itemize}
		\item	process already shifted by the time lab values detect a problem
		\item	harder to find cause-and-effect for diagnosis
	\end{itemize}
\end{frame}

\begin{frame}\frametitle{Monitoring in today's context}

	We don't measure a single number in many cases:

	\includegraphics[width=\textwidth]{\imagedir/monitoring/pharma-spectra.png}

	460 spectra (lines) measured at 650 evenly spaced wavelengths ($x$-axis). The $y$-axis is the absorbance at each wavelength.

	How do we monitor this?
\end{frame}

\begin{frame}\frametitle{Monitoring in today's context}

	Image data: very common situation now
	\begin{itemize}
		\item	many easy measurements for liquids and gases;
		\item	but for solids: use image data
		\item	medical imaging
	\end{itemize}
	\begin{center}
		\includegraphics[width=0.4\textwidth]{\imagedir/data-types/image-data.png}
	\end{center}
	\begin{itemize}
		\item	Very high redundancy: neighbouring pixels are similar (spatially and in time)
	\end{itemize}
\end{frame}

\begin{frame}\frametitle{Monitoring in industrial practice}
	\begin{itemize}
		\item	Widely used in industry, at all levels
		\item	Management: monitor plants, geographic region, countries (e.g. hourly sales by region)
		\begin{itemize}
			\item	Dashboards, ERP, BI, KPI
		\end{itemize}
	\end{itemize}
	\begin{itemize}
		\item	Challenges for you:
		\begin{itemize}
			\item	Getting the data out
			\item	Real-time use of the data (value of data decays exponentially)
			\item	Training is time consuming
			\item	Bandwidth/network/storage
		\end{itemize}
	\end{itemize}
\end{frame}

\begin{frame}[allowframebreaks]\frametitle{General workflow}
	\begin{enumerate}
		\item	Identify variable(s) to monitor.
		\item	Retrieve historical data (computer systems, or lab data, or paper records)
		\item	Import data and just plot it.
		\begin{itemize}
			\item	Any time trends, outliers, spikes, missing data gaps?
		\end{itemize}
		\item	Locate regions of stable, common-cause operation.
		\begin{itemize}
			\item	Remove spikes and outliers
			\item	Cleaned data is your phase 1 data
		\end{itemize}
		\item	Split phase 1 data into a 60\% and 40\% split.
		\begin{itemize}
			\item	The 60\% split is for calculating model limits
			\item	The 40\% is for testing later on.
		\end{itemize}
		\item	Keep outlier data as a separate testing set: to validate detection
		\item	Calculate control limits (UCL, LCL), using formula, using 60\% data chunk
		\item	Test your chart on \textbf{new, unused} data. \emph{How does my chart work?}
		\begin{itemize}
			\item	Quantify type I error on cleaned 40\% chunk
			\item	Quantify type II error on outlier data
		\end{itemize}
		\item	Adjust the limits
		\item	Repeat this step, as needed to achieve levels of error
		\item	Run chart on your desktop computer for a couple of days
		\begin{itemize}
			\item	Confirm unusual events with operators; would they have reacted to it? False alarm?
			\item	Refine your limits
		\end{itemize}
		\item	Not an expert system - will not diagnose problems:
		\begin{itemize}
			\item	use your engineering judgement; look at patterns; knowledge of other process events
		\end{itemize}
		\item	Demonstrate to your colleagues and manager
		\begin{itemize}
			\item	But go with dollar values.
		\end{itemize}
		\item	Installation and operator training will take time
		\item	Listen to your operators
		\begin{itemize}
			\item	make plots interactive - click on unusual point, it drills-down to give more context
		\end{itemize}
	\end{enumerate}
\end{frame}

\begin{frame}\frametitle{Industrial case study: Dofasco}
	\begin{itemize}
		\item	ArcelorMittal in Hamilton (formerly called Dofasco) has used multivariate process monitoring tools since 1990's
		\item	Over 100 applications used daily
		\item	Most well known is their casting monitoring application, Caster SOS (Stable Operation Supervisor)
		\item	It is a multivariate monitoring system
	\end{itemize}
\end{frame}

\begin{frame}\frametitle{Dofasco case study: slabs of steel}

	\includegraphics[width=\textwidth]{\imagedir/examples/Dofasco/Casting-plant-environment.png}

	\emph{All screenshots with permission of Dr. John MacGregor}
\end{frame}

\begin{frame}\frametitle{Dofasco case study: casting}
	\begin{columns}
		\column{5cm}
		\includegraphics[width=\textwidth]{\imagedir/examples/Dofasco/Casting-schematic.png} \column{5cm}
		\includegraphics[width=\textwidth]{\imagedir/examples/Dofasco/Casting-hot-slab.png}
	\end{columns}
\end{frame}

\begin{frame}\frametitle{Dofasco case study: breakout}

	\includegraphics[width=\textwidth]{\imagedir/examples/Dofasco/Casting-breakout.png}
\end{frame}

\begin{frame}\frametitle{Dofasco case study: monitoring for breakouts}

	\includegraphics[width=\textwidth]{\imagedir/examples/Dofasco/Dofasco-monitoring-chart.png}
\end{frame}

\begin{frame}\frametitle{Dofasco case study: monitoring for breakouts}
	\begin{center}
		\includegraphics[width=0.7\textwidth]{\imagedir/examples/Dofasco/Dofasco-monitoring-chart.png}
	\end{center}
	\begin{itemize}
		\item	Stability Index 1 and 2: one-sided monitoring chart
		\item	Warning limits and the action limits.
		\item	A two-sided chart in the middle
		\item	Plenty of other operator-relevant information
	\end{itemize}
\end{frame}

\begin{frame}\frametitle{Dofasco case study: an alarm}

	\includegraphics[width=\textwidth]{\imagedir/examples/Dofasco/Dofasco-monitoring-chart-with-alarm.png}
\end{frame}

\begin{frame}\frametitle{Dofasco case study: economics of monitoring}

	\includegraphics[width=\textwidth]{\imagedir/examples/Dofasco/Breakouts-dofasco-economics.png}
	\begin{itemize}
		\item	Implemented system in 1997; multiple upgrades since then
		\item	Economic savings: more than \$ 1 million/year
		\begin{itemize}
			\item	each breakout costs around \$200,000 to \$500,000
			\item	process shutdowns and/or equipment damage
			\item	more than justifies the costs and person-hours to implement the system
		\end{itemize}
	\end{itemize}
\end{frame}

\begin{frame}\frametitle{Process capability: centered process}

	Process capability ratio (PCR) can be calculated for any attribute.

	$
	\begin{array}{rcl}
		\\
		\text{PCR} &=& \dfrac{\text{Upper specification limit} - \text{Lower specification limit}}{6\sigma}
	\end{array}
	$
	\begin{itemize}
		\item	Use an estimate for $\sigma$
		\item	LSL is not LCL; and USL is not UCL from Shewhart chart
		\item	LSL and USL are set by customers, or internal criteria
	\end{itemize}

	Strong assumptions used for PCR:
	\begin{itemize}
		\item	assumes the attribute has normal distribution (check with \texttt{qqPlot})
		\item	assumes centered system between LSL and USL
		\item	assumes PCR calculated when process was stable
	\end{itemize}
\end{frame}

\begin{frame}\frametitle{PCR interpretation: process ``width''}

	Let mean=80, LSL=65, USL=95 and $\hat{\sigma} = 10$

	Implies: PCR = 0.5
	\includegraphics[width=\textwidth]{\imagedir/monitoring/explain-PCR-half.png}
	\begin{itemize}
		\item	z for LSL = (65 - 80)/10 = -1.5
		\item	z for USL = (95 - 80)/10 = 1.5
		\item	Shaded area probability = \texttt{pnorm(-1.5) + (1-pnorm(1.5))} = 13.4\%
	\end{itemize}
\end{frame}

\begin{frame}\frametitle{PCR interpretation: process ``width''}

	Let mean=80, LSL=65, USL=95 and $\hat{\sigma} = 10/4 = 2.5$

	Implies: PCR = 2.0
	\begin{center}
		\includegraphics[width=0.7\textwidth]{\imagedir/monitoring/explain-PCR-two.png}
	\end{center}
	\begin{itemize}
		\item	z for LSL = (65 - 80)/2.5 = -6
		\item	z for USL = (95 - 80)/2.5 = 6
		\item	Shaded area probability = about 0 (1.973175e-09 x 100)
		\item	Process width: $12 \sigma$
		\item	Why is it called process capability?
		\begin{itemize}
			\item	If you cannot make your variance (standard deviation) smaller, your process is not capable
		\end{itemize}
	\end{itemize}
\end{frame}

\begin{frame}\frametitle{PCR interpretation: uncentered process}
	\begin{itemize}
		\item	Processes not often centered between LSL and USL
	\end{itemize}

	$ \text{PCR}_\text{k} = \text{C}_\text{pk} = \min \left( \dfrac{\text{Upper specification limit} - \bar{\bar{x}}}{3\sigma}; \dfrac{\bar{\bar{x}} - \text{Lower specification limit}}{3\sigma} \right) $
	\begin{itemize}
		\item	$\bar{\bar{x}}$ from Shewhart chart
		\item	One-sided limit: taken on the worst side!
		\item	Cpk = 1.3: minimum requirement
		\item	Cpk = 1.7: requested for safety and other critical applications.
		\item	Cpk = 2.0: termed a 6-sigma process: it can move $6\sigma$ units left or right
	\end{itemize}

	\textbf{Note}: Cpk and Cp are only useful for a process which is stable
\end{frame}

\begin{frame}\frametitle{Example}

	A tank uses small air bubbles to keep solid particles in suspension. If too much air is blown into the tank, then excessive foaming and loss of valuable solid product occurs; if too little air is blown into the tank the particles sink and drop out of suspension.
	
	\begin{center}
		\includegraphics[height=0.5\textheight]{\imagedir/monitoring/tank-suspension.png}
	\end{center}

	Which monitoring chart would you use to ensure the airflow is always near target?
\end{frame}

\begin{frame}\frametitle{Example}

	Describe how a monitoring chart could be used to prevent over-control of a batch-to-batch process. (A batch-to-batch process is one where a batch of materials is processed, followed by another batch, and so on).
\end{frame}

\begin{frame}\frametitle{Example}

	\textbf{Final exam, 2010}

	The most recent estimate of the process capability ratio for a key quality variable was 1.30, and the average quality value was 64.0. Your process operates closer to the lower specification limit of 56.0. The upper specification limit is 93.0.

	$\quad$

	What are the two parameters of the system you could adjust, and by how much, to achieve a capability ratio of 1.67, required by recent safety regulations. Assume you can adjust these parameters independently.
\end{frame}

\begin{frame}\frametitle{Example}

	The following values are the particle size of the most recent 20 shipments from a supplier, taken from their certificates of analysis:

	\texttt{50.9, 52.9, 51.6, 50.8, 54.6, 52.9, 53.1 }

	\texttt{48.4, 51.6, 53.1, 53.8, 52.4, 53.1, 50.8}

	\texttt{54.6, 52.9, 50.0, 53.8, 54.6, 52.2}

	$\qquad$

	Calculate the supplier's capability, given their lower specification limit of $45 \mu m$ and their upper limit at $59 \mu m$.

	$\qquad$

	Clearly state all assumptions you make during the calculations (median = $52.9 \mu m$, sd = $1.64 \mu m$)
\end{frame}

\begin{frame}\frametitle{Example}

	Plastic sheets are manufactured on your blown film line. The Cp value is 1.7. You sell the plastic sheets to your customers with specification of 2 mm $\pm$ 0.4 mm.
	\begin{itemize}
		\item	List three important assumptions you must make to interpret the Cp value.
		\item	What is the theoretical process standard deviation, $\sigma$?
		\item	What would be the Shewhart chart limits for this system using subgroups of size $n=4$?
		\item	Illustrate your answer from part 2 and 3 of this question on a diagram of the normal distribution.
	\end{itemize}
\end{frame}

\begin{frame}\frametitle{Answer}
	\includegraphics[width=\textwidth]{\imagedir/monitoring/plastic-sheet-control-specification-limits.png}
\end{frame}

