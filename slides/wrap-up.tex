% \begin{frame}\frametitle{Univariate section}
% 	\begin{itemize}
% 		\item	Data ink
% 		\item	Clear, concise plots
% 	\end{itemize}
% \end{frame}	
% 
% \begin{frame}\frametitle{Univariate section}
% 	
% 	\begin{itemize}
% 		\item	Histograms
% 		\item	Normal distribution and q-q plot 
% 		\item	External vs internal methods to test for differences
% 	\end{itemize}
% \end{frame}
% 
% \begin{frame}\frametitle{Process monitoring}
% 	\begin{itemize}
% 		\item	Shewhart charts
% 		\item	CUSUM charts
% 		\item	EWMA charts
% 	\end{itemize}
% 	
% 	\includegraphics[width=\textwidth]{\imagedir/monitoring/demo-of-monitoring-chart.png}
% 	\begin{itemize}
% 		\item	Cpk: width of process operation 
% 	\end{itemize}
% \end{frame}
% 
% \begin{frame}\frametitle{Least squares}
% 	\begin{block}
% 		{Model} 
% 		\begin{center}
% 			$y = b_1x_1 + b_2x_2$ 
% 		\end{center}
% 	\end{block}
% 	
% 	\includegraphics[width=0.65\textwidth]{\imagedir/least-squares/least-squares-two-x-variables.png}
% \end{frame}
% 
% \begin{frame}\frametitle{Least squares}
% 	\begin{itemize}
% 		\item	Example of integer and continuous variables 
% 	\end{itemize}
% 	
% 	\includegraphics[width=0.95\textheight]{\imagedir/least-squares/least-squares-two-x-variables-one-integer.png}
% \end{frame}
% 
% \begin{frame}\frametitle{Design and analysis of experiments}
% 	\begin{itemize}
% 		\item	Doing it wrong: 
% 	\end{itemize}
% 	
% 	\includegraphics[width=\textwidth]{\imagedir/doe/COST-contours.png}
% \end{frame}
% 
% \begin{frame}\frametitle{Design and analysis of experiments}
% 	\begin{itemize}
% 		\item	Factorial model: is a local approximation 
% 	\end{itemize}
% 	
% 	\includegraphics[width=\textwidth]{\imagedir/doe/factorial-two-level-surface-example-cropped.png}
% \end{frame}
% 
% \begin{frame}\frametitle{Design and analysis of experiments}
% 	\begin{itemize}
% 		\item	DOE models are just least squares models: 
% 	\end{itemize}
% 	
% 	\includegraphics[width=0.85\textwidth]{\imagedir/doe/Two-factor-least-squares-interpretation.png}
% \end{frame}
% 
% \begin{frame}\frametitle{Design and analysis of experiments}
% 	\begin{itemize}
% 		\item	Use the trade-off table: \textbf{bring a copy of it to the exam} 
% 	\end{itemize}
% 	
% 	\includegraphics[height=0.85\textheight]{\imagedir/doe/DOE-trade-off-table.png}
% \end{frame}
% 
% \begin{frame}\frametitle{Design and analysis of experiments}
% 	\begin{columns}
% 		\column{6cm} 
% 			\includegraphics[height=0.95\textheight]{\imagedir/doe/RSM-base-case-with-CCD-contours.png}
% 		\column{4cm}
% 			Model is only valid \emph{in a local region} 
% 	\end{columns}
% \end{frame}

\begin{frame}\frametitle{Final exam}
	\begin{itemize}
		\item	Thursday, 25 April, 19:00, MDCL/1105
		\item	3 hours 
		\item	Covers all topics taught throughout the term
		\item	Counts 48\% of the course grade
	\end{itemize}
\end{frame}

\begin{frame}\frametitle{Final exam: how to do well}
	\begin{itemize}
		\item	This is a course that gives you a set of tools 
		\item	Think about the question's objective; choose the most appropriate tool 
		\item	Think about the results: what do they mean? Check if they are reasonable. 
	\end{itemize}
\end{frame}

\begin{frame}\frametitle{Final exam: some strategies}
	\begin{itemize}
		\item	Do the hardest and highest weighted questions first 
		\item	Short quick questions at the end 
		\item	\textbf{Don't repeat the question back to me} 
		\item	\textbf{Use bullet points}, where appropriate 
		\item	Answer length should be appropriate to number of marks 
	\end{itemize}
\end{frame}

\begin{frame}\frametitle{Final exam: more strategies}
	\begin{itemize}
		\item	Read the questions carefully: ``\textbf{[2]} Calculate and interpret..''
		\item	Treat the exam like a closed-book exam, referring to your notes when necessary 
		\item	Go through previous assignments 
		\item	Uncover the objective of the question
		\item	Do not just plug in values into equations 
		\item	Copies of previous exam questions are on the course site 
		\begin{itemize}
			\item	Questions prior to 2010 are not mine 
			\item	2010, 2011 and 2012 exam available on website 
			\item	Solutions to most questions are in PID textbook, or in the assignments
		\end{itemize}
	\end{itemize}
	\begin{itemize}
		\item	Bring a copy of DOE table to exam 
		\item	Bring a copy of statistical tables to exam 
	\end{itemize}
\end{frame}

\begin{frame}\frametitle{Final exam: content}
	
	\textbf{\emph{All sections}} appear in the exam:
	\begin{enumerate}
		\item	Visualizing data 
		\item	Univariate statistics 
		\item	Process monitoring 
		\item	Least squares modelling 
		\item	Design and analysis of experiments (DOE) 
		\item	Latent variable methods 
	\end{enumerate}
	\emph{usually in combined form}
\end{frame}

\begin{frame}\frametitle{Finally ...}
	\begin{itemize}
		\item	If you have any corrections to the notes (grammar, spelling, etc) - please don't hesitate to email me 
		\item	Any suggestions to improve the notes gratefully accepted 
		\item	You are the 4th group that I've taught
		\item	The course website will be permanently available 
	\end{itemize}
	\begin{block}{}
		\href{http://learnche.mcmaster.ca/4C3}{http://learnche.mcmaster.ca/4C3}
	\end{block}

	\begin{itemize}
		\item	Course notes will eventually become an online book 
	\end{itemize}
	\begin{block}
		{Process Improvement using Data} 
		\begin{center}
			\href{http://learnche.mcmaster.ca/pid}{http://learnche.mcmaster.ca/pid} 
		\end{center}
	\end{block}
\end{frame}

\begin{frame}\frametitle{Finally ...}
	\begin{itemize}
		\item	I'd like to hear about your successes with these methods 
		\item	Also any issues and problems in using these tools 
		\item	Feel free to email me
		\item	Stay in touch
		\begin{itemize}
			\item	\href{http://ca.linkedin.com/in/kgdunn}{http://ca.linkedin.com/in/kgdunn}
		\end{itemize}
		\item	Thanks for your interest, your questions, your projects, and updates to the notes/slides
	\end{itemize}
\end{frame}
