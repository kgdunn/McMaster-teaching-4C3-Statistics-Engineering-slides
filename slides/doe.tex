\begin{frame}\frametitle{Intro}
	\begin{exampleblock}{}
		{\Huge When was the last time you ran an experiment ?}
	\end{exampleblock}
\end{frame}

\begin{frame}\frametitle{In context}
	\begin{itemize}
		\item	The section we've just finished could be considered: ``{\color{myOrange}{Empirical modelling of systems using a least squares model}}''
		
		\vspace{12pt}
		\item	Experiments are important:
		\begin{itemize}
			\item	We learn more about our systems
			\item	We use the data to fit an empirical model
			\item	Main aim: use the model to optimize a process for \textbf{higher profit}
		\end{itemize}
		
		\vspace{12pt}
		\item	Happenstance (as-is) data 
		\begin{itemize}
			\item	cannot tell cause-and effect
			\item	most often is not in a DOE layout, but might still be valuable to learn from.
		\end{itemize}
	\end{itemize}
\end{frame}

\begin{frame}\frametitle{Topics covered in this section}
	\begin{center}
		\includegraphics[width=\textwidth]{\imagedir/mindmaps/DOE-section-mapping.png}
	\end{center}
\end{frame}

\begin{frame}\frametitle{References}
	\begin{itemize}
		\item	Box, Hunter and Hunter, \emph{Statistics for Experimenters}
		\begin{itemize}
			\item	chapters 10, 11, 12, 13, 15 in first edition
			\item	chapters 5 and 6 in second edition
		\end{itemize}
	\end{itemize}
\end{frame}

\begin{frame}\frametitle{Experiments with a single variable at two levels}
	\begin{itemize}
		\item	Simplest case:
		\begin{itemize}
			\item	catalyst A vs catalyst B
			\item	low RPM vs high RPM
			\item	\emph{etc}
		\end{itemize}
		\item	Measure $n_A$ value from setup A
		\item	Measure $n_B$ values from setup B
		\item	Hold all other variables constant (control disturbances)
	\end{itemize}
\end{frame}

\begin{frame}\frametitle{Recap of group-to-group differences}

	Recap:

	$
	\begin{array}{rcl}
		\\
		s_P^2 &= \dfrac{(n_A -1) s_A^2 + (n_B-1)s_B^2}{n_A - 1 + n_B - 1} \\
		& \\
		z &= \dfrac{(\bar{x}_B - \bar{x}_A) - (\mu_B - \mu_A)}{\sqrt{s_P^2 \left(\dfrac{1}{n_A} + \dfrac{1}{n_B}\right)}}\\
		\\
	\end{array}
	$

	$
	\begin{array}{rcccl}
		\\
		{\scriptstyle (\bar{x}_B - \bar{x}_A) - c_t} \times \sqrt{\scriptstyle s_P^2 \left(\frac{1}{n_A} + \frac{1}{n_B}\right)} &\leq& {\scriptstyle \mu_B - \mu_A} &\leq & {\scriptstyle (\bar{x}_B - \bar{x}_A) + c_t } \times \sqrt{\scriptstyle s_P^2 \left(\frac{1}{n_A} + \frac{1}{n_B}\right)}
	\end{array}
	$
	\begin{itemize}
		\item	Significant difference: does confidence interval span zero?
		\item	Practical difference?
		\begin{itemize}
			\item	width of confidence interval
			\item	where it lies relative to zero
		\end{itemize}
	\end{itemize}
\end{frame}

\begin{frame}\frametitle{Using linear least squares models}
	\begin{itemize}
		\item	Can achieve same result using least squares: $y_i = b_0 + g d_i$
		\item	$d_i = 0$ for A
		\item	$d_i=1$ for B
		\item	and $y_i$ is the response variable.
	\end{itemize}
	
	\vspace{12pt}
	{\color{myOrange}{See question in this week's assignment.}}
	
	\vspace{12pt}
	Two totally different methods; same result! Confirm it for yourself with any other dataset.
\end{frame}

\begin{frame}\frametitle{Importance of randomization}

	Why {\color{purple}{randomize}} experiments?
	\begin{itemize}
		\item	Prevent {\color{purple}{\textbf{\emph{unmeasured}}}}, and {\color{purple}{\textbf{\emph{uncontrollable}}}} disturbances affecting $y$
		\item	Guarantees independence in the data
		\item	We can then use $t$-distributions (which require independence)
	\end{itemize}
	\begin{itemize}
		\item	The example of Fisher: lady and the tea. Modern day example: Coke vs Pepsi.
		\item	Engineering example: A = TK104 and B = TK107
		\begin{itemize}
			\item	$n_A = 8$: [254, 440, 501, 368, 697, 476, 188, 525]
			\item	$n_B = 9$: [338, 470, 558, 426, 733, 539, 240, 628, 517 ]
		\end{itemize}
		\item	Null hypothesis: there is no difference
		\begin{itemize}
			\item	Implies these numbers could have come from either A or B
		\end{itemize}
	\end{itemize}
	{\color{myOrange}{Details of the analysis are given in the course textbook}}
\end{frame}

\begin{frame}\frametitle{Importance of randomization}
	\begin{itemize}
		\item	$n_A = 8$: [254, 440, 501, 368, 697, 476, 188, 525]
		\item	$n_B = 9$: [338, 470, 558, 426, 733, 539, 240, 628, 517 ]
	\end{itemize}
	\begin{itemize}
		\item	Randomly assign ``A'' to any $n_A$ of the values and ``B'' to any $n_B$ of the values
		\item	$\dfrac{(n_A + n_B)!}{n_A! n_B!}$ possible combinations = 24310
		\item	Combinations: number of unique ways to split 17 experiments into 2 groups of $n_A=8$ and $n_B=9$
		\begin{itemize}
			\item	1: AAAAAAAABBBBBBBBB
			\item	2: AAAAAAABABBBBBBBB
			\item	3: AAAAAAABBABBBBBBB
			\item	\emph{etc}
		\end{itemize}
		\item	For each arrangement we calculate: $\bar{y}_A - \bar{y}_B$
		\item	Plot a histogram of this \emph{difference of averages}
		\item	Probability that the actual experiment could have come from chance?
	\end{itemize}
	% Note that running the code for moderate size n's is prohibitive, even on modern computer hardware.
\end{frame}

\begin{frame}\frametitle{Importance of randomization}
	\begin{itemize}
		\item	Probability that the actual experiment could have come from chance?
	\end{itemize}
	\begin{center}
		\includegraphics[width=0.7\textwidth]{\imagedir/doe/single-experiment-randomization.png}
	\end{center}
	\begin{itemize}
		\item	79.6\% combinations have a lower value than actual difference
		\item	Using standard group-to-group difference:
		\begin{itemize}
			\item	$z = 0.8435$
			\item	$Pr(z<0.8435) = 79.3\%$ (DOF = $n_A + n_B - 2$)
		\end{itemize}
	\end{itemize}
	\begin{itemize}
		\item	\textbf{Result}: if we don't randomize, we cannot use $z$-values and confidence intervals - may be misleading.
	\end{itemize}
\end{frame}

\begin{frame}\frametitle{Importance of randomization}
	The previous derivation, used random combinations and made no statistical assumptions.
	
	\vspace{12pt}
	{\color{myOrange}{\emph{Why don't we use this approach instead?}}}
	
	\vspace{12pt}
	The original data set (still a small data set by today's standards) was $n_A = 20$ and $n_B = 23$. There are $\dfrac{(n_A + n_B)!}{n_A! n_B!} \approx 960,567\,\text{million}$ combinations, and it would take about 3 years on a regular computer to do the computation\\ (never mind storing the results.)
	
	\vspace{12pt}
	So as long as we randomize our experiments, we can be guaranteed to use the regular $t$-test.
\end{frame}

\begin{frame}\frametitle{Change one single variable at a time (COST)}
	\begin{center}
		\includegraphics[width=0.95\textwidth]{\imagedir/doe/COST-contours.png}
	\end{center}
	\begin{itemize}
		\item	Base case: $T$=346K, $S$ = 1.5g/L; yield = 63\%.
	\end{itemize}
\end{frame}

\begin{frame}\frametitle{Change one single variable at a time (COST)}
	\begin{itemize}
		\item	Trapped in a sub-optimal solution
		\item	In the previous example: we would have considered experiment 7 to be the optimum
			\begin{itemize}
				\item	experiment 3 is the optimum wrt ``Temperature''
				\item	then experiment 7 is the optimum wrt ``Substrate''
				\item	but, we're still away from the true optimum
			\end{itemize}
		\item	We have known for 80 years now: COST is \textbf{wrong way} to optimize a system
		\item	How to do it better?
	\end{itemize}
\end{frame}

\begin{frame}\frametitle{Why not use existing data?}
	\begin{itemize}
		\item	Existing data = historical data = happenstance data
		\item	This is data without any intentional perturbations
		\item	\textbf{Problem}: we see correlations, but we cannot tell if they are causal
	\end{itemize}
	\includegraphics[width=0.75\textwidth]{\imagedir/doe/yield-pressure-impurity-correlation.png}
\end{frame}

\begin{frame}\frametitle{Terminology: {\color{purple}{Factors}}}
	{\color{purple}{Factor}}: the thing that is being changed.
	
	\begin{itemize}
		\item	growing plants? 
			\begin{itemize}
				\item	water used = [50mL \emph{vs} 80 mL]
			\end{itemize}
		\item	maximizing sales in a store? 
			\begin{itemize}
				\item	height from floor = [3ft \emph{vs} 5ft]
			\end{itemize}
		\item	first date or ``date-night''?
			\begin{itemize}
				\item	action movie \emph{vs} chick flick
			\end{itemize}
		\item	growing plants?
			\begin{itemize}
				\item	fertilizer A \emph{vs} fertilizer B
			\end{itemize}
	\end{itemize}
\end{frame}

\begin{frame}\frametitle{Terminology: {\color{purple}{Response}}}
	{\color{purple}{Response}}: the outcome that is being measured
	\begin{itemize}
		\item	growing plants? 
			\begin{itemize}
				\item	e.g. height of plant after 10 days
				\item	other outcomes are possible
			\end{itemize}
		\item	maximizing sales in a store? 
			\begin{itemize}
				\item	total profit
			\end{itemize}
	\end{itemize}
	
	\vspace{12pt}
	A response variable:
	\begin{itemize}
		\item	is usually (in almost every case) a continuous variable
		\item	should be measured in the same manner for all experiments
		\item	should be reproducibly measurable
		\item	measure as many outcomes as you can to avoid repeating experiments later
	\end{itemize}	
\end{frame}

\begin{frame}\frametitle{Factorial designs: 2 levels for 2 or more factors}
	\begin{itemize}
		\item	Change multiple factors \emph{simultaneously}
		\item	{\color{purple}{Factor}}: is a variable that we can manipulate/adjust/set
		\item	Consider, for now, two levels in each \textbf{{\color{myOrange}{factor}}}. For example:
		\begin{itemize}
			
			\item	continuous: low and high {\color{myOrange}{pH}}
			\item	continuous: short reaction time and long {\color{myOrange}{reaction time}}
			\item	discrete: {\color{myOrange}{catalyst}} A and B
			\item	discrete: {\color{myOrange}{mixing system}} A and B
		\end{itemize}
	\end{itemize}
\end{frame}

\begin{frame}\frametitle{Factorial designs: by example}
	\begin{center}
		\includegraphics[width=0.9\textwidth]{\imagedir/doe/factorial-two-levels-two-variables-contour-plot.png}
	\end{center}
	\begin{itemize}
		\item	We will use this system for our example
		\item	$y$ values are measured with error of $\pm$ 0.8 \% and then rounded
	\end{itemize}
\end{frame}

\begin{frame}\frametitle{Factorial designs: by example}

	Bioreactor example: aim is to maximize the $y$ = conversion [\%]
	\begin{itemize}
		\item	$T$: Temperature: $T_\text{low}$ = 338K and $T_\text{high}$ = 354K
		\item	$S$: Substrate concn: $S_\text{low}$ = 1.25\,g/L and $S_\text{high}$ = 1.75\,g/L
		\item	How is the range chosen?
		\begin{itemize}
			\item	About 25\% of typical operating range if no other prior knowledge. We will consider other criteria later also.
		\end{itemize}
	\end{itemize}
	\begin{itemize}
		\item	Factors are: $T$ and $S$
		\item	Number of experiments (runs): $2^k$; $k$ = number of factors
		\item	{\color{purple}{Standard order}} vs {\color{purple}{Actual execution (run) order}}
	\end{itemize}
	\begin{center}
		\includegraphics[width=\textwidth]{\imagedir/doe/DOE-factorial-factors.png}
	\end{center}
\end{frame}

\begin{frame}\frametitle{Factorial designs: by example}
	\begin{itemize}
		\item	Run your experiments in random order, collect results:
	\end{itemize}
	\begin{center}
		\includegraphics[width=\textwidth]{\imagedir/doe/DOE-factorial-factors-with-result.png}
	\end{center}
	Notes:
	\begin{itemize}
		\item	we don't need to run an experiment at the {\color{purple}{baseline}} (it can be useful though)
		\item	{\color{purple}{baseline}} at $T=\frac{1}{2}\left(338+354\right)$ and $S = \frac{1}{2}\left(1.25+1.75\right)$, i.e. \\
		        {\color{purple}{baseline}} at (346K; 1.5g/L) = midpoint of the factorial
		\item	if we had replicate experiments, then use the average of the response variable
	\end{itemize}
\end{frame}

\begin{frame}\frametitle{Factorial designs: by example}
	\begin{itemize}
		\item	Visualize results: cube plot
	\end{itemize}
	
	\vspace{12pt}
	\begin{center}
		\includegraphics[height=0.7\textheight]{\imagedir/doe/factorial-two-levels-two-variables-no-analysis.png}
	\end{center}
\end{frame}

\begin{frame}\frametitle{Analysis: Main effects}
	\begin{itemize}
		\item	Main effect: difference from high to low level
	\end{itemize}
	\begin{center}
		\includegraphics[width=\textwidth]{\imagedir/doe/factorial-two-levels-two-variables-with-analysis.png}
	\end{center}
	\begin{itemize}
		\item	Where would you run your next experiment(s) to improve yield?
	\end{itemize}
\end{frame}

\begin{frame}\frametitle{Analysis: Main effects}
	\begin{itemize}
		\item	This is the true surface plot:
	\end{itemize}
	\begin{center}
		\includegraphics[width=\textwidth]{\imagedir/doe/factorial-two-level-surface-example-cropped.png}
	\end{center}
\end{frame}

\begin{frame}\frametitle{Analysis: Main effects}
	\begin{itemize}
		\item	No computer? Use an interaction plot {\scriptsize (see notes for section 1 of the course)}
	\end{itemize}
	\begin{center}
		\includegraphics[width=\textwidth]{\imagedir/doe/factorial-two-level-line-plot.png}
	\end{center}
	\begin{itemize}
		\item	Lines are roughly parallel in this case
		\item	The numbers ``1'', ``2'', ``3'', ``4'' refer to the experiment number in standard order
	\end{itemize}
\end{frame}

\begin{frame}\frametitle{Analysis: interaction effects}
	\begin{itemize}
		\item	Consider a different system now (same variables):
	\end{itemize}
	\begin{center}
		\includegraphics[height=0.85\textheight]{\imagedir/doe/factorial-two-level-with-interactions.png}
	\end{center}
\end{frame}

\begin{frame}\frametitle{Analysis: interaction effects}
	\begin{center}
		\begin{tabulary}
			{\linewidth}{|l|c|c||c|} \hline \textbf{ Experiment } & \textbf{$T$ [K]} & \textbf{$S$ [g/L]} & \textbf{$y$ [\%]}\\\hline
			1 & $-$ (390K) & $-$ (0.5 g/L) & 77 \\\hline
			2 & $+$ (400K) & $-$ (0.5 g/L) & 79 \\\hline
			3 & $-$ (390K) & $+$ (1.25 g/L) & 81 \\\hline
			4 & $+$ (400K) & $+$ (1.25 g/L) & 89 \\\hline 
		\end{tabulary}
	\end{center}
	%\includegraphics[width=\textwidth]{\imagedir/doe/DOE-factorial-factors-with-interaction.png}
	
	\vspace{12pt}
	\begin{itemize}
		\item	Main effect of $T$:
		\vspace{24pt}
		\item	Main effect of $S$:
	\end{itemize}
\end{frame}

\begin{frame}\frametitle{Analysis: interaction effects}

	\begin{center}
		\begin{tabulary}
			{\linewidth}{|l|c|c||c|} \hline \textbf{ Experiment } & \textbf{$T$ [K]} & \textbf{$S$ [g/L]} & \textbf{$y$ [\%]}\\\hline
			1 & $-$ (390K) & $-$ (0.5 g/L) & 77 \\\hline
			2 & $+$ (400K) & $-$ (0.5 g/L) & 79 \\\hline
			3 & $-$ (390K) & $+$ (1.25 g/L) & 81 \\\hline
			4 & $+$ (400K) & $+$ (1.25 g/L) & 89 \\\hline 
		\end{tabulary}
	\end{center}
	
	\vspace{12pt}
	\begin{itemize}
		\item	Main effect of $T$: 5\% per 10K; but reported as \textbf{2.5\% per 5K}
		\begin{itemize}
			\item	$\Delta T_{S+} = 8\%$ per 10K
			\item	$\Delta T_{S-} = 2\%$ per 10K
		\end{itemize}
		\item	Main effect of $S$: 7\% per 0.75g/L; {\small report }\textbf{3.5\% per 0.375g/L}
		\begin{itemize}
			\item	$\Delta S_{T+} = 10\%$ per 0.75g/L
			\item	$\Delta S_{T-} = 4\%$ per 0.75g/L
		\end{itemize}
	\end{itemize}
\end{frame}

\begin{frame}\frametitle{Analysis: interaction effects}
	\begin{center}
		\includegraphics[width=\textwidth]{\imagedir/doe/factorial-two-level-line-plot-with-interaction.png}
	\end{center}
	\begin{itemize}
		\item	Lines not parallel
		\item	Indicates magnitude of effect is not the same at both levels of the variable being held constant
		\item	Implies there is an {\color{purple}{interaction}}
		\begin{itemize}
			\item	In this case, interaction between $T$ and $S$
			\item	could also be called the $S$ and $T$ interaction: symmetrical
			\item	called the $T \times S$ interaction (or $S \times T$ interaction)
			\item	it is a 2-factor interaction (2fi)
		\end{itemize}
	\end{itemize}
\end{frame}

\begin{frame}\frametitle{Analysis: interaction effects}

	Recall system with \textbf{no interaction} (earlier example):
	\begin{itemize}
		\item	Main effect of $T$:
		\begin{itemize}
			\item	$T_{S+} = -11\%$ per 16K
			\item	$T_{S-} = -9\%$ per 16K
		\end{itemize}
		\item	Main effect of $S$:
		\begin{itemize}
			\item	$S_{T+} = -7\%$ per 0.5g/L
			\item	$S_{T-} = -5\%$ per 0.5g/L
		\end{itemize}
		
		\vspace{12pt}
		\item	Almost no difference between the values within each main effect. This is how we tell there is no interaction.
	\end{itemize}
\end{frame}

\begin{frame}\frametitle{Analysis: interaction effects}

	System \textbf{with interaction} (second example):
	\begin{itemize}
		\item	Main effect of $T$: \textbf{5\% per 10K}
		\begin{itemize}
			\item	$T_{S+} = 8\%$ per 10K
			\item	$T_{S-} = 2\%$ per 10K
		\end{itemize}
		\item	Main effect of $S$: \textbf{7\% per 0.75g/L}
		\begin{itemize}
			\item	$S_{T+} = 10\%$ per 0.75g/L
			\item	$S_{T-} = 4\%$ per 0.75g/L
		\end{itemize}
		
		\vspace{12pt}
		\item	There was an important phenomenon that we did not capture with the main effects alone
		\item	The main effects are quite different for each estimate.
		\item	We need ``something else'' to capture this interaction
	\end{itemize}
\end{frame}

\begin{frame}\frametitle{Analysis: interaction effects}
	\begin{itemize}
		\item	$T$ interaction with $S$:
		\begin{itemize}
			\item	$\Delta y$ due to $T$ at high $S$: +8
			\item	$\Delta y$ due to $T$ at low $S$: +2
			\item	The half difference: $[+8 - (+2)]/2$ = \textbf{3}
		\end{itemize}
		\item	$S$ interaction with $T$:
		\begin{itemize}
			\item	$\Delta y$ due to $S$ at high $T$: +10
			\item	$\Delta y$ due to $S$ at low $T$: +4
			\item	The half difference: $[+10 - (+4)]/2$ = \textbf{3}
		\end{itemize}
	\end{itemize}

	Interpretation:
	\begin{itemize}
		\item	$T$ and $S$ increase $y$ by a greater amount when both are high
		\item	Similarly, both terms reduce $y$ when they are of opposite sign.
	\end{itemize}

	\textbf{Interaction terms dominate on a ridge}, and are important as we approach an optimum.
\end{frame}

\begin{frame}\frametitle{Visualizing the interaction: we are on a ridge}
	\begin{center}
		\includegraphics[height=0.65\textheight]{\imagedir/doe/factorial-two-level-with-interactions.png}
	\end{center}
	\vspace{-24pt}
	\begin{columns}[t]
		\column{0.50\textwidth}
			\begin{itemize}
				\item	$T$ interacts with $S$:
				\begin{itemize}
					\item	$\Delta y$ due to $T$ at $S_{+}$: +8
					\item	$\Delta y$ due to $T$ at $S_{-}$: +2
				\end{itemize}
			\end{itemize}
		\column{0.50\textwidth}
			\begin{itemize}
				\item	$S$ interacts with $T$:
				\begin{itemize}
					\item	$\Delta y$ due to $S$ at $T_{+}$: +10
					\item	$\Delta y$ due to $S$ at $T_{-}$: +4
				\end{itemize}
			\end{itemize}
	\end{columns}
	\vspace{6pt}
	$T$ and $S$ increase $y$ when they both operate together (they are synergistic, in this example. In other cases they may cancel out).
\end{frame}

\begin{frame}\frametitle{Analysis by least squares modelling}
	\begin{itemize}
		\item	Return back to system with {\color{myOrange}{\textbf{little interaction}}}:
	\end{itemize}
	
	\begin{tabulary}
		{\linewidth}{|l|c|c||c|} \hline \textbf{ Experiment } & \textbf{ $T$ [K] } & \textbf{ $S$ [g/L] } & \textbf{ $y$ [\%] }\\ \hline
		Baseline & \textbf{346 K} & \textbf{1.50} & \\\hline
		1 & $-$ (338K) & $-$ (1.25 g/L) & 69 \\\hline
		2 & $+$ (354K) & $-$ (1.25 g/L) & 60 \\\hline
		3 & $-$ (338K) & $+$ (1.75 g/L) & 64 \\\hline
		4 & $+$ (354K) & $+$ (1.75 g/L) & 53 \\\hline
	\end{tabulary}
	
	\begin{itemize}
		\item	{\color{purple}{Standard form}}: $\dfrac{\text{variable} - \text{center point}}{\text{range}/2}$
		\item	$T_{-} = \dfrac{338 - 346}{(354-338)/2} = \dfrac{-8}{8} = -1$
		\item	$S_{-} = \dfrac{1.25 - 1.50}{(1.75 - 1.25)/2} = \dfrac{-0.25}{0.25} = -1$
		\item	$T_{+} = +1$
		\item	$S_{+} = +1$
	\end{itemize}
\end{frame}

\begin{frame}\frametitle{Analysis by least squares modelling}
	\begin{block}{Least squares model}
		\begin{center}
			$
			\begin{array}{rcl}
				y &= \beta_0 + \beta_Tx_T + \beta_S x_S + \beta_{TS} x_Tx_S + \varepsilon \\
				y &= b_0 + b_Tx_T + b_S x_S + b_{TS} x_Tx_S + e
			\end{array}
			$
		\end{center}
	\end{block}
	\begin{itemize}
		\item	4 parameters to estimate: $b_0, b_T, b_S, b_{TS}$
		\item	4 data points
		\item	Zero degrees of freedom (i.e. $S_E = 0$, no confidence intervals possible)
	\end{itemize}
\end{frame}

\begin{frame}\frametitle{Analysis by least squares modelling}
	\textbf{Aim}: Write out the LS model equation for the 4 data points; stack them as rows in a matrix.
	
	\vspace{12pt}
	For example, for the first experiment from the standard order table, which was run at low $T$ and low $S$:
	
	\begin{center}
		$
		\begin{array}{rcccccccccc}
			y_1 &= &b_0 &+ &b_Tx_T    &+ &b_S x_S   &+ &b_{TS} x_Tx_S     &+ &e_1\\
			y_1 &= &b_0 &+ &b_T T_{-} &+ &b_S S_{-} &+ &b_{TS} T_{-}S_{-} &+ &e_1\\
		\end{array}
		$
	
		\vspace{12pt}
		$
		\begin{array}{rcccl}
			\begin{bmatrix}
				y_1\\
				y_2\\
				y_3 \\
				y_4
			\end{bmatrix}
			&=&
			\begin{bmatrix}
				1 & T_{-} & S_{-} & T_{-}S_{-}\\
				1 & T_{+} & S_{-} & T_{+}S_{-}\\
				1 & T_{-} & S_{+} & T_{-}S_{+}\\
				1 & T_{+} & S_{+} & T_{+}S_{+}\\
			\end{bmatrix}
			\begin{bmatrix}
				b_0 \\
				b_T \\
				b_S \\
				b_{TS}
			\end{bmatrix}
			&+&
			\begin{bmatrix}
				e_1\\
				e_2\\
				e_3 \\
				e_4
			\end{bmatrix}
			\\
			\\
			\mathbf{y} &=& \mathbf{X} \mathbf{b} &+& \,\,\,\,\,\mathbf{e} \\
		\end{array}
		$
	\end{center}
\end{frame}

\begin{frame}\frametitle{Visualizing the least squares modelling}
	\begin{itemize}
		\item	Least squares model for DOE in 2 factors
	\end{itemize}
	\begin{center}
		\includegraphics[width=0.85\textwidth]{\imagedir/doe/two-factor-least-squares-interpretation.png}
	\end{center}
	\begin{itemize}
		\item	Interaction term is small: blue plane is flat
		\item	Interaction term is large: plane has curvature
	\end{itemize}
\end{frame}

\begin{frame}\frametitle{Analysis by least squares modelling}

	$
	\begin{array}{rcccl}
		\begin{bmatrix}
			y_1\\
			y_2\\
			y_3 \\
			y_4
		\end{bmatrix}
		&=&
		\begin{bmatrix}
			1 & T_{-} & S_{-} & T_{-}S_{-}\\
			1 & T_{+} & S_{-} & T_{+}S_{-}\\
			1 & T_{-} & S_{+} & T_{-}S_{+}\\
			1 & T_{+} & S_{+} & T_{+}S_{+}\\
		\end{bmatrix}
		\begin{bmatrix}
			b_0 \\
			b_T \\
			b_S \\
			b_{TS}
		\end{bmatrix}
		&+&
		\begin{bmatrix}
			e_1\\
			e_2\\
			e_3 \\
			e_4
		\end{bmatrix}
		\\
		\\
		\begin{bmatrix}
			69\\
			60\\
			64\\
			53
		\end{bmatrix}
		&=&
		\begin{bmatrix}
			1 & -1 & -1 & +1\\
			1 & +1 & -1 & -1\\
			1 & -1 & +1 & -1\\
			1 & +1 & +1 & +1\\
		\end{bmatrix}
		\begin{bmatrix}
			b_0 \\
			b_T \\
			b_S \\
			b_{TS}
		\end{bmatrix}
		&+&
		\begin{bmatrix}
			e_1\\
			e_2\\
			e_3 \\
			e_4
		\end{bmatrix}
		\\
		\mathbf{y} &=& \mathbf{X} \mathbf{b} &+& \,\,\,\,\,\mathbf{e} \\
	\end{array}
	$
	
	\vspace{12pt}
	$\mathbf{X}$ matrix is trivial to set up:
	\begin{itemize}
		\item	Intercept column: is always a column of 1's
		\item	$b_T$ column: comes directly from standard table
		\item	$b_S$ column: comes directly from standard table
		\item	$b_{TS}$ column: is the product of the $b_T$ and $b_S$ columns
	\end{itemize}
\end{frame}

\begin{frame}\frametitle{Analysis by least squares modelling}
	\begin{itemize}
		\item	$\mathbf{X}^T\mathbf{X} =
		\begin{bmatrix}
			4 & 0 & 0 & 0\\
			0 & 4 & 0 & 0\\
			0 & 0 & 4 & 0\\
			0 & 0 & 0 & 4
		\end{bmatrix}
		$
		\item	$\mathbf{X}^T\mathbf{y} =
		\begin{bmatrix}
			246 \\
			-20 \\
			-12 \\
			-2
		\end{bmatrix}
		$
	\end{itemize}
\end{frame}

\begin{frame}\frametitle{Analysis by least squares modelling}

	$
	\begin{array}{rcl}
		\mathbf{b} &=& (\mathbf{X}^T\mathbf{X})^{-1}\mathbf{X}^T\mathbf{y} =
		\begin{bmatrix}
			1/4 & 0 & 0 & 0\\
			0 & 1/4 & 0 & 0\\
			0 & 0 & 1/4 & 0\\
			0 & 0 & 0 & 1/4
		\end{bmatrix}
		\begin{bmatrix}
			246 \\
			-20 \\
			-12 \\
			-2
		\end{bmatrix}
		\\
		\\
		\mathbf{b} &=&
		\begin{bmatrix}
			61.5 \\
			-5 \\
			-3 \\
			-0.5
		\end{bmatrix}
	\end{array}
	$
	\begin{itemize}
		\item	$y = b_0 + b_Tx_T + b_S x_S + b_{TS} x_Tx_S + e$
		\item	$y = 61.5 - 5 x_T - 3 x_S + - 0.5 x_Tx_S + e$
	\end{itemize}

	You can easily calculate these effects by hand:
	\begin{itemize}
		\item	$(+ 69 + 60 + 64 + 53)/4 = 61.5$
		\item	$(- 69 + 60 - 64 + 53)/4 = -5$
		\item	$(- 69 - 60 + 64 + 53)/4 = -3$
		\item	$(+ 69 - 60 - 64 + 53)/4 = -0.5$
	\end{itemize}
\end{frame}

\begin{frame}\frametitle{Analysis by least squares modelling}
	\vspace{6pt}
	\begin{enumerate}
		\item	$\mathbf{X}^T\mathbf{X}$: zeros on off-diagonals
		\begin{itemize}
			\item	orthogonal matrix
			\item	each column is varied independently of the others
			\item	calculate the $k^\text{th}$ slope coefficient separately: $b_k = \dfrac{x_k^Ty}{x_k^Tx_k}$
		\end{itemize}
		\item	Interpret $b_T = -5$?
		\begin{itemize}
			\item	$x_T$ is the change in {\color{purple}{\emph{normalized temperature}}} by 1 unit
			\item	Changing $x_T$ from 0 to 1 implies $T_\text{actual}$ changes from 346K to 354K (baseline to high level)
			\item	Changing $x_T$ from -1 to 0 implies $T_\text{actual}$ changes from 338K to 346K (low level to baseline)
			\item	$-5$\% decrease in conversion for every 8K increase in temperature
		\end{itemize}
		\item	Now interpret $b_S = -3$?
		\item	How to use this model for a prediction?
	\end{enumerate}
\end{frame}

\begin{frame}\frametitle{Analysis by least squares modelling}
	\begin{itemize}
		\item	{\footnotesize The least squares model was} $y = 61.5 - 5 x_T - 3 x_S - 0.5 x_Tx_S + e$
		\item	The geometric construction was:
	\end{itemize}
	\begin{center}
		\includegraphics[width=\textwidth]{\imagedir/doe/factorial-two-levels-two-variables-with-analysis.png}
	\end{center}
	{\color{myOrange}{\emph{The interpretations agree!}}}
\end{frame}

\begin{frame}\frametitle{Analysis by least squares modelling}

	Return to system with {\color{myOrange}{\textbf{high interaction}}}
	\begin{itemize}
		\item	Base line: $T$ = 395K and $S$ = (1.25+0.5)/2 = 0.875 g/L
		\item	Calculate deviation variables
		\item	Build the matrices and calculate $\mathbf{b} = (\mathbf{X}^T\mathbf{X})^{-1}\mathbf{X}^T\mathbf{y}$
		\item	Verify at home: $y = 81.5 + 2.5 x_T + 3.5 x_S + 1.5 x_T x_S$
	\end{itemize}
	
	\vspace{12pt}
	Large interaction is confirmed in least squares model due to high value of the 1.5 coefficient on the $x_T x_S$ term
\end{frame}

\begin{frame}\frametitle{Analysis by least squares modelling: visualizing it}
	{\color{myOrange}{High interaction system:}}
	\vspace{12pt}
	\begin{columns}
		\column{5cm} 
			\textbf{Ignoring interaction term:}
			\begin{center}
				\includegraphics[width=\textwidth]{\imagedir/doe/factorial-two-level-surface-without-interaction-slides-cropped.png}
			\end{center}
			We are estimating a linear equation using linear least squares. 
		
		\column{5cm} 
			\textbf{With interaction term:}
			\begin{center}
				\includegraphics[width=\textwidth]{\imagedir/doe/factorial-two-level-surface-with-interaction-slides-cropped.png}
			\end{center}
			We are estimating a non-linear equation using linear least squares.
	\end{columns}
\end{frame}

\begin{frame}\frametitle{DOE of a 3-factor experiment}

	Plastics molding factory; waste treatment.
	\begin{itemize}
		\item	Factor 1: $C$: chemical compound added (A or B)
		\item	Factor 2: $T$: treatment temperature (72F or 100F)
		\item	Factor 3: $S$: stirring speed (200 rpm or 400 rpm)
		\item	$y$ = amount of pollutant discharged [lb]
	\end{itemize}
	\begin{center}
		\includegraphics[width=\textwidth]{\imagedir/doe/DOE-3-factor-factorial-example.png}
	\end{center}
	\begin{itemize}
		\item	Categorial variables: A=$-1$ and B=$+1$ (or \emph{vice versa})
	\end{itemize}
\end{frame}

\begin{frame}\frametitle{DOE of a 3-factor experiment}

	Example on the board:
	\begin{enumerate}
		\item	Geometric illustration of the data
		\item	Calculate main effects
		\item	Calculate the 3 two-factor interactions, and the single 3 factor interaction
		\begin{itemize}
			\item	$C \times T$ and $C \times S$ and $T \times S$ and $C \times T \times S$
		\end{itemize}
		\item	Main effects and interactions using least squares (by-hand)
		\item	Computer verification:
		\begin{itemize}
			\item	$y = 11.25 + 6.25x_C + 0.75x_T -7.25x_S + 0.25 x_{CT}$\\
					$-6.75 x_C x_S -0.25 x_T x_S - 0.25 x_Cx_Tx_S$
		\end{itemize}
	\end{enumerate}
\end{frame}

\begin{frame}\frametitle{Summary of factorial designs}
	\begin{itemize}
		\item	Good visual interpretation, even on paper
		\item	Few experiments, but powerful information
		\item	Building blocks for complex designs
		\item	$2^k$ experiments for $k$ factors
		\item	Each factor is varied independently of the others
		\item	Each factor in model can be interpreted independently
		\item	Least squares model easily derived by hand
		\item	\textbf{Main effects cannot be interpreted separate from their interactions}
		\begin{itemize}
			\item	$y = b_0 + b_P x_P + b_Q x_Q + b_{PQ} x_Px_Q + e$
		\end{itemize}
		\item	Sometimes a small effect is desirable: implies $y$ not sensitive that factor
	\end{itemize}
\end{frame}

\begin{frame}\frametitle{Summary of factorial designs}
	\begin{block}{Much more efficient than change one-single factor at-a-time (COST)}
		\begin{center}
			\includegraphics[width=0.65\textwidth]{\imagedir/doe/comparison-of-variances.png}
		\end{center}
	\end{block}
	\begin{itemize}
		\item	COST: cannot estimate interactions
		\item	We could rescue this COST design by adding $y_4$
	\end{itemize}
\end{frame}

\begin{frame}\frametitle{Review: Change one variable at a time}
	\begin{center}
		\includegraphics[width=0.7\textwidth]{\imagedir/doe/COST-contours.png}
	\end{center}
	If COST in a cross shape (expts 2, 3, 4, 5, 6):
	\begin{itemize}
		\item	we cannot estimate interactions
		\item	only a single estimate of each main effect
		\item	rescued to a full factorial: e.g. use expts 2, 3, 6 and add new point below 2, to the right of 6
	\end{itemize}
\end{frame}

% \begin{frame}\frametitle{Review: a better approach - full factorial}
% 	\begin{center}
% 		\includegraphics[width=\textwidth]{\imagedir/doe/factorial-two-levels-two-variables-contour-plot.png}
% 	\end{center}
% \end{frame}
% 
% \begin{frame}\frametitle{Review: DOE of a 3-factor experiment}
% 
% 	Plastics molding factory; waste treatment.
% 	\begin{itemize}
% 		\item	Factor 1: $C$: chemical compound added (A or B)
% 		\item	Factor 2: $T$: treatment temperature (72 F or 100F)
% 		\item	Factor 3: $S$: stirring speed (200 rpm or 400 rpm)
% 		\item	$y$ = amount of pollutant discharged [lb]
% 	\end{itemize}
% 	\begin{center}
% 		\includegraphics[width=\textwidth]{\imagedir/doe/DOE-3-factor-factorial-example.png}
% 	\end{center}
% \end{frame}
% 
% \begin{frame}\frametitle{Review: DOE of a 3-factor experiment}
% 	\begin{enumerate}
% 		\item	Geometric illustration of the data
% 		\item	Calculate main effects, \textbf{C}, \textbf{T} and \textbf{S}
% 		\item	Calculate the 3 two-factor interactions:
% 		\begin{itemize}
% 			\item	\textbf{CT}, \textbf{CS} and \textbf{TS}
% 		\end{itemize}
% 		\item	and the single 3 factor interaction
% 		\begin{itemize}
% 			\item	\textbf{CTS}
% 		\end{itemize}
% 		\item	Main effects and interactions using least squares (by-hand)
% 		\item	Computer verification:
% 		\begin{itemize}
% 			\item	$y = 11.25 + 6.25x_C + 0.75x_T -7.25x_S + 0.25 x_C x_T -6.75 x_C x_S -0.25 x_T x_S - 0.25 x_C x_T x_S$
% 		\end{itemize}
% 	\end{enumerate}
% \end{frame}

\begin{frame}\frametitle{Significance of effects}
	\begin{itemize}
		\item	For a $2^k$ factorial:
		\begin{itemize}
			\item	$2^k$ parameters in the least squares model
			\item	$2^k$ data points collected
			\item	implies $S_E = 0$
			\item	Zero degrees of freedom
		\end{itemize}
	\end{itemize}

	How to judge if an effect is significant? Consider 2 approaches.
\end{frame}

\begin{frame}\frametitle{Significant? : Pareto-plot}
	\begin{itemize}
		\item	$2^4$ factorial: 15 parameters + intercept
		\item	Bar plot: see code on course website. Bar colour indicates sign: blue are negative and orange are positive coefficients.
	\end{itemize}
	\begin{center}
		\includegraphics[width=0.65\textwidth]{\imagedir/doe/pareto-plot-full-fraction.png}
	\end{center}
\end{frame}

\begin{frame}\frametitle{Significant? : Pareto-plot}
	\begin{itemize}
		\item	For normal L/S models: cannot compare coefficients in this way
		\item	Why can we with a DOE model?
		\item	$x_i = \dfrac{x_{\text{actual}, i}- \text{center point}}{\frac{1}{2} (\text{range of X})}$
		\item	Caution: the \emph{range of X} should span a reasonable range
	\end{itemize}
	\begin{itemize}
		\item	Caution: if an interaction is significant (e.g. \textbf{BC}), then no need to test the main effects, \textbf{B} and \textbf{C}
		\begin{itemize}
			\item	these main effects are ``automatically'' significant
			\item	even if they have small numeric coefficients
			\item	since \textbf{B} and \textbf{C} act together to affect response $y$
			\item	so never exclude main effects whose interactions are significant
		\end{itemize}
	\end{itemize}
\end{frame}

\begin{frame}\frametitle{Significant effect?}

	We require degrees of freedom to construct confidence intervals.

	Two ways to get DoF:
	\begin{enumerate}
		\item	Replicate experiments
		\item	Drop out a factor from a full factorial; refit the model without the dropped factor(s).
	\end{enumerate}
	\begin{itemize}
		\item	Complete replication is expensive - redo everything from scratch:
		\begin{itemize}
			\item	Setup equipment
			\item	Run the complete experiment
			\item	Take samples
			\item	Measure results
		\end{itemize}
		\item	There are better ways to spend our experimental budget (see later)
	\end{itemize}
\end{frame}

\begin{frame}\frametitle{Significant? : Replicate runs}
	\begin{itemize}
		\item	E.g replicated $2^3$ factorial: 8 + 8 runs
		\item	$y_{i,1}$ and $y_{i,2}$ at condition i $(i=1, 2, \ldots, 8)$
		\item	$\bar{y}_i = 0.5y_{i,1} + 0.5y_{i,2}$
		\item	$d_i = y_{i,2} - y_{i,1}$,
		\item	$s_i^2 = \dfrac{(y_{i,1} - \bar{y}_i)^2 + (y_{i,2} - \bar{y}_i)^2}{1}$
		\item	$s_i^2 = d_i^2/2$
		\item	Pool variances for all $2^k$ levels
		\item	$\hat{\sigma}^2 = S_E^2 = \dfrac{1}{2}\displaystyle\sum_i^{2^k}{d_i^2}$
		\item	Errors are $t$-distributed with $2^k$ degrees of freedom
	\end{itemize}
\end{frame}

\begin{frame}\frametitle{Significant? : Replicate runs}
	\begin{itemize}
		\item	Once we have $S_E \rightarrow S_E(b_i) \rightarrow$ confidence interval for $b_i$
		\item	$S_E(b_i) = \sqrt{\mathcal{V}\left(b_i\right)} = \sqrt{\dfrac{S_E^2}{\sum{x_i^2}}}$
		\item	$x_i$ is either $-1$ or $+1$ from the $i^\text{th}$ column
		\item	Confidence intervals are independent \emph{!}
		\item	Then determine if a main effect or interaction is significant
	\end{itemize}
	\vspace{12pt}
	{\color{myOrange}{So how do we get degrees of freedom to calculate \( S_E \)?}}
\end{frame}

\begin{frame}\frametitle{Significant? : Replicate runs}

	A $2^3$ factorial with 3 extra runs at the center point:

	$
	\begin{array}{rcl}
		\mathrm{y} &=& \mathrm{X} \mathrm{b} + \mathrm{e}\\
		\mathrm{y} &=&
		\begin{bmatrix}
			1 & -1 & -1 & -1 & +1 & +1 & +1 & -1\\
			1 & +1 & -1 & -1 & -1 & -1 & +1 & +1\\
			1 & -1 & +1 & -1 & -1 & +1 & -1 & +1\\
			1 & +1 & +1 & -1 & +1 & -1 & -1 & -1\\
			1 & -1 & -1 & +1 & +1 & -1 & -1 & +1\\
			1 & +1 & -1 & +1 & -1 & +1 & -1 & -1\\
			1 & -1 & +1 & +1 & -1 & -1 & +1 & -1\\
			1 & +1 & +1 & +1 & +1 & +1 & +1 & +1\\
			1 & 0 & 0 & 0 & 0 & 0 & 0 & 0\\
			1 & 0 & 0 & 0 & 0 & 0 & 0 & 0\\
			1 & 0 & 0 & 0 & 0 & 0 & 0 & 0
		\end{bmatrix}
		\begin{bmatrix}
			b_0 \\
			b_A \\
			b_B \\
			b_{C} \\
			b_{AB} \\
			b_{AC} \\
			b_{BC} \\
			b_{ABC}
		\end{bmatrix}
		+ \mathrm{e}
	\end{array}
	$
	\begin{itemize}
		\item	$S_E(b_i) = \sqrt{\dfrac{S_E^2}{\sum{x_i^2}}} = \sqrt{\dfrac{S_E^2}{8}} $ for all parameters, except the intercept
	\end{itemize}
\end{frame}

\begin{frame}\frametitle{Significant? : Replicate runs}
	\begin{itemize}
		\item	Add as many center points as you like: adds DOF
		\item	without changing orthogonality of $\mathrm{X}$
		\item	decreases $S_E$
	\end{itemize}

	But standard error for factor $b_i = S_E(b_i) = \sqrt{\dfrac{S_E^2}{\sum{x_i^2}}}$
	\begin{itemize}
		\item	denominator not affected by center points
		\item	Only the intercept term, $b_0$, has improved confidence interval
	\end{itemize}
\end{frame}

\begin{frame}\frametitle{Significant? : Replicate runs}

	Reporting the standard error: just write the $S_E(b_i)$ value next to the effect:
	\begin{itemize}
		\item	$\text{Temperature effect}, b_T = 11.5 \pm 0.707$
		\item	$\text{Catalyst effect}, b_K = 0.75 \pm 0.707$
	\end{itemize}

	The $0.707$ value is just $S_E(b_i)$; leave it up to user to choose their level of significance and to calculate the CI.
\end{frame}

\begin{frame}\frametitle{Side note: Box, Hunter and Hunter}
	\begin{itemize}
		\item	Interpretation of effects \textbf{in this course}: \emph{change in response over \textbf{half the range} of the factor}
		\begin{itemize}
			\item	center = 400K
			\item	lower level = 375K
			\item	upper level = 425K
			\item	Effect of ``$-5$'' would be interpreted as ...
			\item	Advantage: matches computer software
			\item	Disadvantage: interpreting binary factors (e.g. catalyst A or B)
		\end{itemize}
	\end{itemize}
	\begin{itemize}
		\item	Box, Hunter and Hunter: \emph{change in response over \textbf{full range} of the factor}
		\begin{itemize}
			\item	Divide BHH effect values by 2
			\item	Divide BHH standard errors by 2 also
			\item	Advantage: binary variables
		\end{itemize}
	\end{itemize}
\end{frame}

\begin{frame}\frametitle{Refitting model}
	\begin{itemize}
		\item	Delete non-significant effects (parameters)
		\item	Interesting note: factors retained in the model keep same slope coefficient values
		\begin{itemize}
			\item	Why? We are only deleting \emph{columns} from the $\mathbf{X}$ matrix
		\end{itemize}
		\item	Now least squares model has residuals and DOF
		\item	$S_E \neq 0$
		\item	Use all previous tools from least squares to check model
		\begin{itemize}
			\item	q-q plot of residuals
			\item	plots of residuals, \emph{etc}
		\end{itemize}
		\item	Use $S_E(b_i)$ to verify the effects are significant
	\end{itemize}
\end{frame}

\begin{frame}\frametitle{Efficiency: COST vs DOE}
	\begin{block}{Much more efficient than one-at-a-time (COST)}
		\begin{center}
			\includegraphics[width=0.65\textwidth]{\imagedir/doe/comparison-of-variances.png}
		\end{center}
	\end{block}
	Consider the $b_T$ main effect as an example:
	\begin{columns}
		\column{3.5cm}
			\begin{itemize}
				\item	$b_T = y_2 - y_1$
				\item	$\mathcal{V}(b_T) = \sigma_y^2 + \sigma_y^2$
				\item	$\mathcal{V}(b_T) = 2\sigma_y^2$
			\end{itemize}
		\column{7cm}
			\begin{itemize}
				\item	$b_T = 0.5(y_2 - y_1) + 0.5(y_4 - y_3)$
				\item	$\mathcal{V}(b_T) = 0.25(\sigma_y^2 + \sigma_y^2) + 0.25(\sigma_y^2 + \sigma_y^2)$
				\item	$\mathcal{V}(b_T) = \sigma_y^2$
			\end{itemize}
	\end{columns}
	\begin{itemize}
		\item	COST cannot estimate interactions!
	\end{itemize}
\end{frame}

\begin{frame}\frametitle{Summary of factorial designs}
	\begin{itemize}
		\item	Good visual interpretation, even on paper
		\item	Few experiments, but powerful information
		\item	Building blocks for complex designs (next part)
		\item	$2^k$ experiments for $k$ factors
		\item	Each factor is varied independently of the others
		\item	Least squares model easily analyzed by hand
		\item	Model coefficients have the lowest variability possible: $(\mathrm{X}^T\mathrm{X})^{-1}S_E^2$
		\item	Small effect sometimes desirable: $y$ not sensitive the $x$
	\end{itemize}
\end{frame}

\begin{frame}\frametitle{Disturbances}
	\begin{itemize}
		\item	Ideally: all disturbances are \textbf{controlled}; have no effect on $y$
	\end{itemize}

	We always have \textbf{unknown, unmeasurable and uncontrollable} disturbances
	\begin{itemize}
		\item	That's why we randomize experiments
		\item	\emph{Example}: a side-reaction due to impurity in reactant affects $y$
		\begin{itemize}
			\item	Impurity is not uniform in the reactant
		\end{itemize}
		\item	\emph{Example}: catalyst deactivation, equipment fouling
		\item	\emph{Example}: you get better as experiments progress (e.g. weightlifting)
	\end{itemize}

	\textbf{Known, or controllable, or measurable} disturbances:
	\begin{itemize}
		\item	Operators, physical equipment
		\item	Pairing: cancel out the disturbance by using the same subject
		\item	Blocking: disturbance is known to affect $y$
		\begin{itemize}
			\item	but we design experiment to minimize its effect.
		\end{itemize}
	\end{itemize}
\end{frame}

\begin{frame}\frametitle{Classifying variables in an experiment}
	Write down \textbf{all possible} variables that can affect $y$. Those variables are either:
	\begin{itemize}
		\item	{\color{purple}{Measurable}} = able to quantify that variable, or not.
		\item	{\color{purple}{Controllable}} = able to adjust that variable, or not.
	\end{itemize}
	\begin{center}
		\includegraphics[width=\textwidth]{\imagedir/doe/disturbance-classification.png}
	\end{center}
	Always record your covariates: add them into L/S model later.
\end{frame}

\begin{frame}\frametitle{Blocking and confounding}
	\begin{itemize}
		\item	3 factor experiment: $2^3$ runs
		\item	To estimate: $\mathrm{b} = [b_0, b_A, b_B, b_C, b_{AB}, b_{AC}, b_{BC}, b_{ABC}]$
	\end{itemize}

	But:
	\begin{itemize}
		\item	Enough material for only 4 runs at a time
		\item	Material could affect response.
		\item	How can we minimize that effect?
	\end{itemize}
\end{frame}

\begin{frame}\frametitle{Blocking and confounding}
	\begin{itemize}
		\item	Intentionally confound with 3-factor interaction: \textbf{ABC}
	\end{itemize}
	\begin{center}
		\includegraphics[width=\textwidth]{\imagedir/doe/table-blocking-3-factor-factorial-no-y.png}
	\end{center}
	\vspace{12pt}
	{\color{myOrange}{Important note}}: you must still run your experiments in random order. Do \textbf{NOT} run all the batch 1, then all the batch 2 experiments.
\end{frame}

\begin{frame}\frametitle{Blocking and confounding}
	\begin{center}
		\includegraphics[width=\textwidth]{\imagedir/doe/blocking-factorial-3-factors.png}
	\end{center}
\end{frame}

\begin{frame}\frametitle{Blocking and confounding}
	\begin{center}
		\includegraphics[width=\textwidth]{\imagedir/doe/table-blocking-3-factor-factorial-with-y.png}
	\end{center}
	\begin{itemize}
		\item	Batch 1: $\widetilde{y}_i \leftarrow y_{i,\text{actual}} + g, \qquad\qquad i = 1, 4, 6, 7$
		\item	Batch 2: $\mathring{y}_i \leftarrow y_{i,\text{actual}} + h, \qquad\qquad i = 2, 3, 5, 8$
	\end{itemize}
	e.g, $h$ = incremental, but \emph{unmeasurable} effect on $y$ due to batch 2 
	
	\vspace{12pt}
	Disturbance bias cancels out. E.g. for $b_A$: $ \widehat{\beta}_A = -\widetilde{y}_1 + \mathring{y}_2 - \mathring{y}_3 + \widetilde{y}_4 - \mathring{y}_5 + \widetilde{y}_6 - \widetilde{y}_7 + \mathring{y}_8 $
\end{frame}

\begin{frame}\frametitle{Blocking and confounding}
	\begin{itemize}
		\item	It is as we have a new variable, \textbf{D}, in the model
	\end{itemize}

	$
	\begin{array}{rcl}
		& &
		\begin{matrix}
			b_0 & b_A & b_B & b_{C} & b_{AB} & b_{AC}& b_{BC} &b_{ABC} & b_{D}
		\end{matrix}
		\\
		\mathrm{y} &=&
		\begin{bmatrix}
			1\,\, & -1\,\, & -1\,\, & -1\,\, & +1\,\, & +1\,\, & +1\,\, & -1\,\, & -1\\
			1 & +1 & -1 & -1 & -1 & -1 & +1 & +1 & +1\\
			1 & -1 & +1 & -1 & -1 & +1 & -1 & +1 & +1\\
			1 & +1 & +1 & -1 & +1 & -1 & -1 & -1 & -1\\
			1 & -1 & -1 & +1 & +1 & -1 & -1 & +1 & +1\\
			1 & +1 & -1 & +1 & -1 & +1 & -1 & -1 & -1\\
			1 & -1 & +1 & +1 & -1 & -1 & +1 & -1 & -1\\
			1 & +1 & +1 & +1 & +1 & +1 & +1 & +1 & +1
		\end{bmatrix}
		\begin{bmatrix}
			b_0 \\
			b_A \\
			b_B \\
			b_{C} \\
			b_{AB} \\
			b_{AC} \\
			b_{BC} \\
			b_{ABC} \\
			b_{D}
		\end{bmatrix}
		+ \mathrm{e}
	\end{array}
	$
	\begin{itemize}
		\item	$\widehat{\beta}_{ABC} \rightarrow \underbrace{\text{ABC interaction}}_{\text{expected to be small}} + \text{raw material effect, }\mathbf{D}$
		\item	Block effect generated from the other 3 effects: \textbf{D = ABC}
		\item	Note the identical column signs for $b_{ABC}$ and $b_{D}$
	\end{itemize}
\end{frame}

\begin{frame}\frametitle{Fractional factorial designs}

	Real systems: many factors affect the $y$

	Cell-culture example: many factors to investigate
	\begin{enumerate}
		\item	the temperature profile
		\begin{itemize}
			\item	$T_{-}$: fast initial ramp then constant
			\item	$T_{+}$: a slow increase over time
		\end{itemize}
		\item	dissolved oxygen
		\item	agitation rate
		\item	pH
		\item	substrate type (A or B)
	\end{enumerate}

	Would require $2^5 = 32$ runs; 10 days per cell culture $\approx$ 1 year
	\begin{itemize}
		\item	Can we run a subset of the full factorial?
		\item	E.g. we are not interested in ``$\text{dissolved oxygen} \times \text{agitation rate} \times \text{pH}$'' interaction
		\item	Only interested in main effects, maybe the 2 factor interactions
		\begin{itemize}
			\item	how many effects is this?
			\item	can we save money and/or time?
		\end{itemize}
	\end{itemize}
\end{frame}

\begin{frame}\frametitle{Half fractions}
	\begin{itemize}
		\item	Called $\tfrac{1}{2}2^k = 2^{k-1}$ factorials. Let's look at a simple example:
		\item	A full factorial for a $2^3 = 8$ system:
	\end{itemize}
	\begin{center}
		\includegraphics[width=\textwidth]{\imagedir/doe/half-fraction-justification.png}
	\end{center}
	\begin{itemize}
		\item	Want to do half the number of experiments: $\tfrac{1}{2}2^3 = 2^{3-1}$
		\item	Which 4 of the 8 original experiments (runs) do we pick?
	\end{itemize}
\end{frame}

\begin{frame}\frametitle{Half fractions}
	\begin{center}
		\includegraphics[width=0.5\textwidth]{\imagedir/doe/half-fraction-in-3-factors-no-labels.png}
	\end{center}
	\begin{itemize}
		\item	Run either the open or closed set of 4 runs
		\item	In the example that follows: run the closed corners
	\end{itemize}
\end{frame}

\begin{frame}\frametitle{Half fractions}
	\begin{itemize}
		\item	Why did we choose those 4 runs? Why not some other combination?
	\end{itemize}
	\vspace{-6pt}
	\begin{center}
		\includegraphics[width=0.58\textwidth]{\imagedir/doe/projectivity-of-a-half-fraction-in-3-factors.png}
	\end{center}
	If we choose our runs in a smart way, then fractional factorials will collapse to full factorials if an effect is insignificant.
	\begin{itemize}
		\item	e.g. if the \textbf{C} effect is insignificant, we have a full factorial in \textbf{A} and \textbf{B}
		\item	Called \emph{projectivity}
	\end{itemize}
\end{frame}

\begin{frame}\frametitle{Factorial design tradeoffs}

	Tradeoff table {\color{myOrange}{(\emph{bring a copy to the exam!})}}
	\begin{center}
		\includegraphics[height=0.85\textheight]{\imagedir/doe/DOE-trade-off-table.png}
	\end{center}
\end{frame}

\begin{frame}\frametitle{Half fractions}
	\begin{itemize}
		\item	The table tells us to generate \textbf{C=AB}
		\begin{itemize}
			\item	Run factor \textbf{A} and \textbf{B} in the usual factorial way ($2^2 = 4$), then
			\item	``\emph{{\color{purple}{generate}} factor} \textbf{C} \emph{from the product of} \textbf{A} and \textbf{B}"
		\end{itemize}
	\end{itemize}
	\begin{center}
		\includegraphics[width=\textwidth]{\imagedir/doe/half-fraction-table.png}
	\end{center}
	\begin{itemize}
		\item	What is {\color{purple}{confounded}} (i.e. which columns are the same)?
		\begin{itemize}
			\item	\textbf{A=BC}
			\item	\textbf{B=AC}
			\item	\textbf{C=AB} (this was intentional - from table)
			\item	\textbf{I=ABC}
		\end{itemize}
	\end{itemize}
	\begin{itemize}
		\item	Notice: we have picked the 4 runs where the \textbf{ABC} interaction is the same sign: all $+$ (i.e. the \textbf{ABC} factor is not varying in these experiments)
	\end{itemize}
\end{frame}

\begin{frame}\frametitle{Half fractions}

	\begin{exampleblock}{}
		{\color{myOrange}{Use the least squares model to determine what we lose by only doing 4 runs, instead of the full 8.}}
	\end{exampleblock}
	$
	\begin{array}{rcl}
		\mathrm{y} &=& \mathrm{X} \mathrm{b} + \mathrm{e}\\
		\begin{bmatrix}
			y_1\\
			y_2\\
			y_3 \\
			y_4
		\end{bmatrix}
		&=&
		\begin{bmatrix}
			+1 & -1 & -1 & +1 & +1 & -1 & -1 & +1\\
			+1 & +1 & -1 & -1 & -1 & -1 & +1 & +1\\
			+1 & -1 & +1 & -1 & -1 & +1 & -1 & +1\\
			+1 & +1 & +1 & +1 & +1 & +1 & +1 & +1
		\end{bmatrix}
		\begin{bmatrix}
			b_0 \\
			b_A \\
			b_B \\
			b_{C} \\
			b_{AB} \\
			b_{AC} \\
			b_{BC} \\
			b_{ABC}
		\end{bmatrix}
		+
		\begin{bmatrix}
			e_1\\
			e_2\\
			e_3 \\
			e_4
		\end{bmatrix}
	\end{array}
	$

	Several problems:
	\begin{itemize}
		\item	$\mathrm{X}$ not orthogonal
		\item	More unknowns than equations
	\end{itemize}
	Model cannot be calculated.
\end{frame}

\begin{frame}\frametitle{Half fractions}
	\begin{itemize}
		\item	Collect perfectly correlated columns together:
	\end{itemize}
	
	\vspace{12pt}
	$
	\begin{array}{rcl}
		\mathrm{y} &=& \mathrm{X} \mathrm{b} + \mathrm{e}\\
		\begin{bmatrix}
			y_1\\
			y_2\\
			y_3 \\
			y_4
		\end{bmatrix}
		&=&
		\begin{bmatrix}
			+1 & -1 & -1 & +1 \\
			+1 & +1 & -1 & -1 \\
			+1 & -1 & +1 & -1 \\
			+1 & +1 & +1 & +1
		\end{bmatrix}
		\begin{bmatrix}
			b_0 + b_{ABC} \\
			b_A + b_{BC} \\
			b_B + b_{AC} \\
			b_{C} + b_{AB}
		\end{bmatrix}
		+
		\begin{bmatrix}
			e_1\\
			e_2\\
			e_3 \\
			e_4
		\end{bmatrix}
	\end{array}
	$
	
	\vspace{12pt}
	\begin{itemize}
		\item	{\color{purple}{Confounding pattern}}:
		\begin{itemize}
			\item	$\widehat{\beta}_A \rightarrow$ \textbf{A + BC}
			\item	$\widehat{\beta}_B \rightarrow$ \textbf{B + AC}
			\item	$\widehat{\beta}_C \rightarrow$ \textbf{C + AB}
			\item	$\widehat{\beta}_0 \rightarrow$ \textbf{I + ABC}
		\end{itemize}
	\end{itemize}
	
	\vspace{12pt}
	What does the confounding pattern tell us?
\end{frame}

\begin{frame}\frametitle{Half fractions}
	\begin{itemize}
		\item	Confounding pattern:
		\begin{itemize}
			\item	$\widehat{\beta}_A \rightarrow$ \textbf{A + BC}
			\item	$\widehat{\beta}_B \rightarrow$ \textbf{B + AC}
			\item	$\widehat{\beta}_C \rightarrow$ \textbf{C + AB} (intentional)
			\item	$\widehat{\beta}_0 \rightarrow$ \textbf{I + ABC}
		\end{itemize}
	\end{itemize}

	We say:
	\begin{itemize}
		\item	\textbf{A} is an \emph{alias} for \textbf{BC}
		\item	\textbf{B} is an \emph{alias} for \textbf{AC}
		\item	\textbf{C} is an \emph{alias} for \textbf{AB}
		\item	Design was generated by \textbf{C = AB}
	\end{itemize}
	
	\vspace{12pt}
	{\color{myOrange}{Homework}}: write out the half fraction design for a 4 factor experiment using the trade-off table (\emph{hint}: see next slide).
\end{frame}

\begin{frame}\frametitle{Generators/Defining relationship}
	\begin{itemize}
		\item	General rule for half-fractions of $2^k$ factorial:
		\begin{itemize}
			\item	write down $k-1$ main factors
			\item	\emph{generate} last factor from product of previous $k-1$ factors
		\end{itemize}
	\end{itemize}

	Example: $2^4$ with \textbf{A}, \textbf{B}, \textbf{C} and \textbf{D}
	\begin{itemize}
		\item	Generator: \textbf{D = ABC} (from the table)
		\item	\emph{There's only 1 generator}, according to the table
	\end{itemize}

	\vspace{12pt}
	Some ``aliasing notation'' rules applied to this example:
	\begin{itemize}
		\item	\textbf{A} $\times$ \textbf{A = I}
		\item	\textbf{B} $\times$ \textbf{B = I}
		\item	\textbf{I} $\times$ \textbf{I = I}
		\item	\textbf{ABC} $\times$ \textbf{D} = \textbf{ABC} $\times$ \textbf{ABC = AABBCC = I I I}, so we get\\ \textbf{I = ABCD}
		\item	We call this the defining relationship for this experiment.
	\end{itemize}
\end{frame}

\begin{frame}\frametitle{Generators/Defining relationship}
	\begin{itemize}
		\item	{\color{purple}{Defining relationship}} = product of all generator combinations, solved for \textbf{I} on the left side
		\item	There is only generator in this example, \textbf{D = ABC}
		\item	\textbf{I = ABCD} is called the \emph{defining relation} for the design
	\end{itemize}

	Aliasing: multiply effect with defining relationship. E.g.:
	\begin{itemize}
		\item	\textbf{A} $\times$ \textbf{I = A} $\times$ \textbf{ABCD = BCD}
		\begin{itemize}
			\item	Main effect \textbf{A} aliased with 3-factor interaction \textbf{BCD}
		\end{itemize}
		\item	\textbf{B} main effect is aliased with:
		\item	\textbf{D} main effect is aliased with:
		\item	\textbf{AB} two-factor interaction:
		\item	\textbf{AC} two-factor interaction:
		\item	\textbf{BCD} three-factor interaction:
	\end{itemize}
	{\color{myOrange}{Homework}}: repeat this now on the half-fraction generated by\\ \textbf{C = AB}.
\end{frame}

\begin{frame}\frametitle{Saturated designs}

	{\color{purple}{Saturated design}}: fewest number of runs as possible for a given number of factors
	\begin{itemize}
		\item	$2^{k-1}$ factorial: half-fraction
		\item	$2^{k-2}$ factorial: quarter-fraction
		\item	$2^{k-p}$ factorial in general
	\end{itemize}
	\begin{itemize}
		\item	Excellent for screening many factors
		\begin{itemize}
			\item	Fewest number of experiments, for a given number of factors
		\end{itemize}
	\end{itemize}
	\begin{exampleblock}{}
		With little/no prior knowledge about a process, always start your first experiments with a saturated, screening design.
	\end{exampleblock}
	{\color{myOrange}{This gives you the greatest value for your experimental budget.}}
\end{frame}

\begin{frame}\frametitle{Saturated designs}
	\begin{itemize}
		\item	Found by moving up the columns in the tradeoff table:
	\end{itemize}
	\begin{center}
		\includegraphics[height=0.85\textheight]{\imagedir/doe/DOE-trade-off-table.png}
	\end{center}
\end{frame}

\begin{frame}\frametitle{Saturated design example}
	\begin{itemize}
		\item	Seven factors: \textbf{A, B, C, D, E, F, G }. A $2^{7-p}$ fraction
		\begin{itemize}
			\item	p=0: 128 runs, 128 parameters estimated
			\item	p=1: 64 runs (half-fraction), 64 parameters
			\item	p=2: 32 runs
			\item	p=3: 16 runs
			\item	p=4: 8 runs: parameters are 7 factors + intercept
		\end{itemize}
	\end{itemize}

	How to design this experiment?
\end{frame}

\begin{frame}\frametitle{Saturated design example}

	1. 8 runs: $2^3$ factorial for factors \textbf{A, B} and \textbf{C}
	\begin{center}
		\includegraphics[width=0.8\textwidth]{\imagedir/doe/saturated-design-base-factorial.png}
	\end{center}
\end{frame}

\begin{frame}\frametitle{Saturated design example}

	2. What about factors \textbf{D, E, F} and \textbf{G}?

	Create these by intentionally confounding with the interactions derived from the \textbf{A}, \textbf{B}, and \textbf{C} factors.

	Generators are:
	\begin{itemize}
		\item	\textbf{D=AB}
		\item	\textbf{E=AC}
		\item	\textbf{F=BC}
		\item	\textbf{G=ABC}
	\end{itemize}
	\begin{center}
		\includegraphics[width=\textwidth]{\imagedir/doe/saturated-design-base-factorial-expanded.png}
	\end{center}
\end{frame}

\begin{frame}\frametitle{Saturated design example}

	3. Determine aliasing. Require defining relationship first.

	General rule for $2^{k-p}$ factorials:
	\begin{itemize}
		\item	produced by $p$ generator
		\item	defining relationship has $2^p$ words
	\end{itemize}

	Defining relationship: all combinations of the generators.
\end{frame}

\begin{frame}\frametitle{Saturated design example}

	3. This case: $2^4 = 16$ words in our defining relation
	\begin{itemize}
		\item	\textbf{I}
		\item	Generators: \textbf{I = ABD = ACE = BCF = ABCG}
		\item	Combine 2: \textbf{I = BDCE = ACDF = CDG = ABEF = BEG = AFG}
		\item	Combine 3: \textbf{I = DEF = ADEG = CEFG = BDFG}
		\item	Combine 4: \textbf{I = ABCDEFG}
	\end{itemize}

	\textbf{I = ABD = ACE = BCF = ABCG}

	\textbf{ = BDCE = ACDF = CDG = ABEF = BEG = AFG}

	\textbf{ = DEF = ADEG = CEFG = BDFG }

	\textbf{ = ABCDEFG}

	Shortest word: 3 factors. Called a resolution III design.
\end{frame}

\begin{frame}\frametitle{Saturated design example}

	4. Confounding pattern for main effects. For \textbf{A}:

	\textbf{AI = BD = CE = ABCF = BCG = ABDCE = CDF = ACDG = BEF = ABEG = FG = ADEF = DEG = ACEFG = ABDFG = BCDEFG}

	We usually only report the main effects and 2fi confounding:
	\begin{itemize}
		\item	$\widehat{\beta}_0$ = ABCDEFG
		\item	$\widehat{\beta}_{\mathbf{A}} \rightarrow$ A + BD + CE + FG
		\item	$\widehat{\beta}_{\mathbf{B}} \rightarrow$ B + AD + CF + EG
		\item	$\widehat{\beta}_{\mathbf{C}} \rightarrow$ C + AE + BF + DG
		\item	$\widehat{\beta}_{\mathbf{D}} \rightarrow$ \textbf{D + AB} + CG + EF
		\item	$\widehat{\beta}_{\mathbf{E}} \rightarrow$ \textbf{E + AC} + BG + DF
		\item	$\widehat{\beta}_{\mathbf{F}} \rightarrow$ \textbf{F + BC} + AG + DE
		\item	$\widehat{\beta}_{\mathbf{G}} \rightarrow$ G + CD + BE + AF
	\end{itemize}
\end{frame}

\begin{frame}\frametitle{Saturated design example}

	5. Choose different confounding if this is not suitable. More simply: just reassign the letters to different factors.

	6. Collect $y$ values from experiment
	\begin{itemize}
		\item	X-matrix: as shown in the table
		\item	Calculate: $\mathrm{b} = \left(\mathrm{X}^T\mathrm{X}\right)^{-1}\mathrm{X}\mathrm{y}$
		\item	$\mathrm{b} = [b_0, b_\mathbf{A}, b_\mathbf{B}, b_\mathbf{C}, b_\mathbf{D}, b_\mathbf{E}, b_\mathbf{F}, b_\mathbf{G}]$
		\item	$\mathrm{X}^T\mathrm{X}$ matrix is diagonal
	\end{itemize}
\end{frame}

\begin{frame}\frametitle{Saturated design example}

	7. Which main effects are important? Use Pareto-plot of effect.
	\begin{center}
		\includegraphics[width=0.7\textwidth]{\imagedir/doe/pareto-plot.png}
	\end{center}
\end{frame}

\begin{frame}\frametitle{Highly fractionated designs}
	\begin{itemize}
		\item	$2^k$ factorials: too many runs for large $k$
		\item	Rather use $2^{k-p}$ fractional factorial
		\item	Budget for 8 experiments:
		\begin{itemize}
			\item	$2^3$ full factorial with \textbf{3 factors}
			\item	$2^{4-1}$ half fraction in \textbf{4 factors}
			\item	$2^{5-2}$ quarter fraction in \textbf{5 factors}
			\item	$2^{6-3}$ fractional factorial in \textbf{6 factors}
			\item	$2^{7-4}$ fractional factorial in \textbf{7 factors}.
		\end{itemize}
	\end{itemize}

	\textbf{What are the trade-offs?}
\end{frame}

\begin{frame}\frametitle{``Generators" and ``Defining relationships" summary}

	For a $2^{k-p}$ factorial with $k$ factors:
	\begin{block}{Purpose of generators?}
		\begin{center}
			To tell how to create (generate) the fractional factorial
		\end{center}
	\end{block}
	\begin{itemize}
		\item	Generators are found from the trade-off table
	\end{itemize}
	\begin{block}{Purpose of the defining relationship?}
		\begin{center}
			To find, in advance, which factors are aliased with each other
		\end{center}
	\end{block}
	\begin{itemize}
		\item	Allows us to adjust our experiment \emph{before} we start it.
		\item	In a $2^{k-p}$ design, there are $2^p$ words in the defining relationship
		\item	Multiply each factor with the defining relationship to see aliasing
	\end{itemize}
\end{frame}

\begin{frame}\frametitle{At-home example 1}
	\begin{enumerate}
		\item	7 factors; smallest number of runs?
		\item	What are generators and defining relationship for part 1?
		\item	Confounding pattern for main effects?
	\end{enumerate}
\end{frame}

\begin{frame}\frametitle{At-home example 2}
	\begin{enumerate}
		\item	8 factors: smallest number of runs?
		\item	What are generators and defining relationship for part 1?
		\item	Confounding pattern for main effects?
		\begin{itemize}
			\item	\emph{Answer}: main effects confounded with 3-factor interactions
		\end{itemize}
	\end{enumerate}
\end{frame}

\begin{frame}\frametitle{Notes for fractionated designs}
	\begin{enumerate}
		\item	Can use different choices for generators
		\item	Pick a choice, then just shuffle assignment of letters to actual factors:
		\begin{itemize}
			\item	\textbf{A} confounded with \textbf{BCD}
			\item	\textbf{B} confounded with \textbf{CA}
			\item	\textbf{C} \emph{etc}
		\end{itemize}
	\end{enumerate}

	A $2^{k-p}$ factorial:
	\begin{itemize}
		\item	is produced by $p$ generators
		\item	defining relation has $2^p$ words
	\end{itemize}
\end{frame}

\begin{frame}\frametitle{Design resolution}
	\begin{block}{What is ``{\color{purple}{Resolution}}'' ?}
		\begin{center}
			The length of the shortest word in the defining relation.
		\end{center}
	\end{block}
	\begin{itemize}
		\item	$2^{7-4}$ example: shortest word = 3 letters: $2^{7-4}_\text{III}$ design
		\begin{itemize}
			\item	main effects confounded with 2-factor interactions
		\end{itemize}
		\item	$2^{8-5}$ example: shortest word = 4 letter: $2^{8-5}_\text{IV}$ design
		\begin{itemize}
			\item	main effects confounded with 3-factor interactions
		\end{itemize}
		\item	$2^{5-1}$ half-fraction design
		\begin{itemize}
			\item	four factors as standard factorial
			\item	factor \textbf{E = ABCD}, so \textbf{I = ABCDE}
			\item	it is a $2^{5-1}_{\text{V}}$ design
		\end{itemize}
	\end{itemize}
\end{frame}

\begin{frame}\frametitle{Design resolution}

	Resolution: \emph{shows how clearly effects are separated}
	\begin{itemize}
		\item	Let main effects = 1
		\item	Two-factor interactions = 2
		\item	Three-factor interactions = 3
	\end{itemize}

	\textbf{Resolution V designs}
	\begin{itemize}
		\item	$5 - 1 = 4$: main effects confounded with 4-factor interactions (fi)
		\item	$5 - 2 = 3$: 2-fi confounded with 3-fi
		\item	$5 - 3 = 2$: 3-fi confounded with 2-fi
	\end{itemize}

	Aim for a higher resolution, but accept a lower resolution initially, in order to test more factors
\end{frame}

\begin{frame}\frametitle{Design resolution}

	\textbf{Resolution III designs}
	\begin{itemize}
		\item	Excellent for initial screening
		\item	Confounding?
	\end{itemize}

	\textbf{Resolution IV designs}
	\begin{itemize}
		\item	Learning about and understanding a system (characterization)
		\item	Confounding?
	\end{itemize}

	\textbf{Resolution V designs and full factorial designs}
	\begin{itemize}
		\item	For optimizing a process, understanding complex effects,
		\item	To develop high-accuracy models
	\end{itemize}
\end{frame}

\begin{frame}\frametitle{Design resolution}
	\begin{center}
		\includegraphics[height=0.85\textheight]{\imagedir/doe/DOE-trade-off-table.png}
	\end{center}
	\see{BHH2, p272. Bring copy of this table to exams.}
\end{frame}

\begin{frame}\frametitle{Saturated designs - screening}
	\begin{itemize}
		\item	Resolution III design
		\item	Screen many factors
		\item	Good for evaluating a new system
		\begin{itemize}
			\item	Lab-scale work
			\item	New product development
			\item	Transfer from the lab to plant scale
		\end{itemize}
	\end{itemize}
\end{frame}

\begin{frame}\frametitle{Saturated design example}
	\begin{center}
		\includegraphics[width=\textwidth]{\imagedir/doe/saturated-experiment-in-7-factors.png}
	\end{center}
	\begin{enumerate}
		\item	Create $\mathbf{X}$ and $\mathbf{y}$ matrices from the table
		\item	Solve for $\mathbf{b}$
		\item	Plot Pareto plot of main effects (they are highly confounded)
	\end{enumerate}
\end{frame}

\begin{frame}\frametitle{Saturated design example}
	\begin{columns}
		\column{4cm} 
			\begin{center}
				\includegraphics[width=1.3\textwidth]{\imagedir/doe/pareto-plot.png} 
			\end{center}
		\column{6cm} 
			\begin{itemize}
				\item	\textbf{A}, \textbf{C} and \textbf{G} 
				\item	\textbf{E}: fairly small 
				\begin{itemize}
					\item	$\widehat{\beta}_{\mathbf{E}} \rightarrow \textbf{E + AC} + BG + DF$ 
					\item	could be due to main effect \textbf{E} 
					\item	or due to \textbf{AC} interaction 
				\end{itemize}
				\item	Factors \textbf{B}, \textbf{D} and \textbf{F} are not important.
			\end{itemize}
	\end{columns}
	\vspace{12pt}
	Next experiments: focus on \textbf{A}, \textbf{C}, \textbf{G} and their interactions.
\end{frame}

\begin{frame}\frametitle{Saturated designs: note}
	\begin{itemize}
		\item	Fraction factorials: $2^{k-p}$ runs
		\begin{itemize}
			\item	for integers $k$ and $p$: $4, 8, 16, 32, 64, 128, \ldots$
		\end{itemize}
		\item	Plackett and Burman designs are for screening also:
		\begin{itemize}
			\item	multiples of 4: $12, 16, 20, 24, 28, \ldots$ runs
		\end{itemize}
		\item	Box and Bisgaard paper: ``What can you find out from 12 experimental runs?''
	\end{itemize}
\end{frame}

\begin{frame}\frametitle{Foldover: de-aliasing}
	\begin{itemize}
		\item	Experiments are not a one-shot operation; always run sequential expts
		\item	Fractional factorial:
		\begin{itemize}
			\item	highly confounded; but say one factor \textbf{C} is important
			\item	switch the sign of \textbf{C}: from \textbf{C} to \textbf{-C}
			\item	repeat the fractional experiments
			\item	it unconfounds \textbf{C}: main effect and all its 2-fi
		\end{itemize}
	\end{itemize}

	Switching the sign of a factor will de-alias its main effect and all its associated two-factor interactions.
\end{frame}

\begin{frame}\frametitle{Foldover: removing 2-fi confounding}
	\begin{itemize}
		\item	Run another fraction, but switch all the signs in the design table
		\begin{itemize}
			\item	i.e. let \textbf{A = -A}, \textbf{B = -B}, \emph{etc}
			\item	Run another fractional factorial
			\item	All 2-fi will be removed from the main effects
		\end{itemize}
	\end{itemize}
\end{frame}

\begin{frame}\frametitle{Projectivity}

	Fractional factorials collapse to full factorials when effects are insignificant.
	\begin{center}
		\includegraphics[width=0.60\textwidth]{\imagedir/doe/projectivity-of-a-half-fraction-in-3-factors.png}
	\end{center}
	Projectivity = $P$ = resolution - 1
\end{frame}

\begin{frame}\frametitle{Example}

	You are developing a new product, but struggling to get product stability (measured in days), to the required level. Aim for stability above 50 days. Four factors considered:
	\begin{itemize}
		\item	\textbf{A} = monomer concentration: 30\% or 50\%
		\item	\textbf{B} = acid concentration: low or high
		\item	\textbf{C} = catalyst level: 2\% or 3\%
		\item	\textbf{D} = temperature: 393K or 423K
	\end{itemize}

	Experiments in standard order:
	\begin{center}
		\includegraphics[width=0.95\textwidth]{\imagedir/doe/experiments-for-stability.png}
	\end{center}
\end{frame}

\begin{frame}\frametitle{Example (continued)}
	\begin{enumerate}
		\item	How was the experimented generated?
		\item	What is the defining relationship?
		\item	What will be aliased with \textbf{A}; with \textbf{D} and with \textbf{BC}?
		\item	Describe the aliasing structure (resolution)?
		\item	What is the model's intercept; main effect for \textbf{A}; and for the \textbf{AD} interaction?
	\end{enumerate}
\end{frame}

\begin{frame}\frametitle{Example (continued)}

	If the least squares model is:

	$ y = 29.5 -5.75x_A -3.75 x_B -1.25 x_C + 0.75 x_D + 0.50 x_A x_B + 1.0 x_A x_C - 1.0 x_A x_D$

	what is the predicted stability when operating at:
	\begin{itemize}
		\item	monomer concentration of 25\%
		\item	low acid concentration
		\item	1.5\% catalyst level
		\item	a temperature of 408 K
	\end{itemize}
\end{frame}

\begin{frame}\frametitle{Response surface methods}

	\textbf{Objective}: achieve the best response using sequential experimentation.
	\begin{center}
		\includegraphics[width=0.7\textwidth]{\imagedir/doe/COST-contours.png}
	\end{center}
	Wasn't the COST approach also sequential experimentation?

	\textbf{Different to COST}: We are changing multiple variables at a time!
\end{frame}

\begin{frame}\frametitle{Single-variable case}
	\begin{center}
		\includegraphics[width=\textwidth]{\imagedir/doe/steepest-ascent-univariately-corrected.png}
	\end{center}
	\begin{itemize}
		\item	Can get to optimum faster if use quadratic (or spline) approximations
	\end{itemize}
\end{frame}

\begin{frame}\frametitle{Single-variable case}

	This is the COST approach:
	\begin{itemize}
		\item	exploratory steps of $\gamma_i$ towards an optimum
		\item	refit the model once we plateau
		\item	repeat
	\end{itemize}
\end{frame}

\begin{frame}\frametitle{Analogy}
	\begin{itemize}
		\item	Analogy for finding the optimum:
	\end{itemize}
	\begin{center}
		\includegraphics[height=0.85\textheight]{\imagedir/doe/pictograms-nps-accessibility-low-vision-access.png} %3626157564_ba129e810a_b.jpg}
	\end{center}
\end{frame}

\begin{frame}\frametitle{2-variable example}
	\begin{columns}
		\column{6cm}
			\begin{center}
				\includegraphics[height=0.95\textheight]{\imagedir/doe/RSM-base-case-with-first-factorial.png}
			\end{center}
		\column{4cm}
			\begin{itemize}
				\item	Baseline:
				\begin{itemize}
					\item	$T$=325K
					\item	$S$ = 0.75g/L
					\item	profit = \$407.
				\end{itemize}
				\item	Example worked out on the board
			\end{itemize}
	\end{columns}
\end{frame}

\begin{frame}\frametitle{2-variable example}
	\begin{columns}
		\column{6cm}
			\begin{center}
				\includegraphics[height=0.95\textheight]{\imagedir/doe/RSM-base-case-with-exploration.png}
			\end{center}
		\column{4cm}
	\end{columns}
\end{frame}

\begin{frame}\frametitle{2-variable example}
	\begin{columns}
		\column{6cm}
			\begin{center}
				\includegraphics[height=0.95\textheight]{\imagedir/doe/RSM-base-case-with-extra-factorial.png}
			\end{center}
	 	\column{4cm}
	\end{columns}
\end{frame}

\begin{frame}\frametitle{2-variable example}
	\begin{columns}
		\column{6cm}
			\begin{center}
				\includegraphics[height=0.95\textheight]{\imagedir/doe/RSM-base-case-with-CCD-factorial.png} 
			\end{center}
		\column{4cm}
	\end{columns}
\end{frame}

\begin{frame}\frametitle{2-variable example}

	Adding second order effects; use a central composite design.
	\begin{itemize}
		\item	CCD design: full factorial + axial points + center points
	\end{itemize}
	\begin{center}
		\includegraphics[width=\textwidth]{\imagedir/doe/central-composite-design.png}
	\end{center}
\end{frame}

\begin{frame}\frametitle{2-variable example}

	Add quadratic terms to model:

	$\hat{y} = b_0 + b_Tx_T + b_S x_S + b_{TS} x_T x_S + b_{TT} x_T^2 + b_{SS} x_S^2$

	$
	\begin{array}{rcl}
		\\
		\\
		\mathrm{y} &=& \mathrm{X} \mathrm{b} + \mathrm{e}\\
		\\
		\begin{bmatrix}
			y_8\\
			y_9\\
			y_{10} \\
			y_{11} \\
			y_{6} \\
			y_{13} \\
			y_{14} \\
			y_{15} \\
			y_{16}
		\end{bmatrix}
		&=&
		\begin{bmatrix}
			1 & -1 & -1 & +1 & +1 & +1\\
			1 & +1 & -1 & -1 & +1 & +1\\
			1 & -1 & +1 & -1 & +1 & +1\\
			1 & +1 & +1 & +1 & +1 & +1\\
			1 & 0 & 0 & 0 & 0 & 0\\
			1 & 0 &-1.41& 0 & 0 & 2\\
			1 & 1.41& 0& 0 & 2 & 0\\
			1 & 0 & 1.41& 0 & 0 & 2\\
			1 &-1.41& 0& 0 & 2 & 0
		\end{bmatrix}
		\begin{bmatrix}
			b_0 \\
			b_T \\
			b_S \\
			b_{TS} \\
			b_{TT} \\
			b_{SS}
		\end{bmatrix}
		+ \mathrm{e}
	\end{array}
	$
\end{frame}

\begin{frame}\frametitle{2-variable example}
	\begin{columns}
		\column{6cm}
			\begin{center}
				\includegraphics[height=0.95\textheight]{\imagedir/doe/RSM-base-case-with-CCD-contours.png} 
			\end{center}
		\column{4cm}
	\end{columns}
\end{frame}

\begin{frame}\frametitle{General approach for RSM}

	1. Start at baseline; run full or fractional factorial
	\begin{itemize}
		\item	$ \hat{y} = b_0 + b_1x_1 + b_2 x_2 + \ldots + b_{12}x_1x_2 + b_{13} x_1 x_3 + \ldots$
	\end{itemize}
	2. Main effects usually greater than 2-factor interaction

	3. Estimate path of steepest ascent (or descent):
	\begin{itemize}
		\item	$\dfrac{\partial \hat{y}}{\partial x_1} = b_1 \qquad\qquad \dfrac{\partial \hat{y}}{\partial x_2} = b_2 \qquad \ldots$
		\item	Move $b_1$ units in $x_1$ \textbf{and} move $b_2$ units in $x_2$, \emph{etc}
		\item	These are coded units. Unscale to real-world units!
		\item	Implement a portion of the full step, e.g. only 25\%
	\end{itemize}

	4. Make several sequential steps until response levels off
	\begin{itemize}
		\item	$y_6 = 600; y_7 = 800, y_8 = 825, y_9 = 750$
	\end{itemize}
\end{frame}

\begin{frame}\frametitle{General approach for RSM}

	5. Use a new factorial
	\begin{itemize}
		\item	perhaps add other factors
		\item	flip signs on binary factors
	\end{itemize}
	6. Repeat steps 1 to 5, until linear model is insufficient
	\begin{itemize}
		\item	Curvature shows up
		\item	2-factor interactions dominate main effects
	\end{itemize}
	7. Estimate a quadratic model
	\begin{itemize}
		\item	Use central composite design; use 3-levels per factor
		\item	Add quadratic terms to model, e.g. $\ldots + b_{11}x_1^2 + b_{22}x_2^2 + \ldots$
	\end{itemize}
	8. Draw contour plots (surfaces) and move to next optimum
\end{frame}

\begin{frame}\frametitle{What is the response variable}
	\begin{itemize}
		\item	Single $y$ is not always feasible
		\item	Use y = ``total costs'', or y = ``net profit''
		\item	Superimpose contour surfaces
	\end{itemize}
	\begin{center}
		\includegraphics[width=0.65\textwidth]{\imagedir/doe/multi-objective-RSM.png}
	\end{center}
	Page 579 of the Hill and Hunter review article - reference in the notes.
\end{frame}


\begin{frame}\frametitle{Evolutionary operation}
	\begin{itemize}
		\item	Similar concept to RSM
		\item	Processes are not constant, the optimum is shifting
		\begin{itemize}
			\item	heat-exchanger fouling
			\item	build-up inside reactors and tubing
			\item	catalyst deactivation
			\item	slowly varying disturbances
		\end{itemize}
	\end{itemize}
	\begin{itemize}
		\item	Iterative hunt for the process optimum:
		\begin{itemize}
			\item	make small perturbations within daily production
			\item	use replicate runs and average
			\item	move along the response surface
		\end{itemize}
	\end{itemize}
\end{frame}

\begin{frame}\frametitle{General approach for experimentation}
	\begin{itemize}
		\item	Box: ``\emph{The best time to run an experiment is after the experiment}''
		\item	Box: ``\emph{Do not spend more than 20\% to 25\% of your time and budget on your first group of experiments}''
	\end{itemize}

	\textbf{Phase 1}: screening runs

	\textbf{Phase 2}: sequential experiments to augment screening

	\textbf{Phase 3}: optimizing; RSM and full factorials

	\textbf{Phase 4}: maintain the optimum, search for better optima
\end{frame}

\begin{frame}\frametitle{Mistakes, missing values, and constraints}
	\begin{itemize}
		\item	Do not or cannot reach $-1$ or $+1$:
		\begin{itemize}
			\item	use a least squares model with the coded value actually used in expt
			\item	loose some orthogonality
		\end{itemize}
	\end{itemize}
	\begin{itemize}
		\item	Missing values
		\begin{itemize}
			\item	main effects estimate multiple times
			\item	solve underdetermined system, or
			\item	drop out insignificant terms (iterate)
		\end{itemize}
	\end{itemize}
\end{frame}

\begin{frame}\frametitle{Handling of constraints}
	\begin{center}
		\includegraphics[width=0.9\textwidth]{\imagedir/doe/two-factors-with-constraint.png}
	\end{center}
	\begin{itemize}
		\item	New runs are not independent; lost orthogonality
		\item	Use ``optimal'' design to locate experiments (next)
	\end{itemize}
\end{frame}

\begin{frame}\frametitle{Optimal designs}
	\begin{itemize}
		\item	What is sub-optimal about our existing designs? \textbf{Nothing!}
		\item	Use optimal designs when:
		\begin{itemize}
			\item	constraints are complex (plane constraints; $k \geq 3$)
			\item	estimating a non-standard model
			\item	running a reduced number of experiments
			\item	have more than 2-levels per factor
			\item	want to add experiments to existing runs
		\end{itemize}
	\end{itemize}
\end{frame}

\begin{frame}\frametitle{Optimal designs}

	Computer-based approach:
	\begin{enumerate}
		\item	User specifies the model (i.e. the parameters)
		\item	Computer finds all possible combinations (grid approach)
		\begin{itemize}
			\item	user can augment this list, called candidate points
			\item	center-points added
		\end{itemize}
		\item	User specifies number of experiments
		\item	Computer iteratively selects the ``optimal'' set
	\end{enumerate}

	Optimality criteria:
	\begin{itemize}
		\item	A-optimal: minimizes $\text{trace}\left\{(\mathbf{X}^T\mathbf{X})^{-1}\right\}$
		\item	D-optimal: maximizes $\text{det}(\mathbf{X}^T\mathbf{X})$
		\item	G-optimal: minimize maximum variance of $\hat{y}$
		\item	V-optimal: minimize average variance of $\hat{y}$
	\end{itemize}
\end{frame}

\begin{frame}\frametitle{Optimal designs}
	\begin{itemize}
		\item	A full factorial design, $2^k$ is already A-, D- G- and V-optimal.
		\item	D-optimal designs work well; used most often.
	\end{itemize}
\end{frame}

\begin{frame}\frametitle{Mixture designs}
	\begin{itemize}
		\item	Fine chemicals, pharmaceuticals, food manufacturing, and polymer processing
		\item	There are screening and optimization designs for mixtures also
		\item	Constraint for mixtures: $\sum_i x_i = 1$
		\item	Cannot be changed independently
	\end{itemize}
	\begin{center}
		\includegraphics[width=\textwidth]{\imagedir/doe/mixture-design.png}
	\end{center}
\end{frame}
