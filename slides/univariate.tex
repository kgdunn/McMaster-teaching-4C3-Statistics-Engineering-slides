\begin{frame}\frametitle{Univariate data analysis: What we will cover}
	\includegraphics[width=\textwidth]{\imagedir/mindmaps/univariate-section-mapping.png}
\end{frame}

\begin{frame}\frametitle{Usage examples for this section}
	\begin{itemize}
		\item	\emph{Co-worker}: Here are the yields from a batch system for the last 3 years (1256 data points)
		\begin{itemize}
			\item	what sort of distribution do the data have?
			\item	yesterday's yield was less than 160 g/L, what are the chances of that?
		\end{itemize}
		\item	\emph{Yourself}: Using historical failure rate data of the pumps, what is the probability that 3 pumps will fail this month?
		\item	\emph{Manager}: does reactor 1 have better final product purity, on average, than reactor 2?
		\item	\emph{Colleague}: what is the 95\% confidence interval for the density of your powder ingredient?
	\end{itemize}
\end{frame}

\begin{frame}\frametitle{References and readings}
	\begin{itemize}
		\item	Any standard statistics text book
		\item	\textbf{Recommended}: Box, Hunter and Hunter, \emph{Statistics for Experimenters}, Chapter 2 (both 1st and 2nd edition)
		\item	\textbf{Wikipedia articles} on any term that seems unfamiliar or that you've forgotten

		\vspace{2cm}
		\item	Hodges and Lehmann, \emph{Basic Concepts of Probability and Statistics}
		\item	Hogg and Ledolter, \emph{Engineering Statistics}
		\item	Montgomery and Runger, \emph{Applied Statistics and Probability for Engineers}
	\end{itemize}
\end{frame}

\begin{frame}\frametitle{Concepts: use this as a checklist at the end of this section}
	\begin{center}
		\includegraphics[height=0.9\textheight]{\imagedir/mindmaps/univariate-section-concepts.png}
	\end{center}
\end{frame}

\begin{frame}\frametitle{Variability: Life is pretty boring without variability}

	\begin{center}
		\includegraphics[width=\textwidth]{\imagedir/concepts/variation/variation-none.png}
	\end{center}
	(and this course would be unnecessary)
\end{frame}

\begin{frame}\frametitle{We have plenty of variability in our recorded data}

	\begin{center}
		\includegraphics[width=\textwidth]{\imagedir/concepts/variation/variation-some.png}
	\end{center}
	\begin{itemize}
		\item	Other unaccounted for sources, often called \textbf{\emph{error}}
	\end{itemize}
\end{frame}

\begin{frame}\frametitle{Variability is created from many sources, sometimes intentionally!}
	\begin{itemize}
		\item	Feedback control: introduces variability
		\item	Operating staff: introduce variability into a process
		\item	Sensor drift, spikes, noise, recalibration shifts, errors in our sample analysis
	\end{itemize}

	\includegraphics[width=\textwidth]{\imagedir/concepts/variation/variation-more.png}
\end{frame}

\begin{frame}\frametitle{Variability due to unintentional circumstances}
	\begin{itemize}
		\item	Raw material properties are not constant
		\item	Production disturbances:
		\begin{itemize}
			\item	external conditions change (ambient temperature, humidity)
			\item	equipment breaks down, wears out, maintenance shut downs
		\end{itemize}
	\end{itemize}

	\includegraphics[width=\textwidth]{\imagedir/concepts/variation/variation-spikes.png}

	All this variability keep us process engineers employed, but it comes at a price.
\end{frame}

\begin{frame}\frametitle{The high cost of variability in your final product}
	\begin{block}
		{Assertion}
		\begin{center}
			Customers expect both uniformity and low cost when they buy your product. Variability defeats both objectives.
		\end{center}
	\end{block}
	\begin{enumerate}
		\item	Customer totally unable to use your product. Examples:
		\begin{itemize}
			\item	fresh milk,
			\item	viscosity too high,
			\item	oil that causes pump failure
		\end{itemize}
		\item	Your product leads to poor performance. Example:
		\begin{itemize}
			\item	customer must put in more energy than usual (melting point too high)
			\item	longer reaction times for off-spec catalyst
		\end{itemize}
		\item	Your brand is diminished
		\begin{itemize}
			\item	Extreme example: Maple Leaf Foods (listeriosis outbreak in 2008)
			\item	Toyota quality issues with accelerators, early 2010
		\end{itemize}
	\end{enumerate}
\end{frame}

\begin{frame}\frametitle{The high cost of variability in your final product}
	Variability also has these costs:
	
	\vspace{12pt}
	\begin{enumerate}
		\item	Inspection costs:
		\begin{itemize}
			\item	too expensive and inefficient to test every product
			\item	low variability means you don't need to inspect every product
		\end{itemize}
		\item	Off-specification products cost you and customer money:
		\begin{itemize}
			\item	reworked
			\item	disposed
			\item	sold at a loss
		\end{itemize}
	\end{enumerate}
\end{frame}

\begin{frame}\frametitle{The high cost of variability in your raw materials}
	\begin{itemize}
		\item	Flip it around: you receive highly variable raw materials:
		\begin{itemize}
			\item	That variability lands up in your product, or
			\item	you incur additional cost (energy/time/materials) to process it
		\end{itemize}
	\end{itemize}
	\begin{center}
		\includegraphics[width=0.8\textwidth]{\imagedir/univariate/feedback-control-variance-reduction.svg.png}
	\end{center}
\end{frame}

\begin{frame}\frametitle{This entire course is about variability}
	\begin{enumerate}
		\item	Visualizing: \emph{show} the variability
		\item	This section: \emph{quantify} variability, then \emph{compare} variability
		\item	Section 3: how variation in one variable \emph{affects} another (least squares)
		\item	Section 4: we intentionally \emph{introduce} variation to learn more about the process (DOE)
		\item	Section 5: construct monitoring charts to \emph{track} variability
		\item	Section 6: dealing with multiple variables, simultaneously extracting information (latent variables)
	\end{enumerate}
\end{frame}

\begin{frame}\frametitle{Histograms summarize variation in a measured variable}

	Shows \emph{number} of samples that occur in a \emph{\textbf{category}}: called a {\color{purple}{\textbf{frequency distribution}}}
	\begin{center}
		\includegraphics[width=0.6\textwidth]{\imagedir/univariate/histogram-children-by-gender.png}
	\end{center}
\end{frame}

\begin{frame}\frametitle{Histograms for a \textbf{continuous} variable}
	Continuous variables: create category bins (usually equal-size)
	\begin{center}
		\includegraphics[width=0.8\textwidth]{\imagedir/univariate/histogram-package-mass.png}
	\end{center}
	\begin{itemize}
		\item	Overfilling to achieve the 1 kg spec. If we can reduce variation: \emph{boss gives us a raise!}
	\end{itemize}
\end{frame}

\begin{frame}\frametitle{Histograms}
	\textbf{In-class exercise}: plot your expectation of these histograms: \vspace{2cm}
	
	\begin{columns}
		\column{0.8\textwidth}
			\begin{itemize}
				\item	numbers thrown from a 6-sided dice
				\item	repeated lab measurement to measure \\$g$ (9.81 $m/s^2$)
				\item	this class's grades for a really easy test
				\item	annual income for people in Canada
				\item	bacterial count [number per cubic inch] from deli meat samples
			\end{itemize}
		\column{0.3\textwidth}
			\includegraphics[width=0.9\textwidth]{\imagedir/univariate/6sided_dice-Wikipedia.png}
	\end{columns}
\end{frame}

\begin{frame}\frametitle{Histograms are all about inferring long-term probabilities}
	You implicitly used a ``long'' time scale for your prior histograms

	\vspace{12pt}
	\textbf{Other examples:}
	\begin{itemize}
		\item	Next throw on a die? A fair die has a 16.67\% chance of showing each number.
		\item	What is the batch yield tomorrow? Long term average $160 g/L \pm 20 g/L$
		\item	Boy or a girl? Canada's sex ratio at birth 1.06:1 (boy:girl)
		\item	Age at death? Canadian life tables: e.g. in 2002, for females:
		\begin{itemize}
			\item	have a 98.86\% chance of reaching age 30
			\item	have a 77.5\% chance of reaching age 75
		\end{itemize}
	\end{itemize}
\end{frame}

\begin{frame}\frametitle{Frequency distribution: you can also use a relative frequency}
	\begin{center}
		\includegraphics[width=0.8\textwidth]{\imagedir/univariate/frequency-histogram.png}
	\end{center}
	{\color{purple}{\textbf{Relative frequency}}}
	\begin{itemize}
		\item	does not require knowing $N$
		\item	can be compared to other distributions (e.g. \emph{side-by-side})
		\item	for large $N$ it quickly resembles the population's distribution
		\item	area under histogram is equal to 1 (related to probability)
	\end{itemize}
\end{frame}

\begin{frame}\frametitle{Frequency distribution}

	See the printed notes for detailed steps to construct a frequency distribution.
	\begin{itemize}
		\item	Categorical variable
		\item	Continuous variable: with and without quantization
	\end{itemize}

	Main steps are:
	\begin{enumerate}
		\item	Decide what you are measuring
		\item	Resolution for the measurement ($x$) axis
		\item	Find number of observations in each bin
		\item	Plot it as a bar plot.
	\end{enumerate}

	If you divide bin count by $N$, then you are plotting the \emph{relative frequency}.
\end{frame}

\begin{frame}\frametitle{Review of nomenclature you are already comfortable with}
	{\color{purple}{\textbf{Population}}}

	Large collection of \emph{potential} measurements (not necessary to be infinite, a large $N$ works well)

	{\color{purple}{\textbf{Sample}}}

	Collection of observations that have actually occurred ($n$ samples)

	In engineering: usually have plenty of data; that often is our population.

	\begin{center}
		\includegraphics[width=\textwidth]{\imagedir/univariate/batch-yields.png}
	\end{center}
\end{frame}

\begin{frame}\frametitle{Quick review of basic nomenclature}
	{\color{purple}{\textbf{Probability}}}

	Area under relative frequency distribution curve equals 1.

	Probability is the fraction of the area under the curve.

	For the previous histograms: mark the area:
	\begin{itemize}
		\item	Probability of grades less than 80\%
		\item	Probability that number on dice is 1 or 2
		\item	Bacterial count per cubic inch is above 10,000
	\end{itemize}
\end{frame}

\begin{frame}\frametitle{Quick review of basic nomenclature}
	{\color{purple}{\textbf{Parameter}}}
	\begin{itemize}
		\item	Value that describes the population's \emph{distribution} (fixed)
	\end{itemize}

	{\color{purple}{\textbf{Statistic}}}
	\begin{itemize}
		\item	An estimate of one of the population's parameters (estimate implies that it varies)
	\end{itemize}

	{\color{purple}{\textbf{Mean}}}
	\begin{itemize}
		\item	Measure of location (position)
	\end{itemize}

	$
	\begin{array}{rcl}
		\text{Population mean:} \qquad \mathcal{E}\left\{x \right\} = &\mu &= \displaystyle \frac{1}{N}\sum{x} \\
		\\
		\text{Sample mean:} \qquad &\bar{x} &= \displaystyle \frac{1}{n}\sum_{i=1}^{n}{x_i}
	\end{array}
	$
	\begin{itemize}
		\item	Mean by itself is not sufficient to describe a (random) variable
	\end{itemize}
\end{frame}

\begin{frame}\frametitle{Quick review of basic nomenclature}
	{\color{purple}{\textbf{Variance (spread)}}}
	\begin{itemize}
		\item	Measure of spread, or variability
		\item	Population variance:
	\end{itemize}

	$
	\begin{array}{rcl}
		\mathcal{V}\left\{x\right\} = \mathcal{E}\left\{ (x - \mu )^2\right\} &=& \sigma^2 \\
		&=& \displaystyle \frac{1}{N}\sum{(x-\mu)^2} \\
	\end{array}
	$
	\begin{itemize}
		\item	Sample variance:
	\end{itemize}
	$
	\begin{array}{rcl}
		s^2 &=& \displaystyle \frac{1}{n-1}\sum_{i=1}^{n}{(x_i - \bar{x})^2}
	\end{array}
	$

	{\color{purple}{\textbf{Degrees of freedom}}}: the denominator term, $n-1$
\end{frame}

\begin{frame}\frametitle{Quick review of basic nomenclature}
	\begin{block}
		{\color{purple}{{\textbf{Outlier}}}}
		\begin{center}
			\emph{A point that is unusual, given the context of the surrounding data}
		\end{center}
	\end{block}
	\begin{itemize}
		\item	4024, 5152, 2314, 6360, 4915, 9552, 2415, 6402, 6261
		\item	4, 61, 12, 64, 4024, 52, -8, 67, 104, 24
	\end{itemize}

	\vspace{12pt}
	\begin{itemize}
		\item	$\bar{x} = 440.4$  and $s = 1260$
		\item	Median = 56.5 and MAD = 57
	\end{itemize}
\end{frame}

\begin{frame}\frametitle{Quick review of basic nomenclature}
	{\color{purple}{\textbf{Median (location)}}}
	\begin{itemize}
		\item	One alternative measure of location.
		\item	Robust statistic: insensitive (robust) to outliers in the data
		\item	The most robust estimator of sample location:
		\begin{itemize}
			\item	breakdown of 50\%: means 50\% of the data contaminated before median breaks down
			\item	breakdown of the mean: $1/n$\% - only 1 bad data point required
		\end{itemize}
	\end{itemize}
	\begin{exampleblock}{Why should you care?}
		Always use the median and MAD (described next) where possible.
	\end{exampleblock}
\end{frame}

\begin{frame}\frametitle{Some newer terms?}
	{\color{purple}{\textbf{Median absolute deviation, MAD (spread)}}}
	\begin{itemize}
		\item	A robust measure of spread
	\end{itemize}
	$ \qquad\qquad \text{mad}\left\{ x_i \right\} = c \cdot \text{median}\left\{ \| x_i - \text{median}\left\{ x_i \right\} \| \right\} $
	\begin{itemize}
		\item	The constant $c = 1.4826$ makes MAD consistent with the standard deviation if $x_i$ are normally distributed.
		\item	Breakdown of MAD: 50\%
		\item	Breakdown of standard deviation: $1/n$\%
	\end{itemize}
	\begin{itemize}
		\item	Read the paper by Rousseeuw (especially grad students):
		\iftoggle{hrymak}{
			\begin{itemize}
				\item	\href{http://literature.connectmv.com/item/173}{http://literature.connectmv.com/item/173}
			\end{itemize}
		}{
			\begin{itemize}
				\item	\href{http://learnche.mcmaster.ca/media/mcmaster/Rousseeuw-tutorial.pdf}{http://learnche.mcmaster.ca/media/mcmaster/Rousseeuw-tutorial.pdf}
			\end{itemize}
		}
	\end{itemize}
	\begin{itemize}
		\item	Another robust estimate of spread: the IQR (inter-quartile range)
		\item	but not as robust as the MAD
	\end{itemize}
\end{frame}

\begin{frame}\frametitle{We will take a look at some basic distributions next}
	\begin{itemize}
		\item	Just a review; please read any of the books listed on the website for more details
		\item	Focus on when to use the distribution
		\item	And how the distribution looks
	\end{itemize}
\end{frame}

\begin{frame}\frametitle{The \textbf{binary} (Bernoulli distribution)}
	For binary events: \textbf{event A} and \textbf{event B}
	\begin{columns}
		\column{6cm}
		\begin{itemize}
			\item	Pass/fail, or yes/no system
			\item	$p(\text{pass}) + p(\text{fail}) = 1$
			\item	Example: what is probability of seeing this sequence if $p(\text{pass}) = 0.7$
			\item	pass, pass, pass
			\begin{itemize}
				\item	$(0.7)(0.7)(0.7) = 0.343$
				\item	\emph{only if each ``pass'' or ``fail''} is independent of the other samples
			\end{itemize}
			\item	pass, fail, pass, fail, pass, fail
			\begin{itemize}
				\item	$(0.7)(0.3)(0.7)(0.3)(0.7)(0.3)$
			\end{itemize}
		\end{itemize}
		\column{5cm}
		\includegraphics[width=\textwidth]{\imagedir/univariate/histogram-70-30.png}
	\end{columns}
\end{frame}

\begin{frame}\frametitle{The \textbf{uniform} distribution}
	Each outcome is equally as likely to occur as all the others. The classic example is dice: each face is equally as likely.

	Probability distribution for an event with 4 possible outcomes:
	\begin{center}
		\includegraphics[width=0.5\textwidth]{\imagedir/univariate/histogram-4-cuts.png}
	\end{center}
\end{frame}

\begin{frame}\frametitle{The \textbf{normal} distribution}

	We need to look at two topics first:
	\begin{enumerate}
		\item	The Central Limit Theorem
		\item	The concept of \emph{Independence}
	\end{enumerate}
\end{frame}

\begin{frame}\frametitle{Central limit theorem, \textbf{and} the important assumptions that go with it}
	\begin{block}
		{\color{purple}{Central limit theorem}}
		\begin{center}
			The average of a sequence of values from \emph{any distribution} will approach the normal distribution, provided the original distribution has finite variance.
		\end{center}
	\end{block}

	\includegraphics[width=0.9\textwidth]{\imagedir/univariate/CLT-derivation.png}

	\textbf{Assumption}: the samples used to compute the average are independent.
	\vspace{4pt}
	\hrule
	\vspace{4pt}
	How is this theorem relevant with engineering data?
	\begin{itemize}
		\item	In many cases we are using averages from our raw data
		\item	These raw data are almost never normally distributed
		\item	But using the assumption that these averages are normally distributed is powerful
	\end{itemize}
\end{frame}

\begin{frame}\frametitle{Central limit theorem: throwing dice {\color{myGreen}{(independent events)}}}
	\includegraphics[width=\textwidth]{\imagedir/univariate/simulate-CLT.png}
\end{frame}

\begin{frame}\frametitle{The assumption of \textbf{independence} is widely used}

	It is a condition for the central limit theorem.
	\begin{block}
		{\color{purple}{Independence}}
		\begin{center}
			The samples are \emph{randomly} taken from a population. If two samples are independent, there is no possible relationship between them.
		\end{center}
	\end{block}

	We frequently violate this assumption in engineering work. Discuss these examples:
	\begin{itemize}
		\item	A questionnaire is given to students. Are the answers independent if students discuss the questionnaire prior to handing it in?
		\item	The snowfall, recorded in inches, for the last 30 days.
		\item	Snowfall, recorded on 3 January for every year since 1976: independent or not?
		\item	The impurity values in the last 10 batches of product produced.
		\item	Exercise in the notes about independence (pump failures).
	\end{itemize}
\end{frame}

\begin{frame}\frametitle{Which one of these series have independent values in time?}
	\begin{columns}[t]
		\column{1.2\textwidth}
			\includegraphics[width=1.\textwidth]{\imagedir/univariate/simulate-independence.png}
		\column{0.20\textwidth}
	\end{columns}
	
	% The middle one
	% The lower one is negatively autocorrelated (and the strongest negative autocorr)
	% The first ine is weakly positively autocorrelated
\end{frame}

\begin{frame}\frametitle{We seldom use the normal distribution's equation directly}

	$ p(x) = \displaystyle \frac{1}{\sqrt{2\pi \sigma^2}}e^{-\frac{\displaystyle (x-\mu)^2}{\displaystyle 2\sigma^2}} $
	\begin{center}
		\includegraphics[width=0.85\textwidth]{\imagedir/univariate/normal-distribution-standardized.png}
	\end{center}
	\begin{itemize}
		\item	$x$: variable of interest
		\item	$p(x)$ probability of obtaining that $x$
		\item	$\mu$ population mean for variable $x$
		\item	$\sigma$ population standard deviation (positive)
		\item	the distribution is symmetric about $\mu$
	\end{itemize}
\end{frame}

\begin{frame}\frametitle{But we do use the properties of the normal distribution frequently}
	\begin{enumerate}
		\item	The maximum value of $p(x)$ occurs at $x = \mu$
		\item	What happens to the shape of $p(x)$ as $\sigma$ gets larger ?
		\item	What happens to the shape of $p(x)$ as $\sigma \rightarrow 0$ ?
		\item	Fill out this table:
	\end{enumerate}
	\begin{center}
		\begin{tabular}{|c|c|c|c|}\hline
			$\mu$		&	$\sigma$ 	&	$x$		&	$p(x)$\\ \hline
			0 			&  	1			& 	0 		&		  \\ \hline
			0 			&	1			& 	1		&  		  \\ \hline
			0 			&	1			& 	-1		&  		  \\ \hline
		\end{tabular}

	\end{center}
	Some rules of thumb:
	\begin{itemize}
		\item	$\sigma$ is the distance from the mean to the point of inflection
		\item	the area from $-\sigma$ to $\sigma$ is about 70\% (68.3\% exactly) of the distribution, with about 15\% outside the $\pm \sigma$ tails
		\item	the tail area outside $\pm 2\sigma$ is about 5\% (2.275 outside each tail)
	\end{itemize}
\end{frame}

\begin{frame}\frametitle{And derive values from the (unit) normal distribution in tables}
	\begin{itemize}
		\item	Used for tables of normal distribution, and cumulative area
		\item	We will use software instead in the course
		\item	The {\color{purple}{standard form}} is defined as $ z = \displaystyle \frac{x - \mu}{\sigma} $
		\item	${\color{red} \mathbf{x}} \sim \mathcal{N}(\mu, \sigma^2)$ \hfill {\color{myOrange}{$\longleftarrow$ do not use the standard deviation, use $\sigma^2$}}
		\item	$z \sim \mathcal{N}(0.0, 1.0)$ after standardization
		\item	Units of $z$ if $x$ were measured in \texttt{kg}, for example?
		\item	Standardization allows us to straightforwardly compare 2 variables that have different means and spreads
		\item	{\color{red}{Recommendation}}: make sure you can read a statistical table (see end of Chapter 2)
	\end{itemize}
\end{frame}

\begin{frame}\frametitle{Statistical tables (graphical)}
	\begin{center}
		\includegraphics[width=\textwidth]{\imagedir/univariate/Statistical-tables/show-pnorm-and-qnorm.png}
	\end{center}
\end{frame}

\begin{frame}\frametitle{Statistical tables (tabular)}
	\begin{center}
		\includegraphics[width=\textwidth]{\imagedir/univariate/Statistical-tables/cumulative-normal-table.png}
	\end{center}
\end{frame}

\begin{frame}\frametitle{In-class exercise}
	\begin{enumerate}
		\item	Assume x = biological activity of a drug, $x \sim \mathcal{N}(26.9, 9.3)$. Probability of $x \leq 30.0$?
		\vspace{24pt}
		\item	Assume x = yield from batch process:
				$$x \sim \mathcal{N}(85 \text{~g/L}, 16\, \text{g}^2\text{.L}^{-2})$$
				What is the proportion of yield values between 77 and 93 g/L?
	\end{enumerate}
	
	% Q1: sigma = 3 units; mean = 27; so at 30, we cover all the area between +/- 1sigma, which is 70%. Then add the area from -Inf to -1Sigma = 15%, to 70% + 15% = 85% (or read it from the tables where z=+1)
	% Q2: sigma = 4 units; mean = 85; z = (77-85)/4 = -2;   z=(93 - 85)/4 = 8/4 =2; so area between -2 and +2 = 95%
\end{frame}

\begin{frame}\frametitle{Checking for normality. Is this sample normally distributed?}
	\begin{center}
		\includegraphics[width=0.85\textwidth]{\imagedir/univariate/check-if-normal.png}
	\end{center}
	``Normally distributed'' is a widely used assumption. How do we check for it? Use a \textbf{q-q plot}
\end{frame}

\begin{frame}\frametitle{Inverse cumulative distribution function (inverse CDF)}
	\begin{itemize}
		\item	\textbf{Cumulative distribution}: area underneath the distribution function
		\item	\textbf{Inverse cumulative distribution}: we know the area, but want to get back to the value along the $z$-axis.
		\item	We use this to construct a q-q plot
	\end{itemize}
	\vspace{-4pt}
	\begin{center}
		\includegraphics[width=0.80\textwidth]{\imagedir/univariate/show-pnorm-and-qnorm.png}
	\end{center}
\end{frame}

\begin{frame}\frametitle{Checking for normality using q-q plots}

	\textbf{Approach}: {\color{myOrange}{compare}}
	\begin{itemize}
		\item	the theoretical distribution against
		\item	the actual data.
	\end{itemize}
	Let's say you have $N$ \emph{actual} observations you want to check.

	\vspace{16pt}
	1. Create $N$ theoretical observations that are normally distributed:
	\begin{itemize}
		\item	Sort from lowest to highest
		\item	The first point is the 1/$N$ x 100 percentile
		\item	The second point is the 2/$N$ x 100 percentile
		\item	Mid-point: 0.5 x 100 = 50th percentile
		\item	Last point: 1.0 x 100 = 100th percentile
		\item	Calculate the theoretical points along the distribution function
	\end{itemize}
\end{frame}

\begin{frame}[fragile]\frametitle{Checking for normality using q-q plots}

	\begin{lstlisting}[R]
	N = 10
	index <- seq(1, N)       #    1,    2, ...   10
	P <- (index - 0.5) / N   # 0.05, 0.15, ... 0.95
	theoretical.quantity <- qnorm(P)
	# [1] -1.64 -1.04 -0.674 -0.385 -0.126
	#      0.125  0.385  0.6744 1.036  1.64
	\end{lstlisting}
	\begin{center}
		\includegraphics[width=.7\textwidth]{\imagedir/univariate/show-pnorm-and-qnorm.png}
	\end{center}
\end{frame}

\begin{frame}[fragile]\frametitle{Checking for normality using q-q plots}
	2. Do the same now for the actual data points
	\begin{itemize}
		\item	Standardize data first: subtract mean, divide by standard deviation
		\item	Sort the data from lowest to highest
		\item	These points should now match the theoretical points from step 1 if they are normally distributed. \emph{{\color{myGreen}{Example}}}:
	\end{itemize}

	\begin{lstlisting}[R]
	yields <- c(86.2, 85.7, 71.9, ... 86.9, 78.4)
	mean.yield <- mean(yields)          # 80.0
	sd.yield <- sd(yields)              # 8.35

	yields.z <- (yields - mean.yield)/sd.yield
	#[1] 0.734  0.674 -0.978  1.82 ... -0.140 0.818 -0.2

	yields.z.sorted <- sort(yields.z)
	#[1] -1.34 -1.04  -0.978 -0.355 ... 0.734 0.818  1.82

	theoretical.quantity   # see prior slide
	#[1] -1.64 -1.04  -0.674 -0.385 ... 0.674 1.036  1.64
	\end{lstlisting}
\end{frame}

\begin{frame}\frametitle{Checking for normality using q-q plots}

	3. Plot theoretical quantities against the actual quantities.
	\begin{itemize}
		\item	\small They should form a 45 degree line if truly from this distribution
	\end{itemize}
	\begin{center}
		\includegraphics[width=0.69\textwidth]{\imagedir/univariate/qqplot-derivation.png}
	\end{center}
\end{frame}

\begin{frame}\frametitle{Checking for normality using q-q plots}
	\begin{enumerate}
		\setcounter{enumii}{4}
		\item	Usually we unscale the $y$-axis (the axis with actual data) so we can see the original data values
		\item	The \texttt{car} library: adds 95\% confidence bounds that should contain the data points
	\end{enumerate}
	\vspace{-8pt}
	\begin{center}
		\includegraphics[width=\textwidth]{\imagedir/univariate/qqplot-from-R.png}

		\begin{columns}[t]
			\column{0.35\textwidth}
				\texttt{qqplot(...)}\\
				\texttt{qqline(...)}
			\column{0.35\textwidth}
				\texttt{library(car)}\\
				\texttt{qqPlot(...)}
		\end{columns}
	\end{center}

\end{frame}

\begin{frame}\frametitle{Example: the 4N4 exam grades from 2013}
	\begin{center}
		\includegraphics[height=.9\textheight]{\imagedir/univariate/qqplot-4N4-2013-grades.png}
	\end{center}
\end{frame}

\begin{frame}\frametitle{Summary so far}

	Aim of this section: \emph{quantify} variability, then \emph{compare} variability
	\begin{itemize}
		\item	Location, Spread, Histograms (distributions)
		\item	Populations and Samples
		\item	Robustness: median and MAD
		\item	Independence
	\end{itemize}
\end{frame}

\begin{frame}\frametitle{Recap}
	\begin{itemize}
		\item	Probability: cumulative area under the distribution curve
		\item	Rules of thumb
		\begin{itemize}
			\item	Area from $-\sigma$ to $\sigma$ is about 70\%, with 15\% outside the $\pm \sigma$ tails
			\item	Area outside $\pm 2\sigma$ is about 5\% (2.275 outside each tail)
		\end{itemize}
	\end{itemize}
	\begin{center}
		\includegraphics[width=0.65\textwidth]{\imagedir/univariate/normal-distribution-standardized.png}
	\end{center}
	\vspace{-16pt}
	\begin{itemize}
		\item	Independence
		\item	$z$-value: $z = \dfrac{x - \text{mean}}{\text{standard deviation}}$
		\vspace{8pt}
		\begin{itemize}
			\item	\textbf{Does not} change the distribution's characteristics
			\item	e.g. positive skew in $x$ will still show in $z$
		\end{itemize}
		\item	Testing for normality (q-q plot)
	\end{itemize}
\end{frame}

\begin{frame}\frametitle{Central limit theorem: extended}

	\includegraphics[width=\textwidth]{\imagedir/univariate/CLT-derivation.png}

	\textbf{Assumes}: the samples used to compute the average are independent.
	\begin{enumerate}
		\item	Estimate of population mean: $\bar{x} = \displaystyle \dfrac{1}{n} \sum_i^{i=n}{x_i}$
		\item	Estimate of population variance: $s^2 =\displaystyle \frac{1}{n-1}\sum_i^{i=n}{(x_i - \bar{x})^2}$
		\item	\textbf{New:} $\bar{x} \sim \mathcal{N}\left(\mu, \sigma^2/n \right)$  \hfill{\color{myGreen}{$\longleftarrow$ really powerful statement}}
	\end{enumerate}

	\emph{Interpretation}

	Repeated estimates of mean are unbiased (tends towards $\mu$ {\color{myOrange}{$\leftarrow$ of the distribution sampled from!}}) and variance of that estimate is decreased. 
	\emph{Computer demo.}
	
	% 
\end{frame}

\begin{frame}\frametitle{Central limit theorem: extended}
	\begin{itemize}
		\item	$x \sim \mathcal{N}\left(\mu, \sigma^2 \right)$
		\item	$\bar{x} \sim \mathcal{N}\left(\mu, \sigma^2/n \right)$
	\end{itemize}
	\includegraphics[width=\textwidth]{\imagedir/univariate/explain-confidence-interval.png}
\end{frame}

\begin{frame}\frametitle{Example}

	Large bale of polymer composite: 9 independent samples taken and sent to lab. The samples are independent estimates of the bale's entire (population) viscosity.
	\begin{itemize}
		\item	$x$ = \texttt{23, 19, 17, 18, 24, 26, 21, 14, 18}
		\item	$\bar{x} = 20.0$
		\item	$s=3.81$
		\item	Assume we know the population $\sigma=3.5$
	\end{itemize}
	\myhrule
	1. Distribution of the sample mean? Parameters of that distribution?
	\begin{itemize}
		\item$\bar{x} \sim \mathcal{N}\left(\mu, \sigma^2/n \right)$
	\end{itemize}
	2. Construct a $z$-value for $\bar{x} $
	\begin{itemize}
		\item	This seems strange at first, but remember $z = \dfrac{x - \text{mean}}{\text{standard deviation}}$
		\item	$z = \dfrac{\bar{x} -\mu}{\sigma/\sqrt{n}}$
	\end{itemize}
\end{frame}

\begin{frame}\frametitle{Example}
	3. The $z$-value is \emph{not known} (why?). So rather find lower and upper bounds that will contain 95\% of all possible $z$-values.
	\begin{itemize}
		\item	$-c_n < z < +c_n$
		\item	$c_n$ = \texttt{qnorm(1-0.05/2)}       \qquad\# $c_n$ is 1.96
	\end{itemize}
	\begin{center}
		\includegraphics[width=0.80\textwidth]{\imagedir/univariate/show-pnorm-and-qnorm.png}
	\end{center}
\end{frame}

\begin{frame}\frametitle{Example}
	4. Unpack the $z$-value inside those bounds, solving for the unknown population mean, $\mu$
	\begin{itemize}
		\item	$\displaystyle - c_n < \frac{\bar{x} - \mu}{\sigma / \sqrt{n}} < c_n$
		\item	$\displaystyle \bar{x} - c_n\frac{\sigma}{\sqrt{n}} < \mu < \bar{x} + c_n\frac{\sigma}{\sqrt{n}}$
		\item	$\displaystyle \text{LB} < \mu < \text{UB}$
	\end{itemize}

	This interval contains, with 95\% confidence, the population mean of the bale's viscosity.
	\begin{itemize}
		\item	Substitute in $\sigma = 3.5$, $c_n$ = \texttt{qnorm(1 - 0.05/2)} = 1.96
		\item	$17.71 < \mu < 22.29$
	\end{itemize}
\end{frame}

\begin{frame}\frametitle{What about the variance, $\sigma$?}

	What if we do not know $\sigma$ (more realistic case)?

	We use $s$ as an estimate of $\sigma$.
	\begin{itemize}
		\item	Create $z = \dfrac{\bar{x} - \mu}{s/\sqrt{n}}$.
		\item	This $z$ value follows the $t$-distribution:
		\begin{itemize}
			\item	because we have used an \emph{estimate} of $\sigma$
			\vspace{6pt}
			\item	Note: $z = \dfrac{\bar{x} - \mu}{\sigma/\sqrt{n}} \sim \mathcal{N}(\mu, \sigma^2/n)$
			\vspace{6pt}
			\item	However: $z = \dfrac{\bar{x} - \mu}{s/\sqrt{n}} \sim t\left(\nu\right)$
			\vspace{6pt}
			\item	\textbf{but we must make a new assumption} if we use $s$: the $x_i$ (raw data) values are normally distributed
		\end{itemize}
		\item	Note that we use $s/\sqrt{n}$ instead of $\sigma/\sqrt{n}$ in the equation for $z$
	\end{itemize}
\end{frame}

\begin{frame}\frametitle{The $t$-distribution is heavily used in this course}
	\begin{center}
		\includegraphics[width=\textwidth]{\imagedir/univariate/t-distribution-comparison.png}
	\end{center}
	It has one parameter: $z \sim t\left(\nu\right)$, where $\nu = n-1$ = degrees of freedom.
	\begin{center}
		\includegraphics[width=0.9\textwidth]{\imagedir/univariate/t-distribution-derivation.png}
	\end{center}
\end{frame}

\begin{frame}\frametitle{Bale example: applying the $t$-distribution}
	\begin{enumerate}
		\item	Construct the $z$-value for the sample average, assuming we don't know $\sigma$
		\begin{itemize}
			\item	$z = \dfrac{\bar{x} - \mu}{s/\sqrt{n}}$
		\end{itemize}
		\item	What distribution does this $z$-value follow? Be specific.
		\begin{itemize}
			\item	$t$-distribution, with 8 degrees of freedom
		\end{itemize}
		\item	Check that $x_i$ raw data follow a normal distribution (assumption)
		\item	Find lower and upper bounds for interval that spans 95\% of the area of this distribution.
		\begin{itemize}
			\item	\texttt{qt(0.025, df=8) = - 2.31} and \texttt{qt(0.975, df=8) = +2.31}
		\end{itemize}
		\item	Substitute the $z$-value into the interval. What is the interval for the population mean this time?
		\begin{itemize}
			\item	$\bar{x} - c_t\dfrac{s}{\sqrt{n}} < \mu < \bar{x} + c_t\dfrac{s}{\sqrt{n}}$
		\end{itemize}
		\item	Compare the two intervals (one with $\sigma$, one with $s$)
		\begin{itemize}
			\item	Using $\sigma$: $17.71 < \mu < 22.29$
			\item	Using $s$: $17.06 < \mu < 22.93$
		\end{itemize}
	\end{enumerate}
\end{frame}

\begin{frame}\frametitle{Confidence Intervals: we have to be \textbf{very} comfortable with this concept}

	\textbf{Example}: a customer is evaluating your product: want a confidence interval (CI) for the impurity level in your sulphuric acid.

	$\qquad$

	\emph{Response}: the range from 429ppm to 673ppm contains the true impurity level with 95\% confidence.

	$\qquad$

	This is a compact representation of the impurity level.

	$\qquad$

	\emph{Alternatively} you could have said:
	\begin{itemize}
		\item	the sample mean from the last year of data is 532 ppm
		\item	the sample standard deviation from the last year of data is 102 ppm [can be back-calculated if we were told $n$]
		\item	the last year of data are normally distributed
	\end{itemize}

	A CI conveys this information equally well, and in a useful manner.
\end{frame}

\begin{frame}\frametitle{Confidence Intervals}

	Previous example: the interval that contains the population viscosity was:

	$
	\begin{array}{rcccl}
		- c_t &\leq& \displaystyle \frac{\bar{x} - \mu}{s/\sqrt{n}} &\leq& +c_t\\
		\bar{x} - c_t \displaystyle \frac{s}{\sqrt{n}} &\leq& \mu &\leq& \bar{x} + c_t\displaystyle\frac{s}{\sqrt{n}} \\
		\text{LB} &\leq& \mu &\leq& \text{UB}
	\end{array}
	$
	\begin{itemize}
		\item	$c_t = 2.31$ for the 95\% confidence interval with 8 degrees of freedom
		\item	LB = $20.0 - 2.92 = 17.1$
		\item	UB = $20.0 + 2.92 = 22.9$
	\end{itemize}
\end{frame}

\begin{frame}\frametitle{Interpreting the {\color{purple}{confidence interval (CI)}}}
	\begin{itemize}
		\item	The CI {\color{myOrange}{\textbf{does not imply}}} that $\bar{x}$ lies in the interval from LB to UB, i.e. the CI is not about $\bar{x}$; it's about $\mu$
		\item	\textbf{Incorrect} to say: (sample) average viscosity is 20 units and lies inside the range of 17.1 to 22.9 with a 95\% probability
		\item	The CI \textbf{does imply} that $\mu$ is expected to lie within that interval with a given level of confidence
		\item	The {\color{myGreen}{\textbf{CI is a range}}} of possible values for $\mu$, not for $\bar{x}$
		\item	UB and LB are a function of the data sample
		\item	If we take a different sample of data, we will get different bounds
		\item	How should the confidence level (probability) be interpreted?
		\begin{itemize}
			\item	\textbf{IT IS}: Probability that the \emph{CI range} contains the true population viscosity, $\mu$
			\item	\textbf{IT IS NOT}: Probability that the true population viscosity, $\mu$ is within the given range {\color{myOrange}{[{\scriptsize $\mu$ doesn't have a probability; it is 100\% fixed}]}}
			\item	If confidence level is 95\%, then 5\% of the time the interval \emph{will not contain} the true mean
			\item	Collect 20 sets of samples, 19 times out of 20 the CI range will contain the true mean
		\end{itemize}
	\end{itemize}
\end{frame}

\begin{frame}\frametitle{Interpreting the confidence interval at different \% values}

	\begin{center}
		\begin{tabular}{c|cc}
			\textbf{Confidence level} & \textbf{LB} & \textbf{UB}\\ \hline
			90\%	& 17.6	&	22.4\\
			95\%	& 17.1	&	22.9\\
			99\%	& 15.7	&	24.2\\
		\end{tabular}
	\end{center}
	\begin{itemize}
		\item	What happens if the level of confidence is 100\%?
		\begin{itemize}
			\item	The confidence interval is then infinite.
			\item	We are 100\% certain this infinite range contains the population mean, however this is not a useful interval.
		\end{itemize}
		\item	What happens if we increase the value of $n$?
		\begin{itemize}
			\item	As $n$ increases, the confidence interval range decreases
			\item	but with diminishing returns (intuitively expected)
		\end{itemize}
	\end{itemize}
\end{frame}

\begin{frame}\frametitle{Summary: CI for a mean}

	Make 2 assumptions about the raw data: (\emph{always check these}!)
	\begin{enumerate}
		\item	$n$ independent points \hfill {\color{myOrange}{$\longleftarrow$ this assumption is always required}}
		\item	from the normal distribution \hfill {\color{myOrange}{$\longleftarrow$ not required if $\sigma$ known}}
	\end{enumerate}

	\vspace{12pt}

	There are 2 cases:
	\begin{enumerate}
		\item[(a)]	$\sigma$ is known, $\bar{x} \sim \mathcal{N}(\mu, \sigma^2/n)$, only requires assumption 1
		\item[(b)]	$\sigma$ unknown, use $s$, and if both assumptions are met, then $\bar{x} \sim t(\nu = n-1)$
	\end{enumerate}
\end{frame}

\begin{frame}\frametitle{Confidence interval calculations when the \textbf{variance is known}}
	\begin{itemize}
		\item	$z = \dfrac{\bar{x} - \mu}{\sigma/\sqrt{n}}$
		\item	z is normally distributed \textbf{if} the $x_i$ used to calculate $\bar{x}$ are independent
	\end{itemize}

	$
	\begin{array}{rcccl}
		- c_n &\leq& \displaystyle \frac{\bar{x} - \mu}{\sigma/\sqrt{n}} &\leq & +c_n\\
		\bar{x} - c_n \dfrac{\sigma}{\sqrt{n}} &\leq& \mu &\leq& \bar{x} + c_n\dfrac{\sigma}{\sqrt{n}} \\
		\text{LB} &\leq& \mu &\leq& \text{UB}
	\end{array}
	$

	$c_n$: from \texttt{qnorm(1 - 0.05/2) = 1.96} when at the 95\% confidence level
\end{frame}

\begin{frame}\frametitle{Confidence interval calculations when the \textbf{variance not known}}
	\begin{itemize}
		\item	$z = \dfrac{\bar{x} - \mu}{s/\sqrt{n}}$
		\item	$z$ is $t$-distributed \textbf{if} the $x_i$ used to calculate $\bar{x}$ are independent \textbf{and} normal
	\end{itemize}

	$
	\begin{array}{rcccl}
		- c_t &\leq& \displaystyle \frac{\bar{x} - \mu}{s/\sqrt{n}} &\leq & +c_t\\
		\bar{x} - c_t \dfrac{s}{\sqrt{n}} &\leq& \mu &\leq& \bar{x} + c_t\dfrac{s}{\sqrt{n}} \\
		\text{LB} &\leq& \mu &\leq& \text{UB}
	\end{array}
	$

	$c_t$: from \texttt{qt(1 - 0.05/2, df=...)} when at the 95\% confidence level
\end{frame}

\begin{frame}\frametitle{Comparison between the two cases: how does it matter?}

	You can prove the confidence interval with unknown variance is wider. (600-level students ... why?)
	\vspace{12pt}
	We expect this intuitively:
	\begin{itemize}
		\item	it reflects our uncertainty of the spread parameter: $s$ vs $\sigma$
		\item	leads to a more conservative result (i.e. wider bound)
	\end{itemize}
\end{frame}

\begin{frame}\frametitle{Many other confidence intervals exist}
	\begin{itemize}
		\item	They exist for population variances: $\text{LB} < \sigma^2 < \text{UB}$
		\begin{itemize}
			\item	ratio of two variances: \emph{do these 2 samples have same variance}?
		\end{itemize}

		\vspace{12pt}
		\item	They exist for proportions:
		\begin{itemize}
			\item	proportion of \emph{packaged pizzas with N or more pepperoni slices is between 86 and 92\%}
			\item	political polls: \emph{party XYZ has 35\% of the vote, the margin of error is 3\%; accurate 19 times out of 20}
		\end{itemize}
	\end{itemize}
\end{frame}

\begin{frame}\frametitle{Tests for differences and similarity}
	\begin{itemize}
		\item	Test a cheaper material, B. Does it work as well as A?
		\item	We want to introduce a new catalyst B. Does it improve our product properties over the current catalyst A?
		\item	Does the 407 route save time going home vs going to work?
	\end{itemize}
	\begin{itemize}
		\item	Sometimes we really don't need a test:
	\end{itemize}
	\begin{center}
		\includegraphics[width=0.8\textwidth]{\imagedir/univariate/system-comparison-boxplot-plots-obvious.png}
	\end{center}
\end{frame}

\begin{frame}\frametitle{Tests for differences and similarity}

	\textbf{Example}

	An engineer needs to verify that new feedback controller (B) on a batch reactor leads to improved yields. Compare with yields from feedback controller A.
	\begin{itemize}
		\item	There are 10 sequential runs with system A, then 10 runs with system B.
		\item	System B will cost us \$400,000 to install, and \$40,000 in annual software license fees.
		\item	A significant difference means \textbf{long-run implementation} of B will lead to an improved yield (not due to chance)
	\end{itemize}

	At the end of the trials you must make a recommendation to your engineering group: \textbf{A or B}.
\end{frame}

\begin{frame}\frametitle{Example for testing differences: the raw data acquired}
	\includegraphics[width=\textwidth]{\imagedir/univariate/system-comparison-wikitable.png}

	\vspace{12pt}
	$\bar{x}_B - \bar{x}_A = 82.93 - 79.89 = 3.04$
\end{frame}

\begin{frame}\frametitle{Example for testing differences: visualizing that data}
	\includegraphics[width=\textwidth]{\imagedir/univariate/system-comparison-boxplot-plots.png}
\end{frame}

\begin{frame}\frametitle{Example for testing differences: look at the data in context}
	\includegraphics[width=\textwidth]{\imagedir/univariate/system-comparison-sequence-plot.png}

	\vspace{0pt}
	Always request contextual data, if available: extremely valuable. {\color{myOrange}{Let's see why.}}
\end{frame}

\begin{frame}\frametitle{One approach: compare the data to a reference set}
	\begin{enumerate}
		\item	We have 300 previous batches operating with system A
		\begin{itemize}
			\item	Calculate average from batch 1, 2, 3, ... 10
			\item	Calculate average from batch 11, 12, 13, ... 20
		\end{itemize}
		\item	Subtract averages: (group average 11 to 20) minus (group average 1 to 10)
		\item	Shift and repeat steps 2 and 3: use batches (2 to 11) and (12 to 21)
		\item	Collect all the difference values
	\end{enumerate}
\end{frame}

\begin{frame}\frametitle{One approach: compare the data to a reference set}

	\includegraphics[width=\textwidth]{\imagedir/univariate/system-comparison-dotplot-grouped.png}
	\begin{itemize}
		\item	31 historical differences out of 281 had a difference value higher than 3.04
		\item	11\% of historical batches had a better difference, by chance, not due to the feedback controller change.
		\item	Is this good enough?
	\end{itemize}

	\textcolor{red}{No assumption of independence, nor any form of distribution assumed!}
\end{frame}

\begin{frame}\frametitle{One approach: compare the data to a reference set}

	Data are not independent, they are autocorrelated.
	\begin{itemize}
		\item	$x_{k+1} = \phi x_{k} + a_k$
		\item	$\phi = -0.3$
		\item	$a_k \sim \mathcal{N}\left(\mu=0, \sigma^2=6.7^2\right)$
	\end{itemize}
	\begin{center}
		\includegraphics[width=0.6\textwidth]{\imagedir/univariate/system-comparison-autocorrelation-scatterplot.png}
	\end{center}
\end{frame}

\begin{frame}\frametitle{Second approach: when no reference data is available}
	\begin{itemize}
		\item	Don't have a \textbf{\emph{suitable}} reference, e.g. just the 20 runs
		\item	We will take a look at the method, and assumptions
	\end{itemize}

	\textbf{Aim}: is $\mu_B > \mu_A$ ? In other words, is $\mu_B - \mu_A > 0$?
	\begin{enumerate}
		\item	Assume data for case A and for case B are normally distributed
		\item	Assume data for sample A and sample B have $\sigma_A = \sigma_B = \sigma$
		\item	Define: sample A is from $\mathcal{N}\left(\mu_A, \sigma^2\right)$; sample B is from $\mathcal{N}\left(\mu_B, \sigma^2\right)$
		\item	From CLT (assumes data in A and data in B are independent):  {\color{myOrange}{\small $\longleftarrow$ but, but ... we know that's not true here!}}
			\begin{itemize}
				\item	$\mathcal{V}\left\{\bar{x}_A\right\} = \dfrac{\sigma^2_A}{n_A}$
				\item	$\mathcal{V}\left\{\bar{x}_B\right\} = \dfrac{\sigma^2_B}{n_B}$
			\end{itemize}
	\end{enumerate}

\end{frame}

\begin{frame}\frametitle{Second approach: when no reference data is available}

	5. Assume: $\bar{x}_A$ and $\bar{x}_B$ are independent, then {\scriptsize from your prior stats course you should know that ...}
	\begin{itemize}
		\item	$\mathcal{V}\left\{\bar{x}_B - \bar{x}_A\right\} = \dfrac{\sigma^2}{n_A} + \dfrac{\sigma^2}{n_B} = \sigma^2 \left(\dfrac{1}{n_A} + \dfrac{1}{n_B}\right)$
	\end{itemize}

	6. Create $z$-value:
	\begin{itemize}
		\item	$z = \dfrac{(\bar{x}_B - \bar{x}_A) - (\mu_B - \mu_A)}{\sqrt{\sigma^2 \left(\dfrac{1}{n_A} + \dfrac{1}{n_B}\right)}}$
	\end{itemize}

	\textbf{Question}: what is the probability of getting a $z$ \textbf{\emph{smaller}} than this?
\end{frame}

\begin{frame}\frametitle{{\color{myOrange}{Side discussion:}} Coating example. Independent?}

	Testing a new coating, B, to repel moisture. Coating is applied to packaging sheets to enhance the moisture barrier property. We take 1 sheet, divide into 16 squares, apply coating A and coating B as shown. Are 16 measured hydrophobicity values independent?
	\begin{center}
		\includegraphics[width=0.65\textwidth]{\imagedir/univariate/sheet-coating-application.png}
	\end{center}
\end{frame}

\begin{frame}\frametitle{{\color{myOrange}{Side discussion:}} Coating example. Independent?}
	\begin{itemize}
		\item	Sheet may not be uniform
		\begin{itemize}
			\item	\textbf{Randomly} assign blocks A and B on the packaging sheet
		\end{itemize}
		\item	Is the sheet representative?
		\begin{itemize}
			\item	Use sheets from different lots from the sheet supplier. Randomly apply A and B to different sheets.
		\end{itemize}
		\item	Couldn't we just measure $\text{B} - \text{A}$ (cancels out non-uniformity)?
		\begin{itemize}
			\item	Yes - paired experiment.
			\item	Note that $\bar{x}_A$ and $\bar{x}_B$ are not independent (violates step 5)
		\end{itemize}
	\end{itemize}
\end{frame}

\begin{frame}\frametitle{{\color{myOrange}{Side discussion:}} Cheaper raw material. Independent?}

	Aim: Testing an alternative, cheaper raw material in our process. Do not want the final product to change.
	\begin{itemize}
		\item	The dispensing system must be modified for raw material B, requiring production line to be shut down for 15 hours.
	\end{itemize}
	\begin{itemize}
		\item	The new supplier has given us 8 representative batches to test; each test takes 3 hours. Run these 8 batches over the weekend: set up dispenser on Friday night (15 hours), run tests from Saturday noon to Sunday noon (8 x 3 = 24hrs), then return the line back to normal for Monday's shift.
	\end{itemize}
	\begin{itemize}
		\item\textbf{\emph{How}} do we violate the assumptions of independence when comparing 8 batches of material A (taken from Thursday and Friday) to the 8 batches from material B (Saturday and Sunday)?
	\end{itemize}
\end{frame}

\begin{frame}\frametitle{{\color{myOrange}{Side discussion:}} Cheaper raw material. Independent?}
	\begin{itemize}
		\item	The 8 tests are run sequentially; \emph{any changes} in conditions between these 8 runs and the 8 runs from material A will be confounded
		\begin{itemize}
			\item	Staff running the equipment on the weekend
			\item	Dispenser change may have modified the process
			\item	Dispenser might be the improvement, not the new material
			\item	Samples waiting to be analyzed on Monday
		\end{itemize}
	\end{itemize}
	\begin{itemize}
		\item	Expensive and impractical to randomize in this case.
		\item	Suboptimal: Run \texttt{A A A A B B B B A A A A B B B B}
		\begin{itemize}
			\item	Run \texttt{A A A A B B B B} once
			\item	Run \texttt{B B B B A A A A} next
		\end{itemize}
		\item	Use a split-plot experimental design
	\end{itemize}

	\textcolor{red}{Randomization is the insurance (cost) we require to avoid being misled.}
\end{frame}

\begin{frame}\frametitle{Back to method 2 for testing differences}

	\textbf{Aim}: is $\mu_B > \mu_A$ ? In other words, is $\mu_B - \mu_A > 0$?

	1. Assume sample A $\sim \mathcal{N}\left(\mu_A, \sigma^2_A\right)$; assume B $\sim \mathcal{N}\left(\mu_B, \sigma^2_B\right)$

	2. Assume equal variance: $\sigma_A = \sigma_B = \sigma$

	3. So from CLT:
	\begin{itemize}
		\item	$\mathcal{V}\left\{\bar{x}_A\right\} = \dfrac{\sigma^2_A}{n_A}$ and $\mathcal{V}\left\{\bar{x}_B\right\} = \dfrac{\sigma^2_B}{n_B}$
	\end{itemize}

	4. Assume: $\bar{x}_A$ and $\bar{x}_B$ are independent - so combine variances:
	\begin{itemize}
		\item	$\mathcal{V}\left\{\bar{x}_B - \bar{x}_A\right\} = \dfrac{\sigma^2}{n_A} + \dfrac{\sigma^2}{n_B} = \sigma^2 \left(\dfrac{1}{n_A} + \dfrac{1}{n_B}\right)$
	\end{itemize}

	6. Create $z$-value. Ask: what is our probability/risk?
	\begin{itemize}
		\item$z = \dfrac{(\bar{x}_B - \bar{x}_A) - (\mu_B - \mu_A)}{\sqrt{\sigma^2 \left(\dfrac{1}{n_A} + \dfrac{1}{n_B}\right)}}$
	\end{itemize}

	7. Create confidence interval. Ask: does is span zero?
	\begin{itemize}
		\item$(\bar{x}_B - \bar{x}_A) - c_n \sqrt{\sigma^2 (\tfrac{1}{n_A} + \tfrac{1}{n_B})} < \mu_B - \mu_A < (\bar{x}_B - \bar{x}_A) + c_n \sqrt{\sigma^2 (\tfrac{1}{n_A} + \tfrac{1}{n_B})}$
	\end{itemize}
\end{frame}

\begin{frame}\frametitle{Method 2a: No reference set, external $\sigma$}

	If we have some \textbf{population estimate} for $\sigma$ (external estimate)

	$
	\begin{array}{rcl}
		z &=& \dfrac{(\bar{x}_B - \bar{x}_A) - (\mu_B - \mu_A)}{\sqrt{\sigma^2 \left( \dfrac{1}{n_A} + \dfrac{1}{n_B}\right)}} \\
		z &=& \dfrac{(82.93-79.89) - (\mu_B - \mu_A)}{\sqrt{6.61^2 \left( \dfrac{1}{10} + \dfrac{1}{10}\right)}} \genfrac{}{}{0pt}{0}{}{\genfrac{}{}{0pt}{0}{}{{\color{myOrange}{\leftarrow \,\text{used 300 samples for ``$\sigma$''}}}}} \\
		z &=& \dfrac{3.04 - 0}{2.956} = 1.03
	\end{array}
	$

	Probability of $z < 1.03$? The area from $-\infty$ up to 1.03 is 84.8\% using the \textbf{\emph{normal distribution}}.

	Chance of $\bar{x}_B - \bar{x}_A > 3.04$ is about 15\%. System B's performance could have been obtained by pure luck in 15\% of cases.
	\begin{itemize}
		\item	Confidence interval at 95\% level:
		\item	$3.04 - 1.96 \sqrt{6.61^2 \times \frac{2}{10}} < \mu_B - \mu_A < 3.04 - 1.96 \sqrt{6.61^2 \times \frac{2}{10}}$
		\item	{\color{myOrange}{$-2.75 < \mu_B - \mu_A < 8.83$}}
	\end{itemize}
\end{frame}

\begin{frame}\frametitle{Method 2b: No reference set, internal $\sigma$}
	\begin{itemize}
		\item	Sample variances: $s_A^2 = 6.81^2$ and $s_B^2 = 6.70^2$
		\item	It just happens that $n_A = n_B = 10$ (can have $n_A \neq n_B$)
		\item	Pool (combine) the variance, using a weighted sum:
	\end{itemize}

	$
	\begin{array}{rcccl}
		s_P^2 &=& \dfrac{(n_A -1) s_A^2 + (n_B-1)s_B^2}{n_A - 1 + n_B - 1} \\
		s_P^2 &=& \dfrac{9\times 6.81^2 + 9 \times 6.70^2}{18} = 45.63
	\end{array}
	$

	\vspace{8pt}
	$
	\begin{array}{rcccl}
		z &=& \dfrac{(82.93 - 79.89) - (\mu_B - \mu_A)}{\sqrt{s_P^2 \left(\dfrac{1}{10} + \dfrac{1}{10}\right)}} \\
		z &=& \dfrac{3.04 - 0}{\sqrt{45.63 \times 2/10}} = 1.01
	\end{array}
	$

	\vspace{8pt}
	Probability of $z < 1.01$? The area from $-\infty$ up to 1.01 is 83.7\% using the \textbf{\emph{t distribution}}.

	Chance of $\bar{x}_B - \bar{x}_A > 3.04$ is about 16\%. System B's performance could have been obtained by pure luck in 16\% of cases.
\end{frame}

\begin{frame}\frametitle{Summary of the second method: when no reference data is available}
	\begin{itemize}
		\item	Start with the definition for z, but use either an external or internal value estimate for $\sigma$
		\item	No external data required, only \textbf{strong} assumptions
		\begin{itemize}
			\item	Variances of A and B are comparable (allows us to pool them - method 2b)
			\item	Assume independence within each sample (not usually true in many practical cases)
			\item	Assume independence between the samples (more likely: data from A does not affect B)
			\item	Each sample, A and B, is assumed to be normally distributed
		\end{itemize}
	\end{itemize}
\end{frame}

\begin{frame}\frametitle{Summary of comparing for differences}

	Compare the 3 estimates (method 1, 2a, 2b): does controller B have better long-term performance than A?
	\begin{itemize}
		\item	\textbf{\emph{Base-case}}: controller B has no long-term effect; we need to prove - convincingly - it does.
	\end{itemize}

	What is the probability we are wrong is stating $\mu_B > \mu_A$?
	\begin{itemize}
		\item	1. Using only reference data: 11\% (about 1 in 10)
		\item	2a. Using 20 experimental runs with external estimate of $\sigma$: 15.2\% (about 1 in 7)
		\item	2b. Using 20 experimental runs with internal estimate of $\sigma$: 16.3\% (about 1 in 6)
	\end{itemize}

	Controller B could have been taken from historical data with a chance of 11\% \textbf{if the reference data are appropriate}

	Method 2 used only experimental (recent) data, but requires randomization (insurance policy - not always cheap)
\end{frame}

\begin{frame}\frametitle{What happens if the samples are not independent? [600-level]}
	Actual engineering data often violate the assumption of independence.

	Simulate sequences of lag-1 autocorrelated data, $x_{k+1} = \phi x_{k} + a_k$ with $a_k \sim \mathcal{N}\left(\mu=0, \sigma^2 = 25.0 \right)$ with length of $n=100$ samples
	\begin{itemize}
		\item[A]	Positively correlated: use $\phi > 0.7$
		\item[B]	Negatively correlated: use $\phi < -0.6$
		\item[C]	No correlation, i.e. totally independent data: $\phi = 0.0$
	\end{itemize}

	What happens to the standard deviation of $\bar{x}$? (simulate many sequences; calculate $\bar{x}$ then stddev, $s$, of these $\bar{x}$ values)

	\myhrule
	\emph{Solution}
	\begin{itemize}
		\item[A]	estimate of $s = 1.66$
		\item[B]	estimate of $s = 0.32$
		\item[C]	estimate of $s = 0.52$ {\color{myOrange}{{\scriptsize (close to the theoretical $\sqrt{\sigma^2/n} = \sqrt{25/100} = 0.5$)}}}
	\end{itemize}
\end{frame}

\begin{frame}\frametitle{Example of testing for differences: impellers}

	Axial and Radial impellers give different mixing times. The objective is to have the \emph{shortest} mixing times possible.
	\begin{center}
		\includegraphics[width=0.55\textwidth]{\imagedir/univariate/Mixing_-_flusso_assiale_e_radiale.jpg}
	\end{center}
	\vspace{-6pt}
	For these two different cases

	(a) $-453 \text{min} < \mu_\text{Axial} - \mu_\text{Radial} < 284 \text{min} $

	(b) $-21 \text{min} < \mu_\text{Axial} - \mu_\text{Radial} < 187 \text{min} $

	what is your recommendation for choosing the one impeller over the other?

	Consider the case of \emph{statistical difference} vs \emph{engineering difference}
\end{frame}

\begin{frame}\frametitle{Example: BOD}
	A system is assumed to have constant BOD during a time when multiple samples were taken using two different methods:
	\begin{center}
		\scalebox{0.80}{
				\begin{tabular}{c|c}
					\textbf{Dilution method} & \textbf{Manometric method}\\
					11 &             25\\
					26 &             3\\
					18 &             27\\
					16 &             30\\
					20 &             33\\
					12 &             16\\
					8  &             28\\
					26 &             27\\
					12 &             12\\
					17 &             32\\
					14 &             16\\
				\end{tabular}
		}
	\end{center}
	\begin{columns}[t]
		\column{0.50\textwidth}
			BOD via \textbf{Dilution method}:
			\begin{itemize}
				\item	$\bar{x}_D = 16.4$
				\item	$s_D = 5.9$
				\item	$n_D$ = 11 samples
			\end{itemize}
		\column{0.50\textwidth}
			BOD via \textbf{Manometric method}:
			\begin{itemize}
				\item	$\bar{x}_M = 22.6$
				\item	$s_M = 9.5$
				\item	$n_M$ = 11 samples
			\end{itemize}
	\end{columns}
	\vspace{6pt}
	\emph{Verify that there is no difference between the two methods}
\end{frame}

\begin{frame}\frametitle{Example of testing for differences: BOD}
	\vspace{12pt}

	\hspace{-0.75cm}\includegraphics[height=.9\textheight]{\imagedir/univariate/BOD-comparison-raw-data-alternative.png}
\end{frame}

\begin{frame}\frametitle{Example of testing for differences: BOD}
	\begin{center}
		\includegraphics[height=.9\textheight]{\imagedir/univariate/BOD-comparison-raw-data.png}
	\end{center}
\end{frame}

\begin{frame}\frametitle{Paired tests: when are they used}

	Recall that the usual approach to test for differences is to keep all conditions as constant as possible and randomize experiments.

	$\quad$

	e.g. adjustable halogen lighting to maximize sales in a store
	\begin{itemize}
		\item	\textbf{A} = soft and dim lighting
		\item	\textbf{B} = brighter lighting (e.g. florescent)
	\end{itemize}

	e.g. raw material \textbf{A} or \textbf{B}
	\begin{itemize}
		\item	randomized choice of \textbf{A} or \textbf{B} is tough to implement
		\item	keeping all process conditions similar is next to impossible
	\end{itemize}

	$\quad$

	Requiring ``keep conditions as constant as possible'' leads to strange/impossible experiments:

	$\quad$

	e.g. drug trials for actual (\textbf{A}) vs placebo (\textbf{B})
	\begin{itemize}
		\item	you'd have to run all experiments on same person in random order
	\end{itemize}
\end{frame}

\begin{frame}\frametitle{Paired tests: how to recognize it when you come across it}

	A paired test: run experiment twice on the same object/batch/area. Also called ``two treatments''.
	\begin{itemize}
		\item	Drug trial: each person randomly receives placebo or drug for 3 weeks; later they receive the opposite, for another 3 weeks.
		\begin{itemize}
			\item	e.g. use blood pressure \textbf{difference} as the value
		\end{itemize}
		\item	Testing two additives, A and B, which is added to a base lubricant. Split base lubricant into 2 halves, for A and B
		\begin{itemize}
			\item	e.g. use difference of kinematic viscosity as the value
		\end{itemize}
		\item	Testing a paper coating to repel moisture. Coating is applied to randomly selected sheets
		\begin{itemize}
			\item	e.g. measure the repellent property difference as the value
		\end{itemize}
	\end{itemize}
\end{frame}

\begin{frame}\frametitle{Paired tests: when should you use one?}

	A paired test is appropriate when there is something in common within pairs of samples in group A and B.

	But the commonality is not between the pairs.

	$\qquad$

	Example: we want to test process \textbf{A} vs \textbf{B} to get a lower energy usage:

	$ \left.
	\begin{array}{r}
		\text{Material 1}\\
		\text{Material 2}\\
		\text{Material 3}\\
		\text{Material 4}\\
		\text{Material 5}
	\end{array}
	\right\} \text{only enough material for 2 experiments} $

	$\qquad$

	So a total of 10 experiments. How to allocate materials to runs \textbf{A} and \textbf{B}?
\end{frame}

\begin{frame}\frametitle{Paired tests: a comparison}

	1. \textbf{Sequentially} {\color{myOrange}{(don't ever do this!)}}

	$
	\begin{array}{cccccccccc}
		\text{A} & \text{A} & \text{A} & \text{A} & \text{A} & \text{B} & \text{B} & \text{B} & \text{B} & \text{B} \\
		1 & 1 & 2 & 2 & 3 & 3 & 4 & 4 & 5 & 5\\
	\end{array}
	$

	$\qquad$

	2. \textbf{Randomly} {\color{myOrange}{(commonality between pairs will not cancel)}}

	$
	\begin{array}{cccccccccc}
		\text{A} & \text{A} & \text{B} & \text{B} & \text{B} & \text{A} & \text{A} & \text{B} & \text{A} & \text{B} \\
		1 & 1 & 2 & 2 & 3 & 3 & 4 & 4 & 5 & 5\\
	\end{array}
	$

	$\qquad$

	3. \textbf{Paired} (but, run the 10 experiments in random order!)

	$
	\begin{array}{cccccccccc}
		\text{A} & \text{B} & \text{A} & \text{B} & \text{A} & \text{B} & \text{A} & \text{B} & \text{A} & \text{B} \\
		1 & 1 & 2 & 2 & 3 & 3 & 4 & 4 & 5 & 5\\
	\end{array}
	$

	$\qquad$

	Procedure followed for 1 and 2? $\text{LB} \leq \mu_A - \mu_B \leq \text{UB}$. How many DOF?

	Procedure for 3 shown next. How many DOF?
\end{frame}

\begin{frame}\frametitle{Paired tests}

	\textbf{Advantages}:
	\begin{itemize}
		\item	systematic (consistent) error from something that should be constant is cancelled out
		\begin{itemize}
			\item	person is constant between \textbf{A} vs \textbf{B}
			\item	lubricant in constant between additive \textbf{A} vs \textbf{B}
			\item	sheet is constant between coating \textbf{A} vs \textbf{B}
		\end{itemize}
		\item	measurement bias is cancelled out
		\item	only the paired differences must be normally distributed, not the raw data
		\item	we only have to ensure the pairs are independent of each other
	\end{itemize}

	$\quad$ \textbf{Disadvantage}: we loose degrees of freedom
\end{frame}

\begin{frame}\frametitle{Paired tests: so how do I analyze data from such a test?}

	\textbf{Question}: is the difference is significant, or is it essentially zero?
	\begin{enumerate}
		\item	Calculate the $n$ differences: $w = [w_1, w_2, \ldots, w_n]$
		\begin{itemize}
			\item	$w_i = x_{A,i} - x_{B,i}$  \qquad\emph{or}\qquad $w_i = x_{B,i} - x_{A,i}$
			\item	just be consistent in your calculation \emph{and} then your interpretation
		\end{itemize}
		\item	Assume these $n$ differences are independent
		\item	Calculate the mean $\overline{w}$ and standard deviation $s_{\overline{w}}$
		\item	$\overline{w}$ should be normally distributed (CLT): $\bar{w} \sim \mathcal{N}\left(\mu_w, \sigma_w^2/n \right)$
		\item	We are testing:
		\begin{itemize}
			\item	$\mu_w = \mu_A - \mu_B$  (if you calculated $w_i = x_{A,i} - x_{B,i}$)
			\item	$\mu_w = \mu_B - \mu_A$  (if you calculated $w_i = x_{B,i} - x_{A,i}$)
		\end{itemize}
	\end{enumerate}
\end{frame}

\begin{frame}\frametitle{Paired tests: so how do I analyze data from such a test?}

	Calculate the $z$-value, then confidence interval. Use the sample standard deviation (implies $t$-distribution)
	\begin{itemize}
		\item	$z = \dfrac{\overline{w} - \mu_w}{s_{\overline{w}} / \sqrt{n}}$
	\end{itemize}
	\begin{itemize}
		\item	$\overline{w} - c_t \dfrac{s_{\overline{w}}}{\sqrt{n}} < \mu_w < \overline{w} + c_t \dfrac{s_{\overline{w}}}{\sqrt{n}}$
	\end{itemize}
	\begin{itemize}
		\item	$c_t$: critical value from $t$-distribution with $n-1$ DOF at the required level of confidence. Use \texttt{qt(...)} function to find $c_t$
	\end{itemize}
\end{frame}

\begin{frame}\frametitle{Paired tests: back to the BOD example}

	Now assume the BOD samples were split in half and analyzed by each method.

	Why is a paired test appropriate?
	\begin{itemize}
		\item	E.g. contaminants in the sample affect both methods
		\item	E.g. delays from sample to analysis might influence BOD
	\end{itemize}
	\begin{itemize}
		\item	$\overline{w} = -6.27$
		\item	$s_{\overline{w}} = 11.77$
		\item	$n$ = 11 samples
	\end{itemize}

	\emph{Verify that there is no difference between the two methods}
\end{frame}

\begin{frame}\frametitle{Paired tests: back to the BOD example}
	\begin{center}
		\includegraphics[height=.9\textheight]{\imagedir/univariate/BOD-comparison-plot.png}
	\end{center}
\end{frame}

